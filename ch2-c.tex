%!TEX root = main.tex
\section{자연과 일상}
\subsection{일러두기}
일러둘 내용을 적습니다.

\subsection{자유발화 질문}
자유발화 질문을 적습니다.

\subsection{목표 어휘}
목표 어휘와 예상 형태, 질문을 적습니다. 다음과 같이 적습니다. \\

\Entry{
  word={가위바위보},
  pred={},
  feat={ms,sp,dp},
  desc={V2 /ㅟ/ V4 /ㅟ/},
  qstn={이기고 지는 것을 가르기 위해 이렇게 이렇게 (조사자들끼리 가위바위보를 하는 시늉을 함) 하는 것을 무엇이라고 하나요?},
  advq={실제로 가위바위보를 할 때에 어떻게 말하시나요? (억양 확인)}
}

\Entry{
  word={금긋다/금그어라},
  pred={긋다, 그셔라, 그스고, 근는다, 그꼬, 그스닝께, 그서서},
  feat={ms,sp,dp},
  desc={/금/(금속)과의 성조 음장 차이},
  qstn={땅따먹기 놀이를 할 때, 뼘을 잰 다음에는 바닥에 뭘 하나요? \\ (바닥에 금을 긋는 시늉을 하며) 이렇게 하는 걸 뭐라고 하나요?},
  advq={벽이나 담장이 갈라지면 어떻게 됐다고 하나요? (목표어휘: 금가다)}
}

\Entry{
  word={윷},
  pred={사짜},
  feat={ms,sp,dp},
  desc={기저 종성 /ㅊ/ \\ /ᅀᅲᆺ/, /늇/ > /윷/},
  qstn={명절에 나무토막 네 개를 던지면서 한 칸을 가거나 다섯 칸까지 가거나 하면서 노는데, 그것을 무엇이라고 하시나요?},
  advq={‘윷놀이’로 답변한 경우) 그때 던지는 막대기를 뭐라고 하시나요? \\ 이걸 던져서 나오는 눈들은 뭐라고 하시나요?}
}

\Entry{
  word={썰매},
  pred={썰매},
  feat={ms,sp,dp},
  desc={얼음/눈 위를 타는 것의 모양, 어형 차이 \\ 스케이트형이 나올 때 모양 차이 여부},
  qstn={얼음판이나 눈 위에서 타고 노는 것이 있는데, 뭘 탄다고 하시나요?},
  advq={썰매를 보통 어디에서 타곤 하나요? \\ 얼음판에서 타는 썰매랑 눈 위에서 타는 썰매가 다른가요?}
}

\Entry{
  word={얼레},
  pred={연자세},
  feat={ms,sp,dp},
  desc={/자새/형의 경우 대상물 확인 \\ 연줄/낚싯줄 감는 것 명칭 차이 여부},
  qstn={(*사진 자료 참고) 연을 날릴 때 실을 어디에 감으시나요?},
  advq={연줄 감는 거랑 낚싯줄 감는 것을 똑같이 얼레라고 하나요?}
}

\Entry{
  word={겨울},
  pred={겨울, 즑, 즐기 (/즐:기/ ‘겨울에’, /즑/은 처격조사와의 연결에서만)},
  feat={ms,sp,dp},
  desc={C1 + V1 /겨/의 구개음화 여부 \\ /겨ᅀᅳᆯ/, /겨을ㅎ/ > /겨울/},
  qstn={(*사진 자료 참고) 썰매 타고 하는 것은 어느 계절에 해요? \\ 가을 다음 계절을 뭐라고 하나요? (눈 오는 계절 금지)},
  advq={겨울철은 뭘 하면서 지내나요? 겨울철에 어떤 음식을 먹나요? \\ (반드시 조사 /-에/를 붙여 이야기하도록 답변을 유도한다.)}
}

\Entry{
  word={춥다},
  pred={춥다 (춥따, 춥꾸, 추:면, 추찌안타)},
  feat={ms,sp,dp},
  desc={/차다/와의 의미 차이},
  qstn={(C06)에는 날씨가 어떻다고 이야기하나요?},
  advq={추운 날씨가 오래 가면 올 겨울에는 무엇이 오래간다고 하나요? (목표 어휘: 추위)}
}

\Entry{
  word={차다},
  pred={차다(차기, 차니, 차서)},
  feat={ms,sp,dp},
  desc={/춥다/와의 의미 차이},
  qstn={얼음이 어떻다고 이야기하나요?(얼음을 만져 손이 시린 시늉을 하며) \\ 겨울철에는 물이 얼음장같이 어떠하다고 하죠.},
  advq={날이/날씨가 이렇다고 하기도 하나요? \\ 사람 성격을 가지고 이렇다고 하기도 하나요?}
}

\Entry{
  word={고드름},
  pred={고두룸, 고드름},
  feat={ms,sp,dp},
  desc={},
  qstn={겨울에 처마 밑에 주렁주렁 매달리는 어름덩어리는 무엇이라고 하시나요?},
  advq={이게 생기면 어떻게 하시나요? 없앨 때는 어떻게 없애나요?}
}

\Entry{
  word={눈(날씨)},
  pred={고두룸, 고드름},
  feat={ms,sp,dp},
  desc={/눈/(신체)과의 성조 음장 차이},
  qstn={비가 아니라, 하늘에서 하얗게 떨어져서 쌓이고 잘 뭉치는 것은 무엇인가요?},
  advq={눈이 내리면 뭘 하고 노나요?}
}

\Entry{
  word={소나기},
  pred={쏘내기},
  feat={ms,sp,dp},
  desc={},
  qstn={하늘이 맑다가 갑자기 비가 오면 뭐라고 하시나요? / 주로 여름에 별안간 세차게 쏟아지다가 곧 그치는 비를 무엇이라고 하시나요?},
  advq={집 밖에 있는데 소나기가 오면 어떻게 하세요? \\ 소나기를 맞으면 어떻게 옷이랑 몸을 말리나요?}
}

\Entry{
  word={벼락},
  pred={베락},
  feat={ms,sp,dp},
  desc={},
  qstn={하늘에서 번개가 치면서 나무나 사람에 떨어지면 무엇이 떨어진다고 하시나요? 갑자기 부자가 된 사람을 [이것] 부자라고도 해요.},
  advq={집 밖에 있는데 이게 치면 어떻게 하세요?}
}

\Entry{
  word={새벽},
  pred={새벽, 새복, 새벅},
  feat={ms,sp,dp},
  desc={},
  qstn={하루 중에서 해가 채 뜨지 않았을 무렵을 무엇이라고 하시나요? \\ 하루 중에서 아침이 밝기 직전일 때를 무엇이라고 하시나요?},
  advq={보통 몇 시까지를 새벽이라고 하세요? \\ 새벽에 잠이 깬 적 있으세요? 새벽에 깨면 몸이/기분이 어떠세요?}
}

\Entry{
  word={저녁},
  pred={저녁},
  feat={ms,sp,dp},
  desc={},
  qstn={하루 중에서 해가 막 질 무렵을 무엇이라고 하시나요?},
  advq={보통 몇 시부터를 저녁이라고 하세요? \\ 저녁 몇 시에 밥을 드세요?}
}

\Entry{
  word={밤(夜)},
  pred={밤},
  feat={ms,sp,dp},
  desc={/밤/(음식)과의 성조 음장 차이},
  qstn={하루 중에 해가 져서 캄캄할 때를 무엇이라고 하시나요?},
  advq={보통 몇 시부터를 밤이라고 하세요? \\ 보통 밤 몇 시에 주무시나요?}
}

\Entry{
  word={길다},
  pred={질다, 질:다, 길:다(지:러야, 기:러야)},
  feat={ms,sp,dp},
  desc={C1 + V1 /기/의 구개음화 여부},
  qstn={(임의로 길이가 다른 두 물체를 제시하며) 이건 이것보다 어때요?},
  advq={요즘 날씨는 해가 늦게 떠서 빨리 지잖아요. 이것을 다른 말로 뭐라고 그러나요? (목표 표현: 해가 길다/짧다)}
}

\Entry{
  word={짧다},
  pred={짧다(짤따, 짤븐, 짤버서)},
  feat={ms,sp,dp},
  desc={},
  qstn={(임의로 길이가 다른 두 물체를 제시하며) 이건 이것보다 어때요?},
  advq={요즘 날씨는 해가 늦게 떠서 빨리 지잖아요. 이것을 다른 말로 뭐라고 그러나요? (목표 표현: 해가 길다/짧다)}
}

\Entry{
  word={하루, 이틀, ... 열흘},
  pred={하루, 이틀, 사을/사흘, 나을/나흘, 닷새/다쎄, 엿새/여쎄, 이레, 여드레, 아으레.아흐레, 여를},
  feat={ms,sp,dp},
  desc={/렷새/ > /엿새/  /니레/, /닐헤/ > /이레/ \\ /알흐래/ > /아흔/},
  qstn={하루, 이틀, 하는 식으로 열흘까지만 천천히 날짜를 세주시겠어요?},
  advq={여행을 갔다 오는데 하루가 걸린다는 문장에, /하루/의 자리에 다른 날짜를 넣어서 천천히 한 번 더 말씀해 주시겠어요?}
}

\Entry{
  word={내일},
  pred={니얼,  (낼:, 내일)},
  feat={ms,sp,dp},
  desc={},
  qstn={오늘의 다음날은 무엇이라고 하시나요?},
  advq={혹시 내일 말고 다른 말도 있나요?}
}

\Entry{
  word={모레},
  pred={},
  feat={ms,sp,dp},
  desc={},
  qstn={(C28)의 다음날은 무엇이라고 하시나요?},
  advq={혹시 모레 말고 다른 말도 있나요?}
}

\Entry{
  word={글피},
  pred={글피},
  feat={ms,sp,dp},
  desc={},
  qstn={(C29)의 다음날은 무엇이라고 하시나요?},
  advq={혹시 글피 말고 다른 말도 있나요? / 자주 쓰시는 말인가요?}
}

\Entry{
  word={어제},
  pred={어제},
  feat={ms,sp,dp},
  desc={/어저께/와의 의미 차이},
  qstn={오늘의 전날은 무엇이라고 하시나요?},
  advq={혹시 어제 말고 다른 말도 있나요? \\ ‘어저께’라고 하면 같은 뜻인가요?}
}

\Entry{
  word={그저께},
  pred={그저끼},
  feat={ms,sp,dp},
  desc={/어저께/, /그제/와의 의미 차이 \\ /그젓긔/ > /그저께/},
  qstn={(C31)의 전날은 무엇이라고 하시나요?},
  advq={혹시 그저께 말고 다른 말도 있나요? \\ /그제/ 라고 하면 같은 뜻인가요?}
}

\Entry{
  word={그끄저께},
  pred={그끄저끼, 긋끄저끼},
  feat={ms,sp,dp},
  desc={},
  qstn={(C32)의 전날은 무엇이라고 하시나요?},
  advq={혹시 그끄저께 말고 다른 말도 있나요?}
}

\Entry{
  word={지렁이},
  pred={지렝이},
  feat={ms,sp,dp},
  desc={/거ᇫ워ᅀᅵ/, /것위/ 등 계통이 다른 어휘 \\ /디롱이/, /디룡이/ > /지렁이/},
  qstn={비가 오면 땅바닥에 나와 기어다니는 것을 무엇이라고 부르나요?},
  advq={주로 어디서 지렁이를 많이 보셨어요?}
}

\Entry{
  word={미꾸라지},
  pred={미꾸리, 미끄락지},
  feat={ms,sp,dp},
  desc={/믯구리/, /밋구리/ 등 이형태},
  qstn={길고 미끌미끌하고, 추어탕을 끓여 먹는 물고기는 무엇이라고 하시나요?},
  advq={미꾸라지는 어떻게 잡나요?}
}

\Entry{
  word={개구리},
  pred={개구락지, 깨구락지},
  feat={ms,sp,dp},
  desc={/먹자구/ 형과의 의미 차이에 유의},
  qstn={풀밭으로 나오기도 하고 물속에 들어가 살기도 하고, 폴짝폴짝 뛰어다니면서 우는 동물을 뭐라고 하시나요?},
  advq={뭔가를 시킬 때마다 반대로만 하는 사람을 뭐라고 하나요? (목표 어휘: 청개구리)}
}

\Entry{
  word={올챙이},
  pred={올챙이},
  feat={ms,sp,dp},
  desc={},
  qstn={(*사진 자료 참고) (C37)가 되기 전에 알애서 막 태어난 새끼를 무엇이라고 하시나요?},
  advq={올챙이는 어디에 가야 많이 볼 수 있나요?}
}

\Entry{
  word={두꺼비},
  pred={두께비},
  feat={ms,sp,dp},
  desc={/두텁/, /두터비/ 등 이형태},
  qstn={개구리(C37번 방언형)랑 비슷한데 조금 크고, 몸이 우둘투둘한 것은 무엇이라고 하시나요?},
  advq={개구리(C37번 방언형)랑 두꺼비랑 또 어떤 차이가 있나요? \\ 두꺼비집 노래는 어떻게 부르나요?}
}

\Entry{
  word={거머리},
  pred={그머리, 금자리, 금저리, 그:머리},
  feat={ms,sp,dp},
  desc={},
  qstn={논에서 일할 때 다리에 달라붙어서 피를 빨아먹는 것을 무엇이라고 하시나요?},
  advq={거머리가 다리에 달라붙으면/거머리한테 물리면 어떻게 하나요?}
}

\Entry{
  word={달팽이},
  pred={달팽이},
  feat={ms,sp,dp},
  desc={},
  qstn={등에 동글동글한 집을 달고 더듬이 두 개를 뻗고 기어다니는 동물은 무엇이라고 하시나요?},
  advq={논밭에 달팽이가 보일 때도 있나요? \\ 달팽이가 기는 속도를 어떻다고 하세요?}
}

\Entry{
  word={다슬기},
  pred={도슬비, 올뱅이},
  feat={ms,sp,dp},
  desc={},
  qstn={(C40)이랑 비슷한데 냇물이나 강물에서 사는 것은 무엇이라고 하시나요? 된장국에 넣어먹기도 하고, 강바닥의 바위틈에서 주울 수 있어요.},
  advq={다슬기는 보통 어떤 식으로/어떻게 잡나요?}
}

\Entry{
  word={우렁이},
  pred={올갱이},
  feat={ms,sp,dp},
  desc={},
  qstn={(C41)랑 비슷한데, 논에서 사는 것은 무엇이라고 하시나요? 조금 더 크고 둥글고 민물에 살고, 쌈밥이나 된장국으로 해먹기도 합니다.},
  advq={우렁이는 보통 어떤 식으로/어떻게 잡나요? \\ 우렁각시 이야기를 아시나요?}
}

\Entry{
  word={새끼(동물)},
  pred={새끼},
  feat={ms,sp,dp},
  desc={/새끼/(사물)와의 성조 음장 차이 \\ V1 /ㅐ-ㅔ/ 대립},
  qstn={송아지, 강아지, 망아지 같이 갓 태어난 어린 동물들을 무엇이라고 하시나요?},
  advq={소나 돼지가 새끼를 낳으면 무엇을 해줘야 하나요? / 새끼를 한 번에 여럿 낳을 때, 가장 먼저 나온 것을 뭐라고 하나요? (목표 어휘: 무녀리)}
}

\Entry{
  word={서캐},
  pred={서캐},
  feat={ms,sp,dp},
  desc={},
  qstn={머리에 생기는 이의 알은 무엇이라고 하시나요?},
  advq={서캐가 있으면 어떻게 없애나요?}
}

\Entry{
  word={구더기},
  pred={구데기, 구디기, 구:디기},
  feat={ms,sp,dp},
  desc={},
  qstn={(C46)가 알을 낳으면 태어나는 하얀 벌레를 뭐라고 하세요? ‘이것이 무서워서 장 못 담근다’는 속담이 있죠.},
  advq={집 안에 이것이 생기면 어떻게 하세요?}
}

\Entry{
  word={파리},
  pred={파리},
  feat={ms,sp,dp},
  desc={},
  qstn={음식에 날아 앉는 날벌레인데 (지시) 손을 이렇게 비비는 것을 무엇이라고 하시나요?},
  advq={이걸 잡는 데 쓰는 도구는 무엇이라고 하시나요? (목표 어휘: 파리채)}
}

\Entry{
  word={벼룩},
  pred={베룩},
  feat={ms,sp,dp},
  desc={},
  qstn={아주 조그마한데, 톡톡 튀어 다녀서 잡기가 어려운 벌레는 무엇이라고 하시나요?‘뛰어야 이것이지’라는 말이 있죠},
  advq={}
}

\Entry{
  word={벌},
  pred={부리, 벌:},
  feat={ms,sp,dp},
  desc={V1 /ㅓ/ 장단음},
  qstn={꿀을 만드는 곤충인데, 집을 건드리면 떼 지어 쫓아와서 쏘는 것을 무엇이라고 하시나요?},
  advq={벌을 집 안이나 바깥에서 만나면 어떻게 하시나요? \\ 되도록 ‘쏘이다’를 수집한 이후) 벌에 쏘였을 때 어떻게 하나요?}
}

\Entry{
  word={쥐},
  pred={지, 쥐(이중모음)},
  feat={ms,sp,dp},
  desc={V1 /ㅟ/},
  qstn={밤에 찍찍거리고 돌아다니면서 곡식을 훔쳐 먹는 동물을 무엇이라고 하시나요?},
  advq={쥐가 집안에 들어온 적이 있나요? \\ 쥐가 들어오면 어떻게 잡으세요?}
}

\Entry{
  word={돼지},
  pred={도야지, 되야지, 돼:지},
  feat={ms,sp,dp},
  desc={/돗/, /도야지/ 등 이형태},
  qstn={뚱뚱하고, 꿀꿀거리면서 울고, 고기로 많이 해먹는 동물을 무엇이라고 하시나요?},
  advq={돼지고기는 어떤 부위를 가장 좋아하세요? \\ 돼지들은 주로 뭘 먹이면서 키우나요?}
}

\Entry{
  word={고양이},
  pred={고이, 괭이, 고얭이, 괭:이, 고양이},
  feat={ms,sp,dp},
  desc={/괴/, /괴양이/ 등 이형태},
  qstn={(C49)를 잘 잡고, 야옹 하고 우는 동물을 뭐라고 하시나요?},
  advq={고양이를 키우시나요? \\ 아시는 분들 중에 고양이를 키우는 분이 계신가요?}
}

\Entry{
  word={개},
  pred={가이},
  feat={ms,sp,dp},
  desc={V1 /ㅐ-ㅔ/ 대립 \\ /가히/, /갛/ > /개/},
  qstn={집에서 기르는 동물인데, ‘멍멍’ 하고 짓는 것을 뭐라고 하시나요?},
  advq={}
}

\Entry{
  word={꿩},
  pred={꿩},
  feat={ms,sp,dp},
  desc={V1 /ㅝ/},
  qstn={사냥꾼이 많이 잡는 새인데, 숨을 때 머리만 처박는 것을 무엇이라고 하시나요? (‘이것 대신 닭’이라는 말이 있죠)},
  advq={수놈과 암놈을 따로 부르는 말이 있나요? 새끼를 따로 부르는 말도 있나요?}
}

\Entry{
  word={매},
  pred={매},
  feat={ms,sp,dp},
  desc={V1 /ㅐ/ 장단음 \\ V1 /ㅐ-ㅔ/ 대립},
  qstn={독수리랑 비슷한 새인데, 길들여서 사냥에 쓰기도 하는 새를 무엇이라고 하시나요? (‘이것의 눈’이라는 말이 있죠)},
  advq={}
}

\Entry{
  word={말(동물)},
  pred={말},
  feat={ms,sp,dp},
  desc={/말/(언어)과의 성조 음장 차이},
  qstn={아주 빠르게 달리고, 마차를 끄는 동물을 무엇이라고 하시나요?},
  advq={}
}

\Entry{
  word={마구간},
  pred={마구깐},
  feat={ms,sp,dp},
  desc={},
  qstn={(C55)이 사는 집은 뭐라고 하시나요?},
  advq={}
}

\Entry{
  word={외양간},
  pred={오양깐, 오양간},
  feat={ms,sp,dp},
  desc={/오희양/, /외향/ > /외양간/},
  qstn={소가 사는 집은 무엇이라고 하시나요? \\ ‘소 잃고 이거 고친다’는 말이 있죠},
  advq={}
}