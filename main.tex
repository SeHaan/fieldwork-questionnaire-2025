% !TEX root = main.tex
\documentclass{snu-fl-questionnaire}

% Use `\includeonly` to compile partially, e.g.,
% \includeonly{ch1-a}

% Hangul font setup
\setmainhangulfont{KoPubWorldBatang_Pro}[
  Script=Hangul,
  Renderer=OpenType,
  Scale = MatchUppercase,
  UprightFont={* Light},
  BoldFont={* Bold},
  AutoFakeSlant = 0.15
]
\setsanshangulfont{KoPubWorldDotum_Pro}[
  Scale = MatchUppercase,
  BoldFont={* Bold},
]

% If there are problems in Hangul Font setup code,
% run the following line to update your cache:
%   $ fc-cache -fv
%
% Note that on Windows, by default, the user-specific font
% directory (i.e. %LOCALAPPDATA%\Microsoft\Windows\Fonts\)
% is not picked up by fc-cache. Either add this directory
% manually or install the font for all users (C:\Windows\Fonts)
% as administrator using the right-click menu.
%
% Or, you can load fonts by their filenames, e.g.,
%
%\setmainhangulfont{KoPubWorldBatang_Pro}[
%  Scale = MatchUppercase,
%  UprightFont = {KoPubWorld Batang_Pro Light.otf}, % Exact filename
%  BoldFont = {KoPubWorld Batang_Pro Bold.otf},     % Exact filename
%  AutoFakeSlant = 0.15
%]
%\setsanshangulfont{KoPubWorldDotum_Pro}[
%  Scale = MatchUppercase,
%  UprightFont = {KoPubWorld Dotum_Pro Medium.otf}, % Exact filename (or Light)
%  BoldFont = {KoPubWorld Dotum_Pro Bold.otf}       % Exact filename
%]

% Metadata
\title{2025 한국 언어조사 질문지}
\author{서울대학교 인문대학 언어학과}
\date{2025년 10월 31일 \textasciitilde{} 11월 1일}
\printdate{2025년 10월 31일}
\issuedate{2025년 10월 31일}
\tel{(02) 880-6163, 6164}


\begin{document}

\frontmatter
\maketitle
\tableofcontents

\chapter{2025학년도 언어조사 개요}
일정: 2025년 10월 31일(금요일) {\textasciitilde} 11월 1일(토요일) \\
출발: 2025년 10월 31일(금요일) 08시 58분, 광명역 \\
숙소: 더소풍펜션(전남 순천시 별량면 동화사길 113) \\
\\
숙소까지의 교통편: KTX 및 렌터카(카니발 (D) 9인승) \\
\\
조사 장소: 전라남도 보성군 벌교읍 회정리 일대

\chapter{2025학년도 언어조사 일정}
\begin{OnehalfSpace}
\begin{tblr}{
  width = \linewidth,
  hlines,
  vlines,
  cells = {c, font=\sffamily},
  colspec = {c c X[c,m] c},
  row{1} = {lightgray},
  cell{2}{1} = {r=8}{},
  cell{10}{1} = {r=9}{},
  cell{11}{2} = {r=2}{},
  cell{14}{2} = {r=2}{},
  cell{8}{4} = {r=3}{},
}
\textbf{일자}
  & \textbf{일정}
  & \textbf{내용}
  & \textbf{장소} \\

{10. 31. \\ (금)}
  & 08:58{\textasciitilde}11:23
  & 이동(광명역 → 순천역)
  & KTX \\

  & 11:30{\textasciitilde}12:30
  & 점심 식사
  & 별량시골밥상 \\

  & 12:30{\textasciitilde}13:30
  & 조사 장소 이동, 조사 준비
  & 렌터카 \\

  & 13:30{\textasciitilde}17:00
  & \textbf{언어조사}
  & \textbf{보성군노인복지관} \\

  & 17:00{\textasciitilde}18:00
  & 저녁 식사
  & 고려회관 \\

  & 18:00{\textasciitilde}18:30
  & 장소 이동
  & 렌터카 \\

  & 18:30{\textasciitilde}21:00
  & 조사 내용 회의
  & 더소풍펜션 \\

  & 21:00{\textasciitilde}
  & 휴식
  & \\
  
{11. 1. \\ (토)}
  &{\textasciitilde}08:00
  & 기상 및 아침 식사
  & \\

  & 08:00{\textasciitilde}08:30
  & 자료제공인 모셔오기
  & 렌터카 \\

  & 
  & 조사 준비
  & 더소풍펜션 \\
  
  & 08:30{\textasciitilde}11:00
  & \textbf{언어조사}
  & \textbf{더소풍펜션} \\

  & 11:30{\textasciitilde}13:00
  & 자료제공인 모셔다 드리기
  & 렌터카 \\

  &
  & 점심 식사
  & 회정국밥 \\

  & 13:00{\textasciitilde}15:00
  & 지역 문화 탐방
  & 순천만국가정원 \\

  & 15:18{\textasciitilde}17:43
  & 이동 (순천역 → 광명역)
  & KTX \\

  & 17:43{\textasciitilde}
  & 짐 정리
  & 서울대학교 관악캠퍼스
\end{tblr}
\end{OnehalfSpace}

\chapter{자료제공인 조사표 (1)}
\Consultant

\chapter{자료제공인 조사표 (2)}
\Consultant

\chapter{자료제공인 조사표 (3)}
\Consultant

\chapter{자료제공인 조사표 (4)}
\Consultant

\chapter{일러두기}

이 질문지는 2024년도 서산 지역어 조사 질문지를 바탕으로 제작되었다.
중요한 방언형이 나타나는 어휘들을 추가하고 불필요한 항목을 삭제하였으며, 그밖의 구조적인 수정을 거쳐 제작되었다. 

\subsection{어휘·음운 조사 요령}
\begin{enumerate}[noitemsep]
  \item <어휘·음운편>의 조사는 반드시 Ⅲ장의 <어휘 의미지도>를 중심으로 진행한다. 주 질문자는 각 대분류의 [자유발화 질문]으로 해당 의미지도 내에 있는 어휘를 검출한 뒤, 자연스럽게 대화하듯이 연관된 질문을 이어가면서 주변 어휘를 수집한다. 자연스러운 대화가 막히는 순간이 오면, 유도질문과 심화질문을 활용하여 다른 어휘로 대화를 이어나갈 수 있다.
  \item <어휘·음운편>에는, 어휘의미론이나 음운 체계(공시음운론), 역사적 음운 변화(통시음운론)의 양상을 파악하기 위해 선정된 어휘가 있다.
  \begin{itemize}[noitemsep]
    \item 어휘의미론 파악을 위한 어휘: 의도한 의미 맥락에서 검출된 이후에는 표준어형이나 예상되었던 방언형에서 연관된 어휘를 제시하여 해당 단어는 쓰지 않는지, 쓴다면 그 의미 분포가 어떻게 다른지 등을 직접적으로 질문할 수 있다. 가령 /부엌/을 의도한 질문에서 /정지/ 등 다른 계통의 어휘가 검출될 경우, /부엌/이라는 단어는 쓰지 않는지, /정지/와 /부엌/의 차이는 무엇인지 등을 질문할 수 있다. 일부 어휘는 정확한 대상 지칭을 위해 Ⅳ장의 <사진 자료>를 활용할 수도 있다.
    \item 공시음운론 파악을 위한 어휘: 최소대립쌍을 이루는 단어가 함께 질문지에 있는 경우가 있고, 특정 환경에서의 음운 실현 양상을 확인해야 하는 경우가 있다. 특히, 기저 종성을 확인하는 어휘들의 경우, 반드시 단답형이 아닌 조사를 포함하는 등의 연속 발화 내에서 검출이 이루어져야 한다.
  \end{itemize}
\end{enumerate}

\newpage
\subsection{문법 조사 요령}
\begin{enumerate}[noitemsep]
  \item <문법편>은 취조받는다거나 시험 문제를 푼다는 느낌이 들지 않도록 자료제공인을 부드럽게 대하는 것이 중요하다.
  \item 조사의 효율성을 위해 앞선 대화 및 어휘·음운 조사 과정에서 확인한 문법 사항은 제외하고 질문한다.
  \item 조사자 중 1인이 문법 조사를 담당하여, 어휘·음운 조사가 이루어지거나 대화가 오가는 동안 자료제공인이 발화하는 문법 사항을 확인한 후 표시한다. 다만, 조사자의 질문 속에 포함된 문법 표지는 자료제공인이 사용하더라도 표시하지 않는다.
  \item 질문의 편의성을 위해, 어휘·음운과 밀접한 질문들은 <어휘·음운편>에도 수록해 놓았다.
  \begin{itemize}[noitemsep]
    \item 문법 유도질문: 대분류 B, C, E의 첫 쪽에 수록되어 있다. [자유발화 질문]과 연계하여, 자유발화 속에서 문법 표지가 발화될 수 있도록 유도한다.
    \item 문법 연계 심화질문: 문법 요소(특히 사동·피동)를 유도할 수 있는 질문 가운데 특정 어휘와 특히 연관된 것은 해당 어휘의 심화질문에 수록해 두었다.
  \end{itemize}
\end{enumerate} 

\subsection{녹음 요령}

녹음본의 시작과 끝에는 다음과 같은 말이 함께 녹음되도록 한다.

\begin{itemize}[noitemsep]
  \item 조사 시작: yyyy년 mm월 dd일 전라남도 보성군 벌교읍 회정리에서 nn세 OOO 선생님을 모시고 언어조사를 시작합니다. 같은 날의 N번째 녹음본입니다. 조사자는 AAA, BBB, CCC입니다.
  \item 조사 끝: (중간부터 XXX 선생님이 합류하셔서 함께 이야기 나눌 수 있었습니다.) 지금까지 yyyy년 mm월 dd일 전라남도 보성군 벌교읍 회정리에서 nn세 OOO 선생님과 mm세 XXX 선생님을 모시고 진행된 언어조사였습니다.
\end{itemize}


\mainmatter
\chapter{어휘·음운편}

\subsection{일러두기}
\begin{enumerate}[noitemsep]
  \item `자유발화 질문'은 자료제공인이 자유롭게 답할 수 있도록 유도하는 질문이다. 이때 등장한 어휘에서 시작하여, 의미지도를 따라 다양한 어휘를 검출하기 위한 질문을 이어나갈 수 있다. 문답 과정에서 자료제공인의 말을 끊는 등 조급한 태도를 취하지 않도록 주의한다.
  \item 질문 내용을 이해하기 어려울 수 있으므로 비언어적 수단을 적극적으로 활용한다. 동작·도구를 나타내는 어휘라면 몸짓을 활용하고, 사물·상황에 관한 어휘라면 Ⅳ장의 <사진 자료>를 활용한다. 사진을 제시할 때는, 한 번에 하나만 보이도록 하여 혼동이 없도록 한다.
\end{enumerate} 

\subsection{어휘표 예시 및 설명}
\Entry{
  word = {벼},
  pred = {나락},
  feat = {ms},
  desc = {%
    벼이삭(나락모가지), 벼의 열매(쌀) / 쭉정이, 식물 벼의 명칭 등 의미 차이 확인
  },
  qstn = {논에서 자라는, 익을수록 고개를 숙이는 작물을 뭐라고 하시나요?},
  advq = {%
    벼가 다 자라면 어떻게 수확하고 처리하세요? \\
    다 자란 벼를 베고 남는 밑부분을 뭐라고 하시나요? (예상형태: 나락뜰컹/나락끌텅)
  }
}

\begin{itemize}[noitemsep]
  \item 01: 어휘의 일련번호
  \item 예상형태: 사전조사 결과 예상되는 방언형
  \item 유의점: 어휘 조사 시의 유의점이 존재한다면 검은색 볼드 처리
  \begin{itemize}[noitemsep]
    \item \sffamily{\textbf{[MS]}}: \alt{형태의미론}{Morpho-Semantics}
    \item \sffamily{\textbf{[SP]}}: \alt{공시음운론}{Synchronic Phonology}
    \item \sffamily{\textbf{[DP]}}: \alt{통시음운론}{Diachronic Phonology}
  \end{itemize}
  \item 설명: 해당 유의점에 대한 설명
  \item 유도질문: 어휘를 유도할 것으로 기대되는 질문
  \item 심화질문: 해당 어휘의 반복 또는 연관 어휘의 인출, 문법 요소의 사용을 유도할 것으로 기대되는 질문
  \item 방언형: 관찰된 방언형 전사
\end{itemize}


% A. 농사와 식물
%!TEX root = main.tex
\section{농사와 식물}

\subsection{자유발화 질문}
\begin{itemize}[noitemsep]
  \item 농사를 지으시나요?
  \item 어떤 농사를 하시나요?
  \item 논/밭에서는 주로 무엇을 기르시나요?
  \item 어떤 작물을 길러보셨나요?
  \item (작물 이름)을 키우려면 어떤 과정이 필요한가요?
  \item 좋아하시는 채소가 있으세요?
  \item 어떤 과일을 자주 드시나요?
\end{itemize}

\subsection{목표 어휘}
% \Entry 부분을 수정하시면 됩니다.
% word: 표준어형
% pred: 예상형태
% feat: 유의점
%       feat에 들어가는 인자는 쉼표(,)로 구분되며,
%       ms, sp, dp 세 종류가 있습니다.
%       feat에 들어간 인자가 볼드체로 표시됩니다.
% desc: 설명
% qstn: 유도질문
% advq: 심화질문
%
% 그리고 줄바꿈이 필요할 땐 엔터가 아니라 '\\'를 사용해주세요(중요!!).

\Entry{
  word = {벼},
  pred = {나락},
  feat = {ms},
  imag = {img/refs/벼},
  desc = {%
    벼이삭(나락모가지), 벼의 열매(쌀), \\
    쭉정이, 식물 벼의 명칭 등 의미 차이 확인
  },
  qstn = {논에서 자라는, 익을수록 고개를 숙이는 작물을 뭐라고 하시나요?},
  advq = {%
    벼가 다 자라면 어떻게 수확하고 처리하세요? \\
    다 자란 벼를 베고 남는 밑부분을 뭐라고 하시나요? (나락뜰컹/나락끌텅)
  }
}

\Entry{
  word={쌀},
  pred={쌀},
  feat={},
  imag={img/refs/쌀},
  desc={},
  qstn={벼(A01)의 껍질을 벗긴 것을 뭐라고 하시나요?},
  advq={%
    쌀을 어떻게 보관하나요? (쌀똑/차똑) \\
    덜 익은 벼이삭(A01')의 껍질을 벗긴 쌀을 뭐라고 하시나요? (묵지쌀/북떼기쌀)
  }
}

\newpage
\Entry{
  word={볍씨},
  pred={씬나락},
  feat={},
  imag={img/refs/볍씨},
  desc={},
  qstn={벼(A01)의 씨를 무엇이라고 하시나요?},
  advq={볍씨/종잣벼라는 말도 쓰시나요?}
}

\Entry{
  word={못자리},
  pred={모짜리},
  feat={},
  imag={img/refs/못자리},
  desc={},
  qstn={모내기하기 전에 볍씨(A03)를 어디에 심나요?},
  advq={%
    못자리와 논은 보통 근처에 있나요? \\
    못자리에서 어떻게 논으로 모를 옮겨 심나요?
  }
}

\Entry{
  word={보리},
  pred={보리},
  feat={},
  imag={img/refs/보리},
  desc={},
  qstn={%
    벼를 수확하고 난 뒤에 겨울에 길러서 봄여름에 수확하는 곡식이 무엇인가요? \\
    옛날에 쌀이 다 떨어지면 대신 먹는 곡식이 무엇이었나요?
  },
  advq={%
    보릿고개라는 말이 정확히 무슨 뜻인지 설명해주실 수 있나요? \\
    보리쌀이라는 말도 쓰시나요?
  }
}

\Entry{
  word={깜부기},
  pred={깜ː북},
  feat={},
  imag={img/refs/깜부기},
  desc={},
  qstn={보리(A05) 이삭(모가지)에 병이 들어 까맣게 되는 것을 뭐라고 하시나요?},
  advq={깜부기병 말고 곡식에 드는 병이 있나요?}
}

\newpage
\Entry{
  word={조},
  pred={서ː숙, 서ː숙쌀},
  feat={},
  imag={img/refs/조},
  desc={},
  qstn={%
    (A08 수수, A09 기장과 함께) 잡곡밥에는 무엇을 넣나요? \\
    벼/보리 말고 농사 짓는 곡식에는 무엇이 있나요? \\
    (조) 노랗고 모래알처럼 알갱이가 조그마한 곡식을 무엇이라고 하나요?
  },
  advq={좁쌀이라는 말도 쓰시나요?}
}

\Entry{
  word={수수},
  pred={쑤시, 쒸시},
  feat={},
  imag={img/refs/수수},
  desc={},
  qstn={알갱이가 붉은색을 띠고 동글동글한데 경단도 만들어 먹는 곡식을 뭐라고 하시나요?},
  advq={%
    (A09 기장과 함께) 수수와 기장을 잡곡밥에 넣기도 하나요? \\
    보름날 먹는 오곡밥에는 무엇을 넣나요?
  }
}

\Entry{
  word={기장},
  pred={-},
  feat={},
  imag={img/refs/기장},
  desc={},
  qstn={조처럼 노란데 조보다는 약간 굵고 알갱이가 더 큰 곡식을 뭐라고 하시나요?},
  advq={}
}

\newpage
\Entry{
  word={옥수수},
  pred={옥소시, 강넹이},
  feat={ms},
  imag={img/refs/옥수수},
  desc={옥수수/강냉이의 의미 차이, \\ 열매와 열매 자루의 의미 차이 확인},
  qstn={통째로 쪄서 먹는데, 찌면 알갱이가 노랗고, 사료로도 쓰는 곡식을 뭐라고 하시나요?},
  advq={%
    옥소시/강넹이라는 말도 쓰나요? 뭐가 다른가요? \\
    벌교에서 옥수수가 나나요? 아니면 다른 지역에서 옥수수가 들어오나요?
  }
}

\Entry{
  word={새끼(줄)},
  pred={사네끼, 사쳉이},
  feat={sp},
  imag={img/refs/새끼(줄)},
  desc={2음절 /ㅐ-ㅔ/대립},
  qstn={지푸라기를 비벼 만든 끈을 뭐라고 하시나요?},
  advq={지푸라기로 새끼를 만드는 것을 뭐라고 하시나요? (새끼를 까다)}
}

\Entry{
  word={새참},
  pred={셋ː꺼리, 셋ː빱, 쎄클},
  feat={},
  imag={img/refs/새참},
  desc={},
  qstn={김도 매고 타작도 하고 일하다가, 시장할 때 드시는 것을 뭐라고 하시나요?},
  advq={새참으로는 주로 어떤 것들을 드시나요?}
}

\Entry{
  word={절구},
  pred={도ː구통},
  feat={},
  imag={img/refs/절구},
  desc={},
  qstn={곡식이나 떡을 찧을 때에는 어디에다 찧으시나요?},
  advq={%
    절구는 주로 돌로 만드나요? \\
    절구를 찧을 때 쓰는 막대기를 뭐라고 하시나요? (도굿대 등)
  }
}

\newpage
\Entry{
  word={겨},
  pred={제},
  feat={},
  imag={img/refs/겨},
  desc={},
  qstn={절구(A13)에 곡식을 찧고 나면 남는 껍질을 뭐라고 하시나요?},
  advq={%
    겨와 쭉정이의 차이점이 무엇인가요? \\
    겉껍질('왕겨' 껏불제 등)과 속껍질('등겨' 누무께, 등게 등)을 각각 뭐라고 하시나요?
  }
}

\Entry{
  word={키},
  pred={쳉이},
  feat={},
  imag={img/refs/키},
  desc={},
  qstn={%
    곡식에서 쭉정이나 티끌을 털어서 골라낼 때 쓰는 것을 뭐라고 하시나요? \\
    예전에 오줌싸개가 쓰고 다니던 것이 무엇인가요?
  },
  advq={}
}

\Entry{
  word={김매기},
  pred={지심메다},
  feat={sp,dp},
  imag={img/refs/김매기},
  desc={\jamoword{gi/zim/m@i/da}>김매다},
  qstn={논밭에 잡초가 많이 나면 어떻게 하시나요? 논에서 잡초를 없애는 것을 무엇이라고 하나요?},
  advq={%
    논에서 하는 것과 밭에서 하는 것 모두 김매기라고 부르나요? \\
    마지막 김매기를 부르는 말이 있나요? (맘ː물, 만드리) \\
    마지막 김매기를 끝내고 하루를 즐기는 풍습이 있나요? 뭐라고 부르나요? (두레지심, 모심다레기, 술멕이, 시름짱, 풍년다레기, 써레시침)
  }
}

\newpage
\Entry{
  word={호미},
  pred={호무, 호멩이},
  feat={},
  imag={img/refs/호미},
  desc={},
  qstn={%
    밭에서 김매기(A16)를 하실 때는 무엇을 가지고 하시나요? \\
    감자나 고구마 등을 캘 때 사용할 때 어떤 도구를 쓰시나요?
  },
  advq={%
    가장 첫 호미질을 부르는 말이 있나요? (호무꾸리) \\
    호미가 농사에서 이것저것 쓰임새가 많다는데, 호미로 할 수 있는 작업에는 무엇이 있나요?
  }
}

\Entry{
  word={써레},
  pred={써ː레},
  feat={dp},
  imag={img/refs/써레},
  desc={써흐레>써레},
  qstn={긴 나무에 발을 박아 놓고 끌면서, 갈아 놓은 논의 바닥을 고르는 농기구를 뭐라고 하시나요? },
  advq={(A18-A21을 함께 진행)}
}

\Entry{
  word={쇠스랑},
  pred={소시랑},
  feat={},
  imag={img/refs/쇠스랑},
  desc={},
  qstn={땅을 파헤쳐 고르거나 두엄을 칠 때 사용하는 갈퀴 모양의 농기구를 뭐라고 하시나요?},
  advq={(A18-A21을 함께 진행)} % TODO: '두엄'도 확인할까?
}

\Entry{
  word={가래(농기구)},
  pred={가레},
  feat={sp},
  imag={img/refs/가래(농기구)},
  desc={2음절 /ㅐ-ㅔ/ 대립},
  qstn={%
    한 사람이 자루를 잡고 두 사람이 줄을 잡아당겨 흙을 퍼서 던지는 농기구를 뭐라고 하시나요?
  },
  advq={%
    가래는 삽과는 다른 것인가요? 어떻게 다른가요? \\
    (A18-A21을 함께 진행)
  }
}

\newpage
\Entry{
  word={도리깨},
  pred={돌께, 도리께},
  feat={},
  imag={img/refs/도리깨},
  desc={},
  qstn={긴 막대기에 싸리나 대나무를 달아 곡식을 두드려 타작하는 농기구를 뭐라고 하시나요?},
  advq={써레/쇠스랑/가래/도리깨(A18-A21)를 요즘 농사지을 때에도 사용하시나요? 옛날 농사에만 사용했나요?}
}

\Entry{
  word={멍석},
  pred={덕썩},
  feat={},
  imag={img/refs/멍석},
  desc={},
  qstn={곡식을 타작하거나 널어 말릴 때 어디에 깔아 놓으시나요?},
  advq={%
    요즘에도 멍석을 깔고 무언가를 말리기도 하나요? 무엇을 말리나요? \\
    곡식을 말리는 멍석을 부르는 말이 따로 있나요? (우게덕썩[화순])
  }
}

\Entry{
  word={깨},
  pred={꾀},
  feat={},
  imag={img/refs/깨},
  desc={},
  qstn={참기름이나 들기름은 무엇을 짜서 만드시나요?},
  advq={참깨와 들깨는 어떻게 구분하나요?}
}

\newpage
\Entry{
  word={나물},
  pred={노무세, 노물},
  feat={},
  imag={img/refs/나물},
  desc={},
  qstn={%
    산이나 들에 나는 풀 중에서 캐다가 무쳐 먹는 것을 뭐라고 하시나요?
    대보름에 오곡밥과 함께 먹는 것을 뭐라고 하시나요?
  },
  advq={%
    나물 종류로는 어떤 것이 있나요? (A25-A27) \\
    명절에 나물반찬 많이 하시나요? 어떤 나물들을 하시나요? (B25 부침개나 A38 배, A39 밤 등 명절 음식 관련)
  }
}

\Entry{
  word={냉이},
  pred={나상구, 나상게, 나셍기, 나싱게, 나셍이},
  feat={dp},
  imag={img/refs/냉이},
  desc={\jamoword{na/zi/}, 낭히>냉이},
  qstn={봄에 들이나 밭에서 뿌리(A40)째로 캐서 된장국을 해 먹는 것을 뭐라고 하시나요?},
  advq={냉이랑 비슷하게 생긴 식물은 무엇이 있나요? 냉이랑 구분할 수 있는 방법이 있나요?}
}

\Entry{
  word={달래},
  pred={달룽게},
  feat={},
  imag={img/refs/달래},
  desc={},
  qstn={%
    파와 같은 냄새가 나고 매운 맛이 나는 나물인데, 양파보다는 훨씬 작은 것은 무엇인가요? \\
    봄에 들에서 나는, 뿌리(A40)는 조그맣고 동글동글하면서 줄기는 실처럼 가늘고 긴 것은 무엇인가요?
  },
  advq={}
}

\newpage
\Entry{
  word={콩나물},
  pred={콩노믈},
  feat={ms},
  imag={img/refs/콩나물},
  desc={무친 콩나물/무치지 않은 콩나물의 명칭 차이},
  qstn={콩에 물을 주어 길러 먹는 것은 뭐라고 하시나요?},
  advq={콩나물로는 무슨 반찬을 해 드시나요? 콩지름/콩질금이라는 말도 쓰시나요?}
}

\Entry{
  word={상추},
  pred={상추},
  feat={},
  imag={img/refs/상추},
  desc={},
  qstn={쌈을 싸 드실 때 어디에 많이 싸서 드시나요?},
  advq={상추는 어느 계절에 기르나요? (C06 겨울 및 관련 어휘)}
}

\Entry{
  word={오이},
  pred={믈외, 외ː},
  feat={},
  imag={img/refs/오이},
  desc={},
  qstn={%
    덩굴에서 나는데 녹색으로 길쭉하게 생기고 겉이 우툴두툴하니, \\
    시원한 맛이 있고 물이 많은 것은 무엇인가요?
  },
  advq={오이로 무슨 반찬을 해 드시나요?}
}

\Entry{
  word={부추},
  pred={솔ː},
  feat={},
  imag={img/refs/부추},
  desc={},
  qstn={파처럼 생겼는데 좀 더 가늘고, 겉절이나 무침으로도 먹는 것을 뭐라고 하시나요?},
  advq={%
    소풀/소불이라는 말은 안 쓰시나요? \\
    부추를 주로 어떤 음식으로 해 먹나요? 무슨 음식에 넣어서 먹나요?
  }
}

\newpage
\Entry{
  word={진달래},
  pred={창꼿},
  feat={ms,sp},
  imag={img/refs/진달래},
  desc={참꽃/진달래의 의미 차이, 나무와 꽃의 명칭 차이},
  qstn={%
    봄에 산에서 가장 먼저 피는 꽃은 무엇인가요? \\
    색깔이 벌겋고 화전을 해 먹거나 술을 담가 먹기도 하는 꽃은 무엇인가요?
  },
  advq={진달레라는 말도 쓰시나요? 창꼿(참꽃)은 무슨 뜻인가요?}
}

\Entry{
  word={철쭉},
  pred={게ː꼿},
  feat={ms},
  imag={img/refs/철쭉},
  desc={개꽃의 의미, 나무와 꽃의 명칭 차이},
  qstn={진달래(A29)와 비슷하게 생겼는데 좀 더 늦게 피고 먹지 못하는 것은 뭐라고 하시나요?},
  advq={%
    철쭉이라는 말도 쓰시나요? 게ː꼿(개꽃)은 무슨 뜻인가요? \\
    진달래와 철쭉의 차이점은 무엇인가요? 둘을 어떻게 구분할 수 있나요?
  }
}

\Entry{
  word={삘기},
  pred={삐ː비, 뛰},
  feat={ms,dp},
  imag={img/refs/삘기},
  desc={삐유기>삘기 /k>p/, 삘기가 다 자라면 띠},
  qstn={들에 나는 풀 종류이고, 실뭉치 같은 속줄기를 뽑아서 먹는 풀을 무엇이라고 하시나요?},
  advq={%
    삐ː비/뛰라는 말도 쓰시나요? 뭐가 다른가요? \\
    삘기를 드셔 보셨나요? 무슨 맛인지 설명해주실 수 있나요? (B33 맛 형용사)
  }
}

\Entry{
  word={머루},
  pred={멀구},
  feat={ms,dp},
  imag={img/refs/머루},
  desc={머뤼>머루},
  qstn={산에서 나는, 포도랑 비슷한데 좀 작고 담금주도 담가 먹는 열매는 무엇인가요?},
  advq={머루와 포도의 차이점은 무엇인가요? 구별하는 법이 있나요?}
}

\newpage
\Entry{
  word={다래},
  pred={다ː레},
  feat={},
  imag={img/refs/다래},
  desc={},
  qstn={산에 많고 크기는 대추만 한데 초록색이고 가을에 누런 풀색으로 익으면 따먹는 열매는 무엇인가요?},
  advq={머루(A34)와 다래는 산에서 난다고 들었는데, 이걸 키우거나 농사짓는 분은 없나요?}
}

\Entry{
  word={고욤},
  pred={괴양감},
  feat={ms},
  imag={img/refs/고욤},
  desc={돌감과의 의미 차이},
  qstn={빛깔은 감이랑 비슷한데 훨씬 작고, 처음에는 떫지만 익으면 달아지는 열매는 무엇인가요?},
  advq={고욤은 돌감(산감)과는 다른 열매인가요? 고욤은 떫다고 들었는데 먹는 방법이 따로 있나요?}
}

\Entry{
  word={복숭아},
  pred={복성},
  feat={},
  imag={img/refs/복숭아},
  desc={},
  qstn={크기는 사과만 한데 색깔은 하얗거나 분홍색이고 물이 많은 과일은 무엇인가요?},
  advq={(A37-A39을 함께 진행) 복숭아를 제사상에 올리기도 하나요?}
}

\Entry{
  word={배(과일)},
  pred={베},
  feat={sp},
  imag={img/refs/배(과일)},
  desc={배(신체), 배(연산)와의 성조, 장단 차이},
  qstn={사과보다 좀 크고 껍질이 노란데, 살이 연하고 달며 물이 많은 과일은 무엇인가요?},
  advq={(A37-A39와 함께) 제사상에 어떤 과일을 올리시나요? 올리는 순서는 어떻게 되나요?}
}

\newpage
\Entry{
  word={밤(열매)},
  pred={밤:},
  feat={sp},
  imag={img/refs/밤(열매)},
  desc={밤(夜)과의 성조, 장단 차이},
  qstn={가시가 난 송이에 싸여 있고 얇고 맛이 떫은 속껍질이 있으며 날것으로 먹거나 굽거나 삶아서 먹을 수 있는 열매를 뭐라고 하시나요?},
  advq={(A37-A39을 함께 진행) 밤을 제사상에 올리기도 하나요?}
}

\Entry{
  word={뿌리},
  pred={꿀텅, 뿌리, 뿌렝이},
  feat={ms,dp},
  imag={img/refs/뿌리},
  desc={꿀텅/뿌리 의미 차이, 불휘/불희>뿌리},
  qstn={보통 땅속에 묻혀서 물과 양분을 빨아들이는 식물의 아래 부분을 무엇이라고 하시나요?},
  advq={꿀텅/뿌리라는 말도 쓰시나요? 어떤 차이가 있나요? 달래(A26)는 뿌리까지도 먹나요?}
}

\Entry{
  word={무},
  pred={무시, 무수},
  feat={sp,dp},
  imag={img/refs/무},
  desc={장모음 여부, \jamoword{mu/zu}/무우>무},
  qstn={하얗고 굵은데, 뿌리(A40)를 먹으려고 키우고, 김장해 먹기도 하는 농작물은 무엇인가요?},
  advq={%
    열무라는 말도 쓰나요? 열무가 혹시 무의 줄기인가요? 무의 줄기도 먹나요? \\
    무를 김장하면 뭐라고 부르나요? (석박지, 똑딱지, 쪼각지 등)
  }
}

% B. 가정과 생활
%!TEX root = main.tex
\section{가정과 생활}
\subsection{일러두기}
일러둘 내용을 적습니다.

\subsection{자유발화 질문}
\begin{itemize}[noitemsep]
  \item 어릴 적에 살던 댁은 어떻게 생겼어요?
  \item 그 집에서 기억에 남는 재밌는 추억이 있으신가요?
  \item 요리하는 것을 좋아하시나요?
  \item 혹시 자신 있는 요리가 있으신가요?
  \item 좋아하시는 음식이 무엇인가요?
\end{itemize}

\subsection{문법 유도질문}
※ [-보다], [-만큼] 등 비교에 활용되는 보조사를 잘 확인하도록 합니다.

\begin{enumerate}[noitemsep]
  \item 여름과 겨울에 각각 어떤 음식 주로 드시나요?
  \item 여름에 먹는 음식과 겨울에 먹는 음식 중 어느 쪽이 더 좋으세요?
  \item 그 이유가 무엇인가요?
\end{enumerate}

\subsection{목표 어휘}
% \Entry 부분을 수정하시면 됩니다.
% word: 표준어형
% pred: 예상형태
% feat: 유의점
%       feat에 들어가는 인자는 쉼표(,)로 구분되며,
%       ms, sp, dp 세 종류가 있습니다.
%       feat에 들어간 인자가 볼드체로 표시됩니다.
% desc: 설명
% qstn: 유도질문
% advq: 심화질문
%
% 그리고 줄바꿈이 필요할 땐 엔터가 아니라 '\\'를 사용해주세요(중요!!).

\textbf{[B01, 03]} 두 어휘의 의미가 겹치는 방언이 존재한다. 이를 확인해야 한다. \\

\Entry{
  word={부엌},
  pred={정지, 정제},
  feat={ms,sp},
  imag={img/refs/부엌},
  desc={아궁이와의 의미 범위, 다른 단어(정지 등), 기저 종성 /ㅋ/},
  qstn={식사를 준비하거나 설거지를 할 때 어디로 가시죠? \\ 집안에서 요리기구들이 주로 어디에 있죠?},
  advq={정지라는 말을 들어보셨나요/쓰시나요? 들어보셨다면 부엌과 정지 사이에 의미 차이가 있나요?}
}

\Entry{
  word={솥},
  pred={솥},
  feat={sp},
  imag={img/refs/솥},
  desc={기저 종성 /ㅌ/},
  qstn={주로 쇠붙이로 만들어진, 밥을 짓거나 국을 끓이는 그릇을 뭐라고 부르시나요?},
  advq={요즘에도 가마솥을 쓰나요? 전기밥솥으로 밥을 지을 때와 가마솥으로 밥을 지을 때는 무슨 차이가 있나요?}
}

\Entry{
  word={아궁이},
  pred={부샄},
  feat={ms,dp},
  imag={img/refs/아궁이},
  desc={부엌과의 의미 범위, 다른 단어(부엌 등), 아궁ㄱ, 아궁지… > 아궁},
  qstn={%
    부엌(B01)에서 솥(B02)을 안치는 곳을 무엇이라고 하시나요? \\
    온돌방을 데울 때 여기다가 불을 때기도 하는데, 어디다 때지요?
  },
  advq={}
}

\Entry{
  word={부지깽이},
  pred={부지땡이},
  feat={ms},
  imag={img/refs/부지깽이},
  desc={기타 이형태(부지대, 부짓대)},
  qstn={아궁이(B03)에 불을 땔 때 쓰는 막대기를 무엇이라고 하시나요?},
  advq={(B05 부삽, B06 화로와 함께 진행)}
}

\Entry{
  word={부삽},
  pred={불까레},
  feat={},
  imag={img/refs/부삽},
  desc={},
  qstn={%
    불덩어리를 담아 옮길 때에 쓰는 조그마한 삽을 무엇이라고 하시나요? \\
    난로에 숯을 퍼 담을 때 쓰는 삽을 무엇이라고 부르시나요?
  },
  advq={%
    부지깽이(B04)와 부삽은 아궁이에 불을 지필 때 사용할 수 있나요? \\
    부삽을 가지고 어디로 불을 옮기나요?
  }
}

\Entry{
  word={화로},
  pred={화:리},
  feat={},
  imag={img/refs/화로},
  desc={},
  qstn={주로 불씨를 보존하거나 난방을 위해 쓰는, 숯불을 담아 놓는 그릇을 무엇이라고 하시나요?},
  advq={겨울에 방 안에서 화로를 쓰나요? 화로를 쓸 때 부지깽이(B04)나 부삽(B05)도 사용하나요?}
}

% new lexicon
\Entry{
  word={땔감/땔나무},
  pred={뗄나무},
  feat={ms,dp},
  imag={img/refs/땔감-땔나무},
  desc={땔감과 땔나무가 구분되는지, 나무(A24)와는 형태가 어떻게 다른지},
  qstn={아궁이 등에 불을 지필 때, 도끼로 패서 쓰는 연료를 무엇이라고 부르시나요?},
  advq={%
    땔나무를 해 오는 행위를 따로 부르는 말이 있나요? \\
    땔나무란 보통 어디에서 어떻게 해 오는지 설명해 주실 수 있으신가요?
  }
}

\Entry{
  word={베개},
  pred={비:게},
  feat={sp},
  imag={img/refs/베개},
  desc={V1, V2 /ㅐ-ㅔ/ 대립},
  qstn={주무실 때 머리에 무엇을 받치고 주무시나요? (`베다'라는 동사를 쓸 경우 여기에 이끌려 정확한 방언형이 나오지 않을 수 있음.)},
  advq={어릴 적에는 어떤 베개를 쓰셨나요? 요즘 나오는 베개들과 다른 점이 있나요?}
}

\Entry{
  word={의자},
  pred={이자},
  feat={sp},
  imag={img/refs/의자},
  desc={V1 /ㅢ/},
  qstn={높은 식탁 같은 데에 앉으실 때는 방바닥에 바로 앉지 않고 어디에 앉으시나요?},
  advq={옛날엔 의자보다 방바닥에 주로 앉으셨나요? 집에 의자가 있었나요?}
}

\Entry{
  word={마루},
  pred={}, % TODO: 예상형태 기재
  feat={},
  imag={img/refs/마루},
  desc={},
  qstn={집채 안에 널빤지나 나무로 넓게 깔아 놓고 쉬는 곳을 무엇이라고 하시나요?},
  advq={%
    한옥에서 살아보신 경험이 있으신가요? \\
    한옥에는 어떤 마루가 있나요? (대청마루, 툇마루 등)
  }
}

\Entry{
  word={벽},
  pred={벡},
  feat={ms},
  imag={img/refs/벽},
  desc={벽과 바람벽을 구분하는지},
  qstn={방과 방 사이를 구분하기 위해 저렇게(벽을 가리킴.) 천장에서 바닥까지 나누어 놓은 것을 무엇이라고 하시나요?},
  advq={바람벽이라는 말을 쓰시나요? 바람벽과 벽의 뜻에 차이가 있나요?}
}

\Entry{
  word={이엉},
  pred={},
  feat={},
  imag={img/refs/이엉},
  desc={},
  qstn={초가지붕을 덮기 위해 짚으로 엮은 것은 무엇이라고 하시나요?},
  advq={초가집의 지붕을 이엉이라고 부르나요? 아니면 지붕의 재료를 이엉이라고 부르나요? 짚으로 된 지붕이 썩으면 이엉을 어떻게 고치나요?}
}

\Entry{
  word={주춧돌},
  pred={주춧독},
  feat={sp,dp},
  imag={img/refs/주춧돌},
  desc={C1 /ㅈ/, 중세어 말음 /ㅀ/},
  qstn={기둥 밑에서 기둥을 받치는 돌은 무엇이라고 하시나요?},
  advq={주춧돌은 네모나게 생겼었나요? 다듬지 않은 돌을 주춧돌로 쓰기도 하나요?}
}

\Entry{
  word={서까래},
  pred={서끌~세끌},
  feat={dp},
  imag={img/refs/서까래},
  desc={혓가래… > 서까래},
  qstn={한옥 지붕을 옆으로 비스듬하게 받치고 있는 긴 통나무를 무엇이라고 하시나요?},
  advq={대들보와 서까래의 차이점이 무엇인가요?}
}

\Entry{
  word={변소},
  pred={뒷간, 벤소, 칙간, 퉁시},
  feat={ms},
  imag={img/refs/변소},
  desc={다른 단어 뒷간},
  qstn={대변이나 소변은 어디에서 보시나요?},
  advq={(`화장실' 등으로 대답한 경우) 주로 시골에서 볼 수 있고 집 밖에 위치한 옛날식 화장실을 무엇이라고 하시나요?}
}

\Entry{
  word={우물},
  pred={세:메, 샘:},
  feat={ms,dp},
  imag={img/refs/우물},
  desc={중세어 \jamoword{s@im/}에서 1음절 \jamoword{@i/}의 변화; 바가지 우물, 두레박 우물, 펌프 우물의 명칭 및 의미 차이},
  qstn={땅을 파서 물을 긷는 곳을 무엇이라고 하시나요?},
  advq={바가지로 물을 뜨는 우물과 두레박으로 뜨는 우물을 다르게 부르나요? 펌프로 물을 퍼올리는 우물은 어떤가요?}
}

\Entry{
  word={도랑},
  pred={께랑, 꼬랑, 도꾸랑},
  feat={ms,dp},
  imag={img/refs/도랑},
  desc={다른 단어(걸, 거렁) 등, 돌항… > 도랑},
  qstn={땅을 좁고 깊게 파서 물이 흐르도록 만든 것을 무엇이라고 하시나요?},
  advq={혹시 사람이 판 게 아니라 원래 자연적으로 흐르던 개울도 도랑이라고 부르나요?}
}

\Entry{
  word={민물},
  pred={민물},
  feat={},
  desc={},
  qstn={강이나 호수 따위와 같이 짠맛이 나지 않는 물을 무엇이라고 하나요?},
  advq={%
    벌교에서는 민물에서 나는 물고기도 먹나요? \\
    민물이 아닌 물도 있나요? (→ D08 바닷물)
  }
}

\Entry{
  word={광주리},
  pred={꽝주리},
  feat={},
  imag={img/refs/광주리},
  desc={},
  qstn={과일과 곡식 등을 담을 수 있는 버들 따위를 엮어 만든 그릇은 무엇인가요?},
  advq={한 광주리, 두 광주리… 하는 식으로 광주리를 물건을 세는 단위로 쓰시기도 하나요?}
}

\Entry{
  word={궤짝},
  pred={궤:짝},
  feat={sp},
  imag={img/refs/궤짝},
  desc={V1의 장모음 여부, V1 /ㅞ/},
  qstn={쌀이나 돈과 같은 물건을 넣도록 네모나게 나무로 짠 가구를 무엇이라고 하시나요?},
  advq={(`궤'라고 대답한 경우) 궤짝이라는 단어는 안 쓰시나요?}
}

\Entry{
  word={맷돌},
  pred={맷독},
  feat={dp},
  imag={img/refs/맷돌},
  desc={중세어 말음 /ㅀ/},
  qstn={콩이나 팥 같은 것을 갈 때 어디에 가시나요? (믹서기 말고 옛날에 쓰던 것, 돌로 만든 것)},
  advq={맷돌을 사람 혼자서 사용할 수 있었나요?}
}

\Entry{
  word={체/어레미/도두미},
  pred={얼게미},
  feat={ms},
  imag={img/refs/체-어레미-도두미},
  desc={`어레미'의 이형태(얼멍이/얼멩이)},
  qstn={가루를 곱게 칠 때 어디에 치시나요?},
  advq={구멍이 굵은 것이랑 가는 것을 다른 이름으로 부르기도 하시나요?}
}

\Entry{
  word={뚝배기},
  pred={투가리, 투거리, 툭시발},
  feat={},
  imag={img/refs/뚝배기},
  desc={},
  qstn={그릇 중에서 찌개를 끓이실 때는 무엇을 주로 쓰시나요?},
  advq={국물 요리 중에 뚝배기는 어떤 요리에 쓰시고, 냄비는 어떤 요리에 쓰시나요?}
}

% new lexicon
\Entry{
  word={차곡차곡},
  pred={차복차복},
  feat={dp},
  desc={C2 /ㄱ/, V2 /ㅗ/},
  qstn={%
    식사를 마치고 나면 설거지를 하잖아요. 설거지한 그릇은 보통 어떻게 두시나요? \\
    (그릇이 잘 정리되어 있는 경우) 찬장에 그릇이 예쁘게 정리되어 있네요. 저런 걸 어떻게 쌓여 있다고 하시나요?
  },
  advq={}
}

\Entry{
  word={깍두기},
  pred={쪼각지},
  feat={},
  imag={img/refs/깍두기},
  desc={},
  qstn={무(A41)를 네모나게 썰어 담근 김치를 무엇이라고 하시나요?},
  advq={총각김치와 깍두기의 차이점은 무엇인가요?}
}

\Entry{
  word={두부},
  pred={뚜부},
  feat={},
  imag={img/refs/두부},
  desc={},
  qstn={콩을 갈아서 만든 것으로 하얗고 네모나게 만들어서 된장찌개에 넣어 드시는 것은 무엇인가요?},
  advq={(`메주'로 대답한 경우) `간수를 넣어 만드는 것'이라고 설명한다.}
}

\Entry{
  word={부침개},
  pred={부처리},
  feat={ms},
  imag={img/refs/부침개},
  desc={다른 단어(전, 찌짐, 빈대떡, 저냐, 누름적 등)},
  qstn={밀가루나 메밀가루를 풀어서 프라이팬에 넓적하게 부쳐 먹는 것을 무엇이라고 하시나요?},
  advq={%
    명절에 만드는 부침개에는 무엇이 있나요? (만들다/만들어라(B36) 유도 가능) \\
    명절 외에도 부침개를 자주 드시나요?
  }
}

\Entry{
  word={주걱},
  pred={주벅},
  feat={sp},
  imag={img/refs/주걱},
  desc={C2 /ㄱ/},
  qstn={솥(B02)에서 밥을 풀 때 무엇으로 푸시나요?},
  advq={밥을 풀 때 말고 주걱을 쓰는 경우가 있나요?}
}

\Entry{
  word={누룽지},
  pred={깡밥, 누른밥},
  feat={ms},
  imag={img/refs/누룽지},
  desc={의미 범위(솥에서 긁어낸 누룽지를 말린 것의 명칭)},
  qstn={솥(B02)에서 밥을 다 펐는데도 솥에 눌어붙은 것은 무엇인가요?},
  advq={(숭늉(B30)과 함께 진행)}
}

\Entry{
  word={숭늉},
  pred={숭넹, 숭님},
  feat={ms},
  imag={img/refs/숭늉},
  desc={기타 이형태(/슉랭/)},
  qstn={누룽지(B29)에 물을 넣고 끓인 것은 무엇인가요?},
  advq={그냥 밥으로도 숭늉을 끓일 수 있나요? 누룽지로 끓인 것만을 숭늉이라고 하나요?}
}

\Entry{
  word={수제비},
  pred={수제비},
  feat={ms},
  imag={img/refs/수제비},
  desc={찹쌀로 만든 것과 밀가루로 만든 것의 구별 유무},
  qstn={밀가루 반죽을 넓적하게 뜯어서 넣어 만든 국을 무엇이라고 하시나요?},
  advq={벌교에서는 수제비를 맑은 국물에 먹나요, 빨간 국물에 먹나요? 아니면 둘 다 먹나요?}
}

\Entry{
  word={튀밥},
  pred={튀밥},
  feat={ms,sp},
  imag={img/refs/튀밥},
  desc={재료가 옥수수, 보리쌀, 콩일 경우 명칭에 차이가 있는지, 다른 단어(포데기, 뻥튀기, 광밥, 강냉이 등)},
  qstn={옥수수(A10)를 튀겨서 만든 간식거리를 무엇이라고 하시나요? 그러면 쌀(A02)을 튀겨 만든 간식거리는 무엇이라고 하시나요? (반드시 두 개를 모두 질문하도록 한다. 두 개를 구분하는 방언이 많다. 기본적으로 튀밥은 튀긴 쌀과 튀긴 옥수수 둘 다 포괄하는 단어다.)},
  advq={(안 나온 어형에 대해 벌교에서 쓰는 말인지 물어보기)}
}

\Entry{
  word={식혜/감주},
  pred={단술, 절주, 절쭈},
  feat={ms},
  imag={img/refs/식혜-감주},
  desc={%
    찹쌀(식혜)과 멥쌀(감주)의 차이 \\
    다른 단어(청감, 단밥, 단술 등)
  },
  qstn={밥에다가 엿기름을 우려낸 물을 부어 삭히면 되나요? (시원하게 만들어 밥알을 띄워 먹기도 함.)},
  advq={%
    식혜/감주 중 한 어휘가 검출되면 다른 쪽 어휘를 역질문한다. \\
    (예: 식혜 검출 → 감주는 무엇인가요?)
  }
}

\Entry{
  word={왜간장},
  pred={외장},
  feat={sp},
  desc={V1 /ㅙ/, 간장을 `지랑'이라 하기도 함에 주의},
  qstn={일본식으로 만든 간장을 뭐라고 하시나요? (집에서 만드는 재래식 장 다음에 들어온 장, 일본간장/양조간장)},
  advq={조선간장이랑 왜간장이랑 차이가 무엇인가요? 쓰임새가 어떻게 다르나요?}
}

\Entry{
  word={짜다, 쓰다, 맵다, 떫다},
  pred={짜다/쓰다, 씨다, 씁쓸허다/멥다/떠:럽다},
  feat={},
  desc={},
  qstn={%
    소금/약/고추/덜 익은 감은 맛이 어떻다고 하죠? \\
    오미자에서 다섯 가지 맛이 나서 오미자라고 하잖아요, 이 다섯 가지 맛이 무엇인지 혹시 아시나요?
  },
  advq={}
}

\Entry{
  word={만들다/만들어라},
  pred={맨들다, 맹글다},
  feat={},
  desc={},
  qstn={집은 짓는다고 하지만 음식은 ( \qquad )다고 하지요.},
  advq={(이전 음식 어휘들에서 포착)}
}

% new lexicon
\Entry{
  word={옮기다},
  pred={앵기다},
  feat={dp},
  desc={V1 /ㅗ/},
  qstn={%
    이 집에서 계속 사셨나요? 아니면 이사하신 적이 있나요? \\
    이사하는 것을 두고 집을 ( \qquad )다고 하지요.
  },
  advq={(이전 음식 어휘들에서 포착)}
}

\Entry{
  word={고쟁이},
  pred={고쟁이},
  feat={},
  imag={img/refs/고쟁이},
  desc={},
  qstn={한복을 입을 때 여자들이 안에 입는 속옷인데, 속곳 위에 입는 것을 무엇이라고 하시나요?},
  advq={고쟁이는 겨울에도 입나요? 여름에만 입나요?}
}

\Entry{
  word={모시},
  pred={모시},
  feat={ms,sp},
  imag={img/refs/모시},
  desc={다른 단어 저, 저마, 저포, 장단음, C2 /ㅅ/},
  qstn={삼베하고 비슷한데 좀 더 촘촘하고, 여름에 옷을 해 입으면 시원한 것은 무엇인가요?},
  advq={`모시떡' 할 때 모시가 이 모시와 같나요?}
}

\Entry{
  word={골무},
  pred={골미, 꼴미},
  feat={},
  imag={img/refs/골무},
  desc={},
  qstn={바느질할 때 손가락에 끼는 것을 무엇이라고 하시나요?},
  advq={골무는 어느 손가락에 끼나요?}
}

\Entry{
  word={가위},
  pred={가세, 가세끼, 가세기},
  feat={sp,dp},
  imag={img/refs/가위},
  desc={V2 /ㅟ/, \jamoword{g@z/ay/} > 가위},
  qstn={%
    종이나 옷감은 무엇으로 자르시나요? \\
    국수나 냉면 면을 자를 때 무엇을 쓰시나요?
  },
  advq={옛날 가위도 요즘 가위와 비슷하게 생겼나요?}
}

\Entry{
  word={고름(의복)},
  pred={고름},
  feat={},
  imag={img/refs/고름(의복)},
  desc={},
  qstn={한복을 입을 때 옷깃을 여미는 끈을 무엇이라고 하시나요?},
  advq={고름은 어떻게 묶나요?}
}

% new lexicon
\Entry{
  word={구멍},
  pred={구녁, 구녕, 구먹, 굼기},
  feat={},
  desc={},
  qstn={%
    어떤 물건에 뻥 뚫려 있는 자리를 뭐라고 부르나요? \\
    종기 같은 게 곪아 터지면 그 자리가 뚫려 있잖아요. 그곳을 부르는 이름이 있나요?
  },
  advq={}
}

% C. 자연과 일상
%!TEX root = main.tex
\section{자연과 일상}

\subsection{자유발화 질문}
\begin{itemize}[noitemsep]
  \item 어린 시절에는 주로 무엇을 하고 노셨나요?
  \item 작은 동물들을 잡으며 노신 적이 있나요? 어떤 동물이었나요?
  \item 하루 일과가 어떻게 되세요? 시간대별로 설명해주실 수 있나요?
\end{itemize}

\subsection{문법 유도질문}
※ 시간, 장소, 동반 등의 의미를 가진 조사를 잘 확인하도록 합니다.

\begin{enumerate}[noitemsep]
  \item 주말에 주로 어떤 활동을 하시나요?
  \item 어디 가서 하시나요?
  \item 누구 데리고 가세요?
\end{enumerate}

\subsection{목표 어휘}
% \Entry 부분을 수정하시면 됩니다.
% word: 표준어형
% pred: 예상형태
% feat: 유의점
%       feat에 들어가는 인자는 쉼표(,)로 구분되며,
%       ms, sp, dp 세 종류가 있습니다.
%       feat에 들어간 인자가 볼드체로 표시됩니다.
% desc: 설명
% qstn: 유도질문
% advq: 심화질문
%
% 그리고 줄바꿈이 필요할 땐 엔터가 아니라 '\\'를 사용해주세요(중요!!).

\Entry{
  word={가위바위보},
  pred={},
  feat={ms,sp,dp},
  imag={img/refs/가위바위보},
  desc={V2 /ㅟ/ V4 /ㅟ/},
  qstn={이기고 지는 것을 가르기 위해 이렇게 이렇게 (조사자들끼리 가위바위보를 하는 시늉을 함) 하는 것을 무엇이라고 하나요?},
  advq={실제로 가위바위보를 할 때에 어떻게 말하시나요? (억양 확인)}
}

\newpage
\Entry{
  word={금긋다/금그어라},
  pred={긋다, 그셔라, 그스고, 근는다, 그꼬, 그스닝께, 그서서},
  feat={ms,sp,dp},
  imag={img/refs/금긋다-금그어라},
  desc={/금/(금속)과의 성조 음장 차이},
  qstn={땅따먹기 놀이를 할 때, 뼘을 잰 다음에는 바닥에 뭘 하나요? \\ (바닥에 금을 긋는 시늉을 하며) 이렇게 하는 걸 뭐라고 하나요?},
  advq={벽이나 담장이 갈라지면 어떻게 됐다고 하나요? (목표어휘: 금가다)}
}

\Entry{
  word={윷},
  pred={사짜},
  feat={ms,sp,dp},
  imag={img/refs/윷},
  desc={기저 종성 /ㅊ/ \\ /\jamoword{zyus/}/, /늇/ > /윷/},
  qstn={명절에 나무토막 네 개를 던지면서 한 칸을 가거나 다섯 칸까지 가거나 하면서 노는데, 그것을 무엇이라고 하시나요?},
  advq={`윷놀이'로 답변한 경우) 그때 던지는 막대기를 뭐라고 하시나요? \\ 이걸 던져서 나오는 눈들은 뭐라고 하시나요?}
}

\Entry{
  word={썰매},
  pred={썰매},
  feat={ms,sp,dp},
  imag={img/refs/썰매},
  desc={얼음/눈 위를 타는 것의 모양, 어형 차이 \\ 스케이트형이 나올 때 모양 차이 여부},
  qstn={얼음판이나 눈 위에서 타고 노는 것이 있는데, 뭘 탄다고 하시나요?},
  advq={썰매를 보통 어디에서 타곤 하나요? \\ 얼음판에서 타는 썰매랑 눈 위에서 타는 썰매가 다른가요?}
}

\Entry{
  word={얼레},
  pred={연자세},
  feat={ms,sp,dp},
  imag={img/refs/얼레},
  desc={/자새/형의 경우 대상물 확인 \\ 연줄/낚싯줄 감는 것 명칭 차이 여부},
  qstn={(*사진 자료 참고) 연을 날릴 때 실을 어디에 감으시나요?},
  advq={연줄 감는 거랑 낚싯줄 감는 것을 똑같이 얼레라고 하나요?}
}

\Entry{
  word={겨울},
  pred={겨울, 즑, 즐기 (/즐:기/ `겨울에', /즑/은 처격조사와의 연결에서만)},
  feat={ms,sp,dp},
  imag={img/refs/겨울},
  desc={C1 + V1 /겨/의 구개음화 여부 \\ /\jamoword{gye/zvr/}/, /겨을ㅎ/ > /겨울/},
  qstn={(*사진 자료 참고) 썰매 타고 하는 것은 어느 계절에 해요? \\ 가을 다음 계절을 뭐라고 하나요? (눈 오는 계절 금지)},
  advq={겨울철은 뭘 하면서 지내나요? 겨울철에 어떤 음식을 먹나요? \\ (반드시 조사 /-에/를 붙여 이야기하도록 답변을 유도한다.)}
}

\Entry{
  word={춥다},
  pred={춥다 (춥따, 춥꾸, 추:면, 추찌안타)},
  feat={ms,sp,dp},
  desc={/차다/와의 의미 차이},
  qstn={(C06)에는 날씨가 어떻다고 이야기하나요?},
  advq={추운 날씨가 오래 가면 올 겨울에는 무엇이 오래간다고 하나요? (목표 어휘: 추위)}
}

\Entry{
  word={차다},
  pred={차다(차기, 차니, 차서)},
  feat={ms,sp,dp},
  desc={/춥다/와의 의미 차이},
  qstn={얼음이 어떻다고 이야기하나요?(얼음을 만져 손이 시린 시늉을 하며) \\ 겨울철에는 물이 얼음장같이 어떠하다고 하죠.},
  advq={날이/날씨가 이렇다고 하기도 하나요? \\ 사람 성격을 가지고 이렇다고 하기도 하나요?}
}

\Entry{
  word={고드름},
  pred={고두룸, 고드름},
  feat={ms,sp,dp},
  imag={img/refs/고드름},
  desc={},
  qstn={겨울에 처마 밑에 주렁주렁 매달리는 어름덩어리는 무엇이라고 하시나요?},
  advq={이게 생기면 어떻게 하시나요? 없앨 때는 어떻게 없애나요?}
}

\Entry{
  word={눈(날씨)},
  pred={고두룸, 고드름},
  feat={ms,sp,dp},
  imag={img/refs/눈(날씨)},
  desc={/눈/(신체)과의 성조 음장 차이},
  qstn={비가 아니라, 하늘에서 하얗게 떨어져서 쌓이고 잘 뭉치는 것은 무엇인가요?},
  advq={눈이 내리면 뭘 하고 노나요?}
}

\Entry{
  word={소나기},
  pred={쏘내기},
  feat={ms,sp,dp},
  imag={img/refs/소나기},
  desc={},
  qstn={하늘이 맑다가 갑자기 비가 오면 뭐라고 하시나요? / 주로 여름에 별안간 세차게 쏟아지다가 곧 그치는 비를 무엇이라고 하시나요?},
  advq={집 밖에 있는데 소나기가 오면 어떻게 하세요? \\ 소나기를 맞으면 어떻게 옷이랑 몸을 말리나요?}
}

\Entry{
  word={벼락},
  pred={베락},
  feat={ms,sp,dp},
  imag={img/refs/벼락},
  desc={},
  qstn={하늘에서 번개가 치면서 나무나 사람에 떨어지면 무엇이 떨어진다고 하시나요? 갑자기 부자가 된 사람을 [이것] 부자라고도 해요.},
  advq={집 밖에 있는데 이게 치면 어떻게 하세요?}
}

\Entry{
  word={새벽},
  pred={새벽, 새복, 새벅},
  feat={ms,sp,dp},
  desc={},
  qstn={하루 중에서 해가 채 뜨지 않았을 무렵을 무엇이라고 하시나요? \\ 하루 중에서 아침이 밝기 직전일 때를 무엇이라고 하시나요?},
  advq={보통 몇 시까지를 새벽이라고 하세요? \\ 새벽에 잠이 깬 적 있으세요? 새벽에 깨면 몸이/기분이 어떠세요?}
}

\Entry{
  word={저녁},
  pred={저녁},
  feat={ms,sp,dp},
  desc={},
  qstn={하루 중에서 해가 막 질 무렵을 무엇이라고 하시나요?},
  advq={보통 몇 시부터를 저녁이라고 하세요? \\ 저녁 몇 시에 밥을 드세요?}
}

\Entry{
  word={밤[夜]},
  pred={밤},
  feat={ms,sp,dp},
  desc={/밤/(음식)과의 성조 음장 차이},
  qstn={하루 중에 해가 져서 캄캄할 때를 무엇이라고 하시나요?},
  advq={보통 몇 시부터를 밤이라고 하세요? \\ 보통 밤 몇 시에 주무시나요?}
}

\Entry{
  word={길다},
  pred={질다, 질:다, 길:다(지:러야, 기:러야)},
  feat={ms,sp,dp},
  imag={img/refs/길다},
  desc={C1 + V1 /기/의 구개음화 여부},
  qstn={(임의로 길이가 다른 두 물체를 제시하며) 이건 이것보다 어때요?},
  advq={요즘 날씨는 해가 늦게 떠서 빨리 지잖아요. 이것을 다른 말로 뭐라고 그러나요? (목표 표현: 해가 길다/짧다)}
}

\Entry{
  word={짧다},
  pred={짧다(짤따, 짤븐, 짤버서)},
  feat={ms,sp,dp},
  imag={img/refs/짧다},
  desc={},
  qstn={(임의로 길이가 다른 두 물체를 제시하며) 이건 이것보다 어때요?},
  advq={요즘 날씨는 해가 늦게 떠서 빨리 지잖아요. 이것을 다른 말로 뭐라고 그러나요? (목표 표현: 해가 길다/짧다)}
}

\Entry{
  word={하루, 이틀, …, 열흘},
  pred={하루, 이틀, 사을/사흘, 나을/나흘, 닷새/다쎄, 엿새/여쎄, 이레, 여드레, 아으레.아흐레, 여를},
  feat={ms,sp,dp},
  desc={/렷새/ > /엿새/  /니레/, /닐헤/ > /이레/ \\ /알흐래/ > /아흔/},
  qstn={하루, 이틀, 하는 식으로 열흘까지만 천천히 날짜를 세주시겠어요?},
  advq={여행을 갔다 오는데 하루가 걸린다는 문장에, /하루/의 자리에 다른 날짜를 넣어서 천천히 한 번 더 말씀해 주시겠어요?}
}

\Entry{
  word={내일},
  pred={니얼, (낼:, 내일)},
  feat={ms,sp,dp},
  desc={},
  qstn={오늘의 다음날은 무엇이라고 하시나요?},
  advq={혹시 내일 말고 다른 말도 있나요?}
}

\Entry{
  word={모레},
  pred={},
  feat={ms,sp,dp},
  desc={},
  qstn={(C19)의 다음날은 무엇이라고 하시나요?},
  advq={혹시 모레 말고 다른 말도 있나요?}
}

\Entry{
  word={글피},
  pred={글피},
  feat={ms,sp,dp},
  desc={},
  qstn={(C20)의 다음날은 무엇이라고 하시나요?},
  advq={혹시 글피 말고 다른 말도 있나요? / 자주 쓰시는 말인가요?}
}

\Entry{
  word={어제},
  pred={어제},
  feat={ms,sp,dp},
  desc={/어저께/와의 의미 차이},
  qstn={오늘의 전날은 무엇이라고 하시나요?},
  advq={혹시 어제 말고 다른 말도 있나요? \\ `어저께'라고 하면 같은 뜻인가요?}
}

\Entry{
  word={그저께},
  pred={그저끼},
  feat={ms,sp,dp},
  desc={/어저께/, /그제/와의 의미 차이 \\ /그젓긔/ > /그저께/},
  qstn={(C22)의 전날은 무엇이라고 하시나요?},
  advq={혹시 그저께 말고 다른 말도 있나요? \\ /그제/ 라고 하면 같은 뜻인가요?}
}

\Entry{
  word={그끄저께},
  pred={그끄저끼, 긋끄저끼},
  feat={ms,sp,dp},
  desc={},
  qstn={(C23)의 전날은 무엇이라고 하시나요?},
  advq={혹시 그끄저께 말고 다른 말도 있나요?}
}

\Entry{
  word={지렁이},
  pred={지렝이},
  feat={ms,sp,dp},
  imag={img/refs/지렁이},
  desc={/\jamoword{gez/que/zi/}/, /것위/ 등 계통이 다른 어휘 \\ /디롱이/, /디룡이/ > /지렁이/},
  qstn={비가 오면 땅바닥에 나와 기어다니는 것을 무엇이라고 부르나요?},
  advq={주로 어디서 지렁이를 많이 보셨어요?}
}

\Entry{
  word={미꾸라지},
  pred={미꾸리, 미끄락지},
  feat={ms,sp,dp},
  imag={img/refs/미꾸라지},
  desc={/믯구리/, /밋구리/ 등 이형태},
  qstn={길고 미끌미끌하고, 추어탕을 끓여 먹는 물고기는 무엇이라고 하시나요?},
  advq={미꾸라지는 어떻게 잡나요?}
}

\Entry{
  word={개구리},
  pred={개구락지, 깨구락지},
  feat={ms,sp,dp},
  imag={img/refs/개구리},
  desc={/먹자구/ 형과의 의미 차이에 유의},
  qstn={풀밭으로 나오기도 하고 물속에 들어가 살기도 하고, 폴짝폴짝 뛰어다니면서 우는 동물을 뭐라고 하시나요?},
  advq={뭔가를 시킬 때마다 반대로만 하는 사람을 뭐라고 하나요? (목표 어휘: 청개구리)}
}

\Entry{
  word={올챙이},
  pred={올챙이},
  feat={ms,sp,dp},
  imag={img/refs/올챙이},
  desc={},
  qstn={(*사진 자료 참고) (C27)가 되기 전에 알애서 막 태어난 새끼를 무엇이라고 하시나요?},
  advq={올챙이는 어디에 가야 많이 볼 수 있나요?}
}

\Entry{
  word={두꺼비},
  pred={두께비},
  feat={ms,sp,dp},
  imag={img/refs/두꺼비},
  desc={/두텁/, /두터비/ 등 이형태},
  qstn={(C27)이랑 비슷한데 조금 크고, 몸이 우둘투둘한 것은 무엇이라고 하시나요?},
  advq={(C27)이랑 두꺼비랑 또 어떤 차이가 있나요? \\ 두꺼비집 노래는 어떻게 부르나요?}
}

\Entry{
  word={거머리},
  pred={그머리, 금자리, 금저리, 그:머리},
  feat={ms,sp,dp},
  imag={img/refs/거머리},
  desc={},
  qstn={논에서 일할 때 다리에 달라붙어서 피를 빨아먹는 것을 무엇이라고 하시나요?},
  advq={거머리가 다리에 달라붙으면/거머리한테 물리면 어떻게 하나요?}
}

\Entry{
  word={달팽이},
  pred={달팽이},
  feat={ms,sp,dp},
  imag={img/refs/달팽이},
  desc={},
  qstn={등에 동글동글한 집을 달고 더듬이 두 개를 뻗고 기어다니는 동물은 무엇이라고 하시나요?},
  advq={논밭에 달팽이가 보일 때도 있나요? \\ 달팽이가 기는 속도를 어떻다고 하세요?}
}

\Entry{
  word={다슬기},
  pred={도슬비, 올뱅이},
  feat={ms,sp,dp},
  imag={img/refs/다슬기},
  desc={},
  qstn={(C31)이랑 비슷한데 냇물이나 강물에서 사는 것은 무엇이라고 하시나요? 된장국에 넣어먹기도 하고, 강바닥의 바위틈에서 주울 수 있어요.},
  advq={다슬기는 보통 어떤 식으로/어떻게 잡나요?}
}

\Entry{
  word={우렁이},
  pred={올갱이},
  feat={ms,sp,dp},
  imag={img/refs/우렁이},
  desc={},
  qstn={(C32)랑 비슷한데, 논에서 사는 것은 무엇이라고 하시나요? 조금 더 크고 둥글고 민물에 살고, 쌈밥이나 된장국으로 해먹기도 합니다.},
  advq={우렁이는 보통 어떤 식으로/어떻게 잡나요? \\ 우렁각시 이야기를 아시나요?}
}

\Entry{
  word={새끼(동물)},
  pred={새끼},
  feat={ms,sp,dp},
  desc={/새끼/(사물)와의 성조 음장 차이 \\ V1 /ㅐ-ㅔ/ 대립},
  qstn={송아지, 강아지, 망아지 같이 갓 태어난 어린 동물들을 무엇이라고 하시나요?},
  advq={소나 돼지가 새끼를 낳으면 무엇을 해줘야 하나요? \\ 새끼를 한 번에 여럿 낳을 때, 가장 먼저 나온 것을 뭐라고 하나요? (목표 어휘: 무녀리)}
}

\Entry{
  word={서캐},
  pred={서캐},
  feat={ms,sp,dp},
  desc={},
  qstn={머리에 생기는 이의 알은 무엇이라고 하시나요?},
  advq={서캐가 있으면 어떻게 없애나요?}
}

\Entry{
  word={구더기},
  pred={구데기, 구디기, 구:디기},
  feat={ms,sp,dp},
  desc={},
  qstn={(C37)이 알을 낳으면 태어나는 하얀 벌레를 뭐라고 하세요? `이것이 무서워서 장 못 담근다'는 속담이 있죠.},
  advq={집 안에 이것이 생기면 어떻게 하세요?}
}

\Entry{
  word={파리},
  pred={파리},
  feat={ms,sp,dp},
  imag={img/refs/파리},
  desc={},
  qstn={음식에 날아 앉는 날벌레인데 (지시) 손을 이렇게 비비는 것을 무엇이라고 하시나요?},
  advq={이걸 잡는 데 쓰는 도구는 무엇이라고 하시나요? (목표 어휘: 파리채)}
}

\Entry{
  word={벼룩},
  pred={베룩},
  feat={ms,sp,dp},
  imag={img/refs/벼룩},
  desc={},
  qstn={아주 조그마한데, 톡톡 튀어 다녀서 잡기가 어려운 벌레는 무엇이라고 하시나요? `뛰어야 이것이지'라는 말이 있죠},
  advq={}
}

\Entry{
  word={벌},
  pred={부리, 벌:},
  feat={ms,sp,dp},
  imag={img/refs/벌},
  desc={V1 /ㅓ/ 장단음},
  qstn={꿀을 만드는 곤충인데, 집을 건드리면 떼 지어 쫓아와서 쏘는 것을 무엇이라고 하시나요?},
  advq={벌을 집 안이나 바깥에서 만나면 어떻게 하시나요? \\ 되도록 `쏘이다'를 수집한 이후) 벌에 쏘였을 때 어떻게 하나요?}
}

\Entry{
  word={쥐},
  pred={지, 쥐(이중모음)},
  feat={ms,sp,dp},
  imag={img/refs/쥐},
  desc={V1 /ㅟ/},
  qstn={밤에 찍찍거리고 돌아다니면서 곡식을 훔쳐 먹는 동물을 무엇이라고 하시나요?},
  advq={쥐가 집안에 들어온 적이 있나요? \\ 쥐가 들어오면 어떻게 잡으세요?}
}

\Entry{
  word={돼지},
  pred={도야지, 되야지, 돼:지},
  feat={ms,sp,dp},
  imag={img/refs/돼지},
  desc={/돗/, /도야지/ 등 이형태},
  qstn={뚱뚱하고, 꿀꿀거리면서 울고, 고기로 많이 해먹는 동물을 무엇이라고 하시나요?},
  advq={돼지고기는 어떤 부위를 가장 좋아하세요? \\ 돼지들은 주로 뭘 먹이면서 키우나요?}
}

\Entry{
  word={고양이},
  pred={고이, 괭이, 고얭이, 괭:이, 고양이},
  feat={ms,sp,dp},
  imag={img/refs/고양이},
  desc={/괴/, /괴양이/ 등 이형태},
  qstn={(C40)을 잘 잡고, 야옹 하고 우는 동물을 뭐라고 하시나요?},
  advq={고양이를 키우시나요? \\ 아시는 분들 중에 고양이를 키우는 분이 계신가요?}
}

\Entry{
  word={개},
  pred={가이},
  feat={ms,sp,dp},
  imag={img/refs/개},
  desc={V1 /ㅐ-ㅔ/ 대립 \\ /가히/, /갛/ > /개/},
  qstn={집에서 기르는 동물인데, `멍멍' 하고 짓는 것을 뭐라고 하시나요?},
  advq={}
}

% new lexicon
\Entry{
  word={여우},
  pred={여시, 여꾸},
  feat={ms,dp},
  imag={img/refs/여우},
  desc={다양한 조사 결합형을 조사할 수 있도록 할 것; /\jamoword{ye/zu/}{\textasciitilde}\jamoword{yez/q/}/ > /여우/},
  qstn={개와 비슷하게 생긴 야생 동물인데, 털빛이 붉고 귀가 뾰족한 것을 뭐라고 하시나요?},
  advq={여우와 관련된 전승을 아는 게 있으신가요?}
}

\Entry{
  word={꿩},
  pred={꿩},
  feat={ms,sp,dp},
  imag={img/refs/꿩},
  desc={V1 /ㅝ/},
  qstn={사냥꾼이 많이 잡는 새인데, 숨을 때 머리만 처박는 것을 무엇이라고 하시나요? (`이것 대신 닭'이라는 말이 있죠)},
  advq={수놈과 암놈을 따로 부르는 말이 있나요? 새끼를 따로 부르는 말도 있나요?}
}

\Entry{
  word={매},
  pred={매},
  feat={ms,sp,dp},
  imag={img/refs/매},
  desc={V1 /ㅐ/ 장단음 \\ V1 /ㅐ-ㅔ/ 대립},
  qstn={독수리랑 비슷한 새인데, 길들여서 사냥에 쓰기도 하는 새를 무엇이라고 하시나요? (`이것의 눈'이라는 말이 있죠)},
  advq={}
}

\Entry{
  word={말(동물)},
  pred={말},
  feat={ms,sp,dp},
  imag={img/refs/말(동물)},
  desc={/말/(언어)과의 성조 음장 차이},
  qstn={아주 빠르게 달리고, 마차를 끄는 동물을 무엇이라고 하시나요?},
  advq={}
}

\Entry{
  word={마구간},
  pred={마구깐},
  feat={ms,sp,dp},
  imag={img/refs/마구간},
  desc={},
  qstn={(C47)이 사는 집은 뭐라고 하시나요?},
  advq={}
}

\Entry{
  word={외양간},
  pred={오양깐, 오양간},
  feat={ms,sp,dp},
  imag={img/refs/외양간},
  desc={/오희양/, /외향/ > /외양간/},
  qstn={소가 사는 집은 무엇이라고 하시나요? \\ `소 잃고 이거 고친다'는 말이 있죠},
  advq={}
}

% D. 바다
%!TEX root = main.tex
\section{바다}
\subsection{일러두기}
일러둘 내용을 적습니다.

\subsection{자유발화 질문}
자유발화 질문을 적습니다.

\subsection{목표 어휘}
% \Entry 부분을 수정하시면 됩니다.
% word: 표준어형
% pred: 예상형태
% feat: 유의점
%       feat에 들어가는 인자는 쉼표(,)로 구분되며,
%       ms, sp, dp 세 종류가 있습니다.
%       feat에 들어간 인자가 볼드체로 표시됩니다.
% desc: 설명
% qstn: 유도질문
% advq: 심화질문
%
% 그리고 줄바꿈이 필요할 땐 엔터가 아니라 '\\'를 사용해주세요(중요!!).

\Entry{
  word={어부},
  pred={},
  feat={ms,sp,dp},
  imag={img/refs/어부},
  desc={},
  qstn={생선이나 그런 해산물을 잡아서 파시는 분들을 뭐라고 부르나요?},
  advq={이런 분들이 모여서 사는 곳이 있나요? 어떤 곳인가요?}
}

\Entry{
  word={그물},
  pred={},
  feat={ms,sp,dp},
  imag={img/refs/그물},
  desc={},
  qstn={물고기를 잡는 방법에는 어떤 것이 있을까요? \\ 요즘 전어가 철인데, 전어를 어떻게 잡는지 아시나요?},
  advq={그물은 배 위에서 던지는 방법만 쓰나요? 아니면 물고기나 다른 해산물들을 잡을 수 있는 여러 방법이 있나요?}
}

\Entry{
  word={아가미},
  pred={아가미},
  feat={ms,sp,dp},
  imag={img/refs/아가미},
  desc={},
  qstn={생선 손질해 보신 적 있으신가요? 어떻게 손질하나요? \\ 물고기가 물속에서 숨을 쉴 수 있게 해주는 부위를 무엇이라고 하나요?},
  advq={게, 새우가 이미 관찰되었을 시) 물고기 말고 게나 새우에도 아가미나 비슷한 역할을 하는 부위가 있나요? 그 부분을 무엇이라고 부르나요?}
}

\Entry{
  word={지느러미},
  pred={지느레미},
  feat={ms,sp,dp},
  imag={img/refs/지느러미},
  desc={},
  qstn={생선 손질해 보신 적 있으신가요? 어떻게 손질하나요? 물고기의 꼬리에서 파닥거려서 헤엄칠 수 있게 하는 부위를 무엇이라고 하나요?},
  advq={꼬리 말고 등이나 옆면에 달린 비슷한 부위도 지느러미라고 하나요?}
}

\Entry{
  word={바닷물},
  pred={},
  feat={ms,sp,dp},
  imag={img/refs/바닷물},
  desc={},
  qstn={바다는 개울물과 다르게 짠데, 그 짠 물을 무엇이라고 부르나요?},
  advq={짠 물은 모두 그렇게 부르나요? 아니면 바다에 있는 짠물만 그렇게 부르나요?}
}

\Entry{
  word={파랗다},
  pred={},
  feat={ms,sp,dp},
  imag={img/refs/파랗다},
  desc={},
  qstn={날씨가 좋은 날, 바다나 하늘의 색깔을 어떻다고 말씀하시나요?},
  advq={바다의 색깔을 표현하는 다른 말도 있을까요?}
}

\Entry{
  word={밀물},
  pred={},
  feat={ms,sp,dp},
  imag={img/refs/밀물},
  desc={},
  qstn={서산 바다는 물이 들어왔다가, 나갔다가 하잖아요. 이렇게 들어오는 물이나 물이 들어오는 그때를 무엇이라고 부르시나요?},
  advq={날에 따라서 물이 들어오는 정도가 다른가요? 많이 들어오는 날은 무엇이라고 부르나요?}
}

\Entry{
  word={썰물},
  pred={},
  feat={ms,sp,dp},
  imag={img/refs/썰물},
  desc={/혈물/ > /썰물/},
  qstn={서산 바다는 물이 들어왔다가, 나갔다가 하잖아요. 이렇게 나가는 물이나 물이 빠지는 그때를 무엇이라고 부르시나요?},
  advq={날에 따라서 물이 빠지는 정도가 다른가요? 많이 빠지는 날은 무엇이라고 부르나요?}
}

\Entry{
  word={갯벌},
  pred={},
  feat={ms,sp,dp},
  imag={img/refs/갯벌},
  desc={개펄과의 의미 차이 여부},
  qstn={조개나 소라를 잡아보신 적이 있으신가요? 어떻게 잡으셨나요? \\ 바다에 물이 싹 빠지고 드러난 질퍽질퍽한 땅을 무엇이라고 부르나요?},
  advq={갯벌에서는 어떤 해산물을 잡을 수 있나요? \\ 갯벌 일을 전문적으로 하시는 분들을 부르는 이름이 있나요?}
}

\Entry{
  word={곶},
  pred={},
  feat={ms,sp,dp},
  imag={img/refs/곶},
  desc={기저 종성 /ㅈ/},
  qstn={황금산과 몽돌해변은 어느 동네에 있나요? (독곶, 독곶리) \\ 독곶리처럼, 바다 쪽으로, 뾰족하게 뻗은 육지를 무엇이라고 하나요?},
  advq={독곶리를 부르는 다른 이름은 없나요? \\ 옛날에는 그 지역을 뭐라고 불렀나요?}
}

\Entry{
  word={만},
  pred={},
  feat={ms,sp,dp},
  imag={img/refs/만},
  desc={},
  qstn={서산 북쪽 바다(가로림만)와 남쪽 바다(천수만)를 무엇으로 부르나요? \\ 바다 쪽에서 육지 속으로 파고들어 와 있는 곳을 무엇이라고 하나요?},
  advq={가로림만이나 천수만을 부르는 다른 이름은 없나요? \\ 옛날에는 그 바다를 뭐라고 불렀나요?}
}

\Entry{
  word={통발},
  pred={},
  feat={ms,sp,dp},
  imag={img/refs/통발},
  desc={},
  qstn={물고기나 다른 해산물을 잡을 때, 어떤 방법들이 있나요? \\ 10월이 꽃게 철이었는데, 꽃게잡이는 어떻게 하는지 아시나요?},
  advq={통발에는 주로 무엇이 잡히나요? \\ 통발 말고 물고기를 가두어서 잡는 방법은 또 없나요?}
}

\Entry{
  word={새우},
  pred={물쌔우, 새우, 대화, 새우, 새오},
  feat={ms,sp,dp},
  imag={img/refs/새우},
  desc={민물 or 바다, 크기에 따른 어휘 차이 \\ /\jamoword{sa/bqi/}/ > ???},
  qstn={얼마 전에 철이었는데, 가재랑 닮았고 등이 굽어있는 해산물을 뭐라고 부르나요? 크기에 따라서 먹는 방법이 조금씩 다릅니다.},
  advq={큰 것과 작은 것을 부르는 이름이 다른가요? 새우는 큰 것만 가리키나요 작은 것만 가리키나요?}
}

\Entry{
  word={게},
  pred={그이},
  feat={ms,sp,dp},
  imag={img/refs/게},
  desc={V1 /ㅐ-ㅔ/ 대립},
  qstn={서산의 유명한 해산물에는 뭐가 있나요? \\ 집게발을 가졌고, 옆으로 걸어다니는 해산물을 무엇이라고 부르나요?},
  advq={게를 사용한 맛있는 요리를 소개해 주실 수 있나요? \\ 게는 어떻게 손질하나요?}
}

\Entry{
  word={바위},
  pred={},
  feat={ms,sp,dp},
  imag={img/refs/바위},
  desc={V2 /ㅟ/ \\ 바회, 바히... > 바위},
  qstn={꽃게 말고, 갯벌에서 보이는 작은 게들은 보통 어디에 사나요? / 고둥이나 따개비는 어디에 붙어서 사나요? / 큰 돌을 뭐라고 부르나요?},
  advq={바다에 잠겨있어서 보일락말락 하는 커다란 돌도 바위라고 부르나요?}
}

% E. 사회적 지식
%!TEX root = main.tex
\section{사회적 지식}
\subsection{일러두기}
일러둘 내용을 적습니다.

\subsection{자유발화 질문}
자유발화 질문을 적습니다.

\subsection{목표 어휘}
% \Entry 부분을 수정하시면 됩니다.
% word: 표준어형
% pred: 예상형태
% feat: 유의점
%       feat에 들어가는 인자는 쉼표(,)로 구분되며,
%       ms, sp, dp 세 종류가 있습니다.
%       feat에 들어간 인자가 볼드체로 표시됩니다.
% desc: 설명
% qstn: 유도질문
% advq: 심화질문
%
% 그리고 줄바꿈이 필요할 땐 엔터가 아니라 '\\'를 사용해주세요(중요!!).

\Entry{
  word={마을},
  pred={동네/마실},
  feat={ms,sp,dp},
  imag={img/refs/마을},
  desc={/ᄆᆞᅀᆞᇕ.../ > /마을/},
  qstn={지금 사시는 곳 or 예전에 살아오셨던 곳을 소개해주세요. \\ 집집들이 모여서 서로 돕고 사는 곳을 뭐라고 부르나요?},
  advq={옆 동네/옆 마을과는 어떤 관계인가요?}
}

\Entry{
  word={외국},
  pred={},
  feat={ms,sp,dp},
  imag={img/refs/외국},
  desc={V1 /ㅚ/},
  qstn={우리나라 말고 다른 곳으로 여행 가보신 적 있으신가요? / 어디를 가고 싶으신가요? / 우리나라가 아닌 다른 나라를 무엇이라고 부르나요?},
  advq={다른 가족분들이 외국에 대해 해주신 이야기가 있을까요?}
}

\Entry{
  word={어디},
  pred={위디, 워디, 어디서:, 어서, 어이서, 워이서, 워디서},
  feat={ms,sp,dp},
  desc={},
  qstn={친구분이 급하게 막 가고 있을 때, 어떤 곳을 향해 가는지 어떻게 물어보시나요? / 어떤 곳에서 왔는지는 어떻게 물어보나요?},
  advq={어린 아이라면 어떻게 물어보시겠어요? \\ 어머니 혹은 아버지라면 어떻게 물어보시겠어요?}
}

\Entry{
  word={모시다/모셔라},
  pred={메시다. 모시고, 모셔야},
  feat={ms,sp,dp},
  imag={img/refs/모시다-모셔라},
  desc={},
  qstn={자녀분들이 성인이 되고 나서, 혹은 성인이 되고 나신 후 부모님과 함께 여행하신 적이 있으신가요? / 그분들과 어떻게 여행을 다니셨나요?},
  advq={지금 자녀분들과 함께, 혹은 과거로 돌아가 부모님과 함께 가고 싶은 여행지가 있으신가요?}
}

\Entry{
  word={하나, 둘, ... 열},
  pred={하나, 둘:, 싯/스이/셋:, 닛/느이/넷:, 다섯, 여섯, 일곱, 여덜, 아홉, 열:},
  feat={ms,sp,dp},
  desc={},
  qstn={물건을 세듯이 하나부터 열까지만 천천히 숫자를 세주시겠어요?},
  advq={너희 중 하나가 밥을 먹느냐는 질문에, /하나/의 자리에 다른 숫자를 넣어서 천천히 한 번 더 말씀해 주시겠어요?}
}

\Entry{
  word={스물, 서른, ... 아흔},
  pred={스물, 서른, 마은/마흔, 쉰/시운, 예순, 이른, 여든, 아은/아흔},
  feat={ms,sp,dp},
  desc={/마ᅀᅳᆫ/ 관련 이형태 \\ /스믈ㅎ.../ > /스물//셜흔.../ > /서른/ \\ /려쉰.../ > /예순//닐흔.../ > /일흔/},
  qstn={이번에는 스물, 서른 하는 식으로 백까지만 천천히 숫자를 세주시겠어요?},
  advq={내 나이가 열이다라는 문장에, /열/의 자리에 스물, 서른 등의 숫자를 넣어서 천천히 한 번 더 말씀해주시겠어요?}
}

\Entry{
  word={에누리},
  pred={에느리},
  feat={ms,sp,dp},
  desc={},
  qstn={시장에서 물건을 사실 때, 무조건 써진 가격으로만 사야 하나요? \\ 조금 싸게 사려고 가격을 깎는 것을 무엇이라고 하나요?},
  advq={에누리를 잘하려면 어떻게 해야 하나요?}
}

\Entry{
  word={되},
  pred={데/되(단모음)},
  feat={ms,sp,dp},
  imag={img/refs/되},
  desc={V1 /ㅚ/},
  qstn={시장에서 콩이나 팥 같은 것을 어떤 단위로 파나요? \\ 열 홉이 모이면 무엇이 되나요?},
  advq={한 되씩 혹은 두세 되씩 파는 물건에는 또 무엇이 있나요?}
}

\Entry{
  word={말(단위)},
  pred={말},
  feat={ms,sp,dp},
  imag={img/refs/말(단위)},
  desc={/말/(동물), /말/(언어)와의 성조 음장 차이},
  qstn={시장에서 기름을 짜기 위한 들깨 같은 것을 어떤 단위로 파나요? \\ 열 되가 모이면 무엇이 되나요?},
  advq={한 말씩 혹은 두세 말씩 파는 물건에는 또 무엇이 있나요?}
}

\Entry{
  word={배(연산)},
  pred={},
  feat={ms,sp,dp},
  desc={/배/(과일), /배/(신체)와의 성조 음장 차이},
  qstn={한 말은 한 되의 열 무엇이라고 하시나요? \\ 곱절을 다른 말로 무엇이라고 하시나요?},
  advq={n 배가 관찰된 경우) 혹시 n을 빼고 발음하면 어떻게 되나요?}
}

\Entry{
  word={금(금속)},
  pred={},
  feat={ms,sp,dp},
  imag={img/refs/금(금속)},
  desc={/금긋다/의 /금/과의 성조 음장 차이},
  qstn={시장에서는 보석도 파나요? / 비싼 보석이나 반지, 목걸이는 어디서 파나요? / 결혼하실 때 그곳에서 반지(가락지)를 맞추셨나요?},
  advq={금덩어리를 정해진 양으로 네모낳게 만든 것을 무엇이라고 하나요?}
}

% F. 신체
%!TEX root = main.tex
\section{신체}
\subsection{일러두기}
일러둘 내용을 적습니다.

\subsection{자유발화 질문}
\begin{itemize}[noitemsep]
  \item 요즘 날이 추운데, 어디 불편하시거나 편찮으신 데는 없으세요?
  \item 감기에 걸리면 여기저기가 많이 아프잖아요, 어디가 아프다고 주로 얘기하세요?
\end{itemize}

\subsection{목표 어휘}
% \Entry 부분을 수정하시면 됩니다.
% word: 표준어형
% pred: 예상형태
% feat: 유의점
%       feat에 들어가는 인자는 쉼표(,)로 구분되며,
%       ms, sp, dp 세 종류가 있습니다.
%       feat에 들어간 인자가 볼드체로 표시됩니다.
% desc: 설명
% qstn: 유도질문
% advq: 심화질문
%
% 그리고 줄바꿈이 필요할 땐 엔터가 아니라 '\\'를 사용해주세요(중요!!).

\Entry{
  word={위(방향)},
  pred={욱},
  feat={sp,dp},
  imag={img/refs/위(방향)},
  desc={V1 /ㅟ/ \\ /웋/ > /위/},
  qstn={%
    요통: 그러면 (손으로 가리키며) 여기 엉치뼈 쪽이 아프신 거에요? \\
    두통: 옆에서 누르는 듯이 아프세요? 아니면…. (손으로 머리를 누르며)
  },
  advq={많이 힘드시겠어요…. 그 근처에 더 불편한 곳은 없으세요?}
}

\Entry{
  word={아래},
  pred={아레},
  feat={sp},
  imag={img/refs/아래},
  desc={V2 /ㅐ-ㅔ/ 대립},
  qstn={%
    요통: 그러면 (손으로 가리키며) 여기 날개뼈 근처가 아프신 거에요? \\
    무릎 통증: 여기 무릎 위에 톡 튀어나온 부분이 아프세요? 아니면….
  },
  advq={많이 힘드시겠어요…. 그 근처에 더 불편한 곳은 없으세요?}
}

\Entry{
  word={머리카락},
  pred={멀크락},
  feat={},
  imag={img/refs/머리카락},
  desc={},
  qstn={(지시) 이것은 무엇이라고 하시나요?},
  advq={요즘 사람들 보면, 색깔도 바꾸고 파마도 하고 하잖아요, 어떻게 보이세요? 어떤 게 제일 예쁘다고 생각하세요?}
}

\Entry{
  word={비듬},
  pred={지게미},
  feat={},
  imag={img/refs/비듬},
  desc={},
  qstn={날이 추워지거나 머리를 잘 안 감으면, 머리에서 하얀 게 떨어지잖아요, 그걸 뭐라고 부르세요?},
  advq={그런 (비듬이 많은/비듬이 막 떨어지는) 사람을 보신 적이 있으세요? 그러면 혹시 조금 더럽다고 느껴질까요?}
}

\Entry{
  word={눈(신체)},
  pred={눈},
  feat={sp},
  imag={img/refs/눈(신체)},
  desc={/눈/(날씨)과의 성조·음장 차이},
  qstn={%
    안경이나 돋보기 같은 것을 사용하세요? 항상 필요하신 편이세요? 언제부터 필요하다고 생각하셨어요? \\
    (지시) 이것은 무엇이라고 하시나요?
  },
  advq={젊으셨을 적에는 어땠어요? (군필 남성의 경우 사격 이야기를 곁들여도 좋을 듯)}
}

\Entry{
  word={귀},
  pred={귀},
  feat={sp},
  imag={img/refs/귀},
  desc={V1 /ㅟ/},
  qstn={%
    (지시) 이것은 무엇이라고 하시나요? \\
    소리를 잘 듣는 사람에게 무엇이 밝다고 이야기하시나요?
  },
  advq={무언가를 잘 들으려고 집중할 때, (귀에 손을 모아 가져다 대고 허리를 살짝 숙이며) 이렇게 하는 것을 어떻게 한다고 이야기하시나요?}
}

\Entry{
  word={말(언어)},
  pred={말:},
  feat={sp},
  desc={/말/(동물), /말/(단위)과의 성조·음장 차이},
  qstn={사람들이 입으로 소리를 내어서 다른 사람에게 뭐라고 하는 것을 무엇이라고 하시나요? (또는 말과 관련된 여러 속담 사용)},
  advq={}
}

\Entry{
  word={혀},
  pred={세},
  feat={ms,sp},
  imag={img/refs/혀},
  desc={/혓바닥/과의 의미 차이 \\ C1 + V1 /혀/},
  qstn={(지시) 이것은 무엇이라고 하시나요?},
  advq={혀랑 관련된 재밌는 표현들 없을까요? 흔히 말을 잘하는 사람한테 혀를 잘 놀린다고 말하기도 하잖아요.}
}

\Entry{
  word={딸꾹질},
  pred={포깍질, 포꼭질},
  feat={},
  imag={img/refs/딸꾹질},
  desc={},
  qstn={밥을 갑자기 빨리 먹거나 목이 마르면 목에서 뭔가 꺽꺽하고 올라오는데 그것을 무엇이라고 하시나요?},
  advq={그걸 멈추려면 어떻게 하는 게 가장 효과가 좋으셨나요?}
}

\Entry{
  word={언청이},
  pred={얼쳉이},
  feat={ms},
  desc={/결슌/ 등 계통이 다른 어휘},
  qstn={태어날 때부터 윗입술이 찢어진 사람을 무엇이라고 하시나요? (비칭임에 유의)},
  advq={그런 사람을 보신 적이 있으신가요?}
}

\Entry{
  word={하품},
  pred={하얌},
  feat={},
  imag={img/refs/하품},
  desc={},
  qstn={잠이 올 때 입이 저절로 벌어지면서 숨을 크게 들이쉬는 것을 무엇 한다고 하시나요?},
  advq={이것이 혹시 버릇없는 행동일 수 있을까요? 그렇다면 하품이 나올 때 지켜야 할 예의가 있을까요? (손으로 입을 가린다거나 하는)}
}

\Entry{
  word={가래},
  pred={가레},
  feat={},
  desc={},
  qstn={%
    감기 걸리면 무엇 때문에 가장 힘드세요? \\
    감기에 걸리면 목에 잔뜩 끼어서 기침이 나오게 하는 것을 무엇이라고 부르시나요?
  },
  advq={저는 아무리 기침을 해도 잘 안 뱉어지더라구요. 혹시 잘 뱉는 법을 아시나요?}
}

\Entry{
  word={목},
  pred={목},
  feat={ms},
  imag={img/refs/목},
  desc={/모가지/와의 의미 차이},
  qstn={%
    감기 걸리면 어디가 제일 아프세요? / 혹시 지금 자세는 편안하세요? \\
    지금까지 말씀 많이 하셨는데, 어디 불편하신 데는 없으세요?
  },
  advq={동물을 잡을 때도 여기를 먼저 치나요?}
}

\Entry{
  word={뺨},
  pred={빰:},
  feat={dp,ms},
  imag={img/refs/뺨},
  desc={/뺨따귀/ 등이 비칭인지 \\ /볼/과의 차이},
  qstn={%
    서로 싸울 때, 화가 나면 어디를 치기도 하지요? \\
    흥부가 놀부 부인한테 주걱으로 어디를 맞았다고 하나요?
  },
  advq={친구끼리 장난으로 여기를 때리는 것도 문제가 될 수 있을까요?}
}

\Entry{
  word={주름살},
  pred={주룸쌀},
  feat={},
  imag={img/refs/주름살},
  desc={},
  qstn={%
    나이가 드시고 난 뒤에 가장 크게 달라진 게 무엇이라고 생각하시나요? \\
    나이가 드신 분의 피부가 자글자글해지는 것을 무엇이라고 부르시나요?
  },
  advq={부모님의 것을 볼 때/자신의 것을 볼 때/자녀의 것을 볼 때 어떤 기분이 드셨나요?}
}

\Entry{
  word={배(신체)},
  pred={베},
  feat={sp},
  imag={img/refs/배(신체)},
  desc={/ㅔ/와 /ㅐ/의 구분(전남 일부 지역에는 남아 있다고 보고됨); /배/(과일), /배/(계산)와의 성조·음장 차이},
  qstn={소화가 잘 안되시지는 않으세요? 어떤 음식이 잘 안 받는 것 같다거나 하시는 건 없으세요? 그 음식을 드시면 어떠세요?},
  advq={윗배가 아픈 것과 아랫배가 아픈 게 어떻게 다른가요?}
}

\Entry{
  word={무릎},
  pred={물팍},
  feat={sp},
  imag={img/refs/무릎},
  desc={기저 종성 /ㅍ/},
  qstn={오래 걸으시면 어디가 주로 아프세요? \\ 비가 올 때 주로 어디가 쑤시는 편이세요?},
  advq={많이 불편하시겠어요…. 조금이라도 덜 아프려면 어떻게 해야 되나요?}
}

\Entry{
  word={뼈},
  pred={뻬다구},
  feat={},
  imag={img/refs/뼈},
  desc={},
  qstn={(지시) 살 속에 들어 있는 딱딱한 것이 무엇인가요?},
  advq={닭이나 생선 말고, 다른 동물 뼈를 보신 적 있으세요? \\ 동물 뼈도 사람 뼈랑 비슷하게 생겼나요?}
}

\Entry{
  word={고름(신체)},
  pred={고름},
  feat={sp},
  desc={V2 /ㅡ/ \\ /고름/(의복)과의 비교},
  qstn={상처가 바로 낫지 않고 곪으면 노랗게 무엇이 나오나요?},
  advq={
    고름이 생기면 짜는 게 좋을까요, 아니면 그대로 두는 게 좋을까요? 혹시 이유도 아시나요? \\
    (B41 구멍 관련) 종기 같은 게 곪아 터지면 그 자리가 뚫려 있잖아요. 그곳을 부르는 이름이 있나요?
  }
}

% new lexicon
\Entry{
  word={부풀다},
  pred={부크다},
  feat={dp},
  desc={C2 /ㅍ/, V2 /ㅜ/},
  qstn={%
    기름진 음식 먹으면 소화가 잘 안 되고, 배에 가스가 차고 그렇잖아요. 이럴 때 배가 어떻게 된다고 하시나요? \\
    사람이 임신하면 배가 이렇게 차오르잖아요. 그걸 배가 어떻게 된다고 하시나요? \\
    풍선에 바람을 넣으면 풍선이 어떻게 된다고 하시나요?
  },
  advq={}
}

\Entry{
  word={엄살},
  pred={엄살},
  feat={},
  desc={},
  qstn={별로 아프지도 않은데 거짓말로 아프다고 막 소리치는 것을 걸 무엇 피운다, 혹은 무엇 부린다고 하나요?},
  advq={누가 아프다고 말할 때 엄살인지 아닌지 알 수 있는 방법이 있을까요?}
}

\Entry{
  word={귀신},
  pred={}, % TODO: 예상형태 기재
  feat={sp},
  imag={img/refs/귀신},
  desc={V1 /ㅟ/},
  qstn={사람이 죽으면 이것이 된다고도 하고, 한을 많이 품고 있어서 산 사람을 괴롭힌다고 하는데, 무엇을 표현하는 말인가요?},
  advq={실제로 귀신을 보신 적이 있으신가요? \\ 혹시 어떤 종류의 귀신들이 있는지 아시나요?}
}

\Entry{
  word={방귀},
  pred={방:구},
  feat={ms},
  imag={img/refs/방귀},
  desc={/똥:/도 쓰이는지 확인할 것},
  qstn={배가 아프면 냄새나는 가스가 나오기도 하는데, 이걸 뭐라고 하나요? \\ 고구마 많이 먹으면 어떤 냄새가 지독해지나요?},
  advq={(F23 구린내와 함께 진행)}
}

\Entry{
  word={구린내},
  pred={},
  feat={ms},
  desc={/고린내/와의 의미 차이},
  qstn={그러면 방귀를 뀌면 어떤 냄새가 난다고 하나요?},
  advq={(/고린내/도 검출된 경우) 두 냄새는 어떤 차이가 있나요?}
}

\Entry{
  word={고린내},
  pred={꼬랑네},
  feat={ms},
  desc={/구린내/와의 의미 차이},
  qstn={안 씻은 발이나 발가락 사이에서는 어떤 냄새가 난다고 하나요?},
  advq={(/구린내/도 검출된 경우) 두 냄새는 어떤 차이가 있나요?}
}

\Entry{
  word={씻다/씻어라},
  pred={씿다, 씨끄다, 싯다}
  feat={dp},
  imag={img/refs/씻다-씻어라},
  desc={/싯다/, /슺다/... > /씻다/; 어원적으로 사동 표지가 개재되었을 가능성도 있음},
  qstn={기분 나쁜 냄새가 나거나 꼬질꼬질해 보이는 사람한테는 뭐라고 명령하시겠어요?},
  advq={목욕이든 샤워든 했는지 물어볼때는 어떻게 물어보세요? \\ 이제 목욕한다고 가족들한테 얘기하실 때 어떻게 말씀하세요?}
}

\Entry{
  word={굵다},
  pred={굵:다, 꿁다, 퉁겁다(퉁거와)},
  feat={ms},
  imag={img/refs/굵다},
  desc={/두껍다/와의 의미 차이},
  qstn={엄지손가락은 새끼손가락보다 어떻다고 하죠?},
  advq={사전이나 전화번호부는 공책보다 굵은 건가요, 두꺼운 건가요? \\ 콩은 쌀보다 알이 굵은가요, 두꺼운가요?}
}

\Entry{
  word={가늘다},
  pred={가늘다, 가늘허다},
  feat={ms},
  imag={img/refs/가늘다},
  desc={/얇다/와의 의미 차이},
  qstn={새끼손가락은 엄지손가락보다 어떻다고 하죠?},
  advq={공책은 사전이나 전화번호부보다 가는 건가요, 얇은 건가요? \\ 보슬비는 억세게 내리는 장맛비보다 빗방울이 가는가요, 얇은가요?}
}

% G. 기타 동사
%!TEX root = main.tex
\section{기타 동사}

\subsection{목표 어휘}
% \Entry 부분을 수정하시면 됩니다.
% word: 표준어형
% pred: 예상형태
% feat: 유의점
%       feat에 들어가는 인자는 쉼표(,)로 구분되며,
%       ms, sp, dp 세 종류가 있습니다.
%       feat에 들어간 인자가 볼드체로 표시됩니다.
% desc: 설명
% qstn: 유도질문
% advq: 심화질문
%
% 그리고 줄바꿈이 필요할 땐 엔터가 아니라 '\\'를 사용해주세요(중요!!).

\Entry{
  word={매맞다/매맞아라},
  pred={}, % TODO: 예상형태 기재
  feat={sp},
  imag={img/refs/매맞다-매맞아라},
  desc={V1 /ㅐ/ 장단음 \\ V1 /ㅐ-ㅔ/ 대립},
  qstn={어린 시절 부모님은 엄한 분이셨나요? / 어른이 혼을 내면서 회초리 같은 것을 떄리고 있으면, 아이가 무엇을 맞고 있다고 하나요?},
  advq={(매를 맞은 경우) 그때는 친구분들도 대부분 그랬나요? \\ (매를 맞지 않은 경우) 혹시 부모님이 그러시지 않은 이유를 아시나요?}
}

\Entry{
  word={잃다/잃어버리다},
  pred={}, % TODO: 예상형태 기재
  feat={ms},
  desc={/잊어버리다/와의 어휘 병합, 의미 차이},
  qstn={소중하게 가지고 계시던 게 없어졌던 적이 있으신가요? / 가지고 있던 물건이 갑자기 어디론가 없어지면 그것을 어떻게 했다고 하시나요?},
  advq={잃어버리셔서 많이 슬프셨겠어요. 도둑맞거나 한 것은 아니지요?}
}

\Entry{
  word={잊다/잊어버리다},
  pred={}, % TODO: 예상형태 기재
  feat={ms},
  desc={/잃어버리다/와의 어휘 병합, 의미 차이},
  qstn={생각하고 있던 것이 머릿속에서 생각 나지 않을 때는 어떻게 했다고 하시나요?},
  advq={무언가 잊어버리셔서 곤란하신 적이 있었나요?}
}

\newpage
\Entry{
  word={가르치다},
  pred={}, % TODO: 예상형태 기재
  feat={ms},
  imag={img/refs/가르치다},
  desc={/가리키다/와의 어휘 병합, 의미 차이},
  qstn={최근에 새로운 사실을 배운 적 또는 알려주신 적이 있으신가요? 선생님이 학생에게 이런저런 것을 알려주는 것을 무엇을 한다고 하나요?},
  advq={정말 많은 도움이 되었겠어요! 그걸 다른 분에게도 말씀해주신 적이 있을까요?}
}

\Entry{
  word={가리키다},
  pred={}, % TODO: 예상형태 기재
  feat={ms},
  imag={img/refs/가리키다},
  desc={/가르치다/와의 어휘 병합, 의미 차이},
  qstn={윗분을 손가락으로 이렇게(허공을 가리키며) 하면 버릇없다고 하잖아요, 어떻게 하면 버릇이 없다고 하는 건가요?},
  advq={그럼 윗분을 예의 바르게 콕 집으려면 어떻게 해야 되나요?}
}

% new lexicon
\Entry{
  word={묻다[問]},
  pred={무꼬, 물꼬, 물코},
  feat={},
  desc={`-고' 결합 시 자음의 발음, /묻다/[埋]와의 성조·음장 차이},
  qstn={%
    음식을 사러 시장에 가셨어요. 그런데 앞에 있는 음식이 마음에 들어서 가격을 알고 싶어요. 그러면 보통 어떻게 하시나요? \\
    다른 곳으로 여행을 나갔는데, 낯선 곳이라 길을 잃었어요. 그런데 마침 옆에 그 동네 사람이 보이네요. 그러면 어떻게 하시겠어요?
  },
  advq={}
}

% H. 친족어
%!TEX root = main.tex
\section{친족어}
\subsection{일러두기}
일러둘 내용을 적습니다.

\subsection{자유발화 질문}
\begin{itemize}[noitemsep]
  \item 가족분들 소개해주실 수 있나요? (가족사항 및 선대 거주지 조사와 연계)
  \item 친척분들도 벌교에 사세요?
  \item 결혼하시기 전에는 어떤 분들과 함께 사셨어요?
  \item 친척분들은 어떤 분들이세요?
  \item 기억에 남는 명절 이야기 해주실 수 있으세요?
\end{itemize}

\subsection{목표 어휘}
% \Entry 부분을 수정하시면 됩니다.
% word: 표준어형
% pred: 예상형태
% feat: 유의점
%       feat에 들어가는 인자는 쉼표(,)로 구분되며,
%       ms, sp, dp 세 종류가 있습니다.
%       feat에 들어간 인자가 볼드체로 표시됩니다.
% desc: 설명
% qstn: 유도질문
% advq: 심화질문
%
% 그리고 줄바꿈이 필요할 땐 엔터가 아니라 '\\'를 사용해주세요(중요!!).

\Entry{
  word={모친},
  pred={어무니, 엄마},
  feat={},
  desc={},
  qstn={},
  advq={}
}

\Entry{
  word={모친의 모친},
  pred={(외)할무니},
  feat={},
  desc={``(외)함씨'' 계열의 어형 등장 가능성},
  qstn={},
  advq={(외)함씨/함쎄라는 말은 안 쓰시나요?}
}

\newpage
\Entry{
  word={모친의 부친},
  pred={(외)한아부지},
  feat={},
  desc={},
  qstn={},
  advq={}
}

\Entry{
  word={모친의 손위 여성형제},
  pred={이모},
  feat={},
  desc={},
  qstn={},
  advq={}
}

\Entry{
  word={모친의 손위 여성형제의 남편},
  pred={-},
  feat={},
  desc={},
  qstn={},
  advq={}
}

\Entry{
  word={모친의 손위 남성형제},
  pred={외삼춘},
  feat={},
  desc={},
  qstn={},
  advq={}
}

\newpage
\Entry{
  word={모친의 손위 남성형제의 아내},
  pred={-},
  feat={},
  desc={},
  qstn={},
  advq={}
}

\Entry{
  word={모친의 손아래 여성형제},
  pred={이모},
  feat={},
  desc={},
  qstn={},
  advq={}
}

\Entry{
  word={모친의 손아래 여성형제의 남편},
  pred={-},
  feat={},
  desc={},
  qstn={},
  advq={}
}

\Entry{
  word={모친의 손아래 남성형제},
  pred={외삼춘},
  feat={},
  desc={},
  qstn={},
  advq={}
}

\newpage
\Entry{
  word={모친의 손아래 남성형제의 아내},
  pred={-},
  feat={},
  desc={},
  qstn={},
  advq={}
}

\Entry{
  word={부친},
  pred={아부지, 아부니},
  feat={},
  desc={},
  qstn={},
  advq={}
}

\Entry{
  word={부친의 모친},
  pred={할무니},
  feat={},
  desc={``함씨'' 계열의 어형 등장 가능성},
  qstn={},
  advq={함씨/함쎄라는 말은 안 쓰시나요?}
}

\Entry{
  word={부친의 부친},
  pred={한아부지},
  feat={},
  desc={},
  qstn={},
  advq={%
    한아부지의 여자 형제는 뭐라고 부르나요? (데:고모[보성]) \\
    한아부지의 남동생은 뭐라고 부르나요? (종ː지할아버지[광산])
  }
}

\newpage
\Entry{
  word={부친의 손위 여성형제},
  pred={고모},
  feat={},
  desc={},
  qstn={},
  advq={}
}

\Entry{
  word={부친의 손위 여성형제의 남편},
  pred={-},
  feat={},
  desc={``작숙[담양/진도/완도]'' 등장 가능성},
  qstn={},
  advq={}
}

\Entry{
  word={부친의 손위 남성형제},
  pred={큰아부지, 삼춘},
  feat={},
  desc={},
  qstn={},
  advq={}
}

\Entry{
  word={부친의 손위 남성형제의 아내},
  pred={큰어무니, 큰어메},
  feat={},
  desc={},
  qstn={},
  advq={}
}

\newpage
\Entry{
  word={부친의 손아래 여성형제},
  pred={고모},
  feat={},
  desc={},
  qstn={},
  advq={}
}

\Entry{
  word={부친의 손아래 여성형제의 남편},
  pred={-},
  feat={},
  desc={},
  qstn={},
  advq={}
}

\Entry{
  word={부친의 손아래 남성형제},
  pred={작은아부지},
  feat={},
  desc={},
  qstn={},
  advq={}
}

\Entry{
  word={부친의 손아래 남성형제의 아내},
  pred={작은어무니},
  feat={},
  desc={``세메[영광]'' 계열의 어형 등장 가능성},
  qstn={},
  advq={}
}

\newpage
\Entry{
  word={손위 여성형제},
  pred={누ː(님), 성},
  feat={},
  desc={},
  qstn={},
  advq={}
}

\Entry{
  word={손위 여성형제의 남편},
  pred={성부, 메형, 메양, 맹ː, 자형},
  feat={},
  desc={},
  qstn={},
  advq={}
}

\Entry{
  word={손위 남성형제},
  pred={오레비, 오라부니, 오빠, 성(님), 헹(님)},
  feat={},
  desc={},
  qstn={},
  advq={}
}

\Entry{
  word={손위 남성형제의 아내},
  pred={성수, 오라부덕, 올케},
  feat={},
  desc={},
  qstn={},
  advq={}
}

\newpage
\Entry{
  word={손아래 여성형제},
  pred={동상, 동셍},
  feat={},
  desc={},
  qstn={},
  advq={}
}

\Entry{
  word={손아래 여성형제의 남편},
  pred={메제},
  feat={},
  desc={},
  qstn={},
  advq={}
}

\Entry{
  word={손아래 남성형제},
  pred={동상, 동셍},
  feat={},
  desc={},
  qstn={},
  advq={}
}

\Entry{
  word={손아래 남성형제의 아내},
  pred={동상으덕},
  feat={},
  desc={},
  qstn={},
  advq={}
}

\newpage
\Entry{
  word={아내},
  pred={-},
  feat={},
  desc={`젊은 아내' 각시, \\ `처' 큰마느레, `첩' 작은각시},
  qstn={},
  advq={}
}

\Entry{
  word={아내의 모친},
  pred={-},
  feat={},
  desc={},
  qstn={},
  advq={}
}

\Entry{
  word={아내의 부친},
  pred={젱ː인},
  feat={},
  desc={},
  qstn={},
  advq={}
}

\newpage
\Entry{
  word={아내의 손위 여성형제},
  pred={처헹},
  feat={},
  desc={},
  qstn={},
  advq={}
}

\Entry{
  word={아내의 손위 여성형제의 남편},
  pred={동세},
  feat={},
  desc={},
  qstn={},
  advq={}
}

\Entry{
  word={아내의 손위 남성형제},
  pred={-},
  feat={},
  desc={},
  qstn={},
  advq={}
}

\Entry{
  word={아내의 손위 남성형제의 아내},
  pred={-},
  feat={},
  desc={``처남우덕[담양/나주/목포]'' 등장 가능성},
  qstn={},
  advq={}
}

\newpage
\Entry{
  word={아내의 손아래 여성형제},
  pred={-},
  feat={},
  desc={},
  qstn={},
  advq={}
}

\Entry{
  word={아내의 손아래 여성형제의 남편},
  pred={-},
  feat={},
  desc={},
  qstn={},
  advq={}
}

\Entry{
  word={아내의 손아래 남성형제},
  pred={-},
  feat={},
  desc={},
  qstn={},
  advq={}
}

\Entry{
  word={아내의 손아래 남성형제의 아내},
  pred={-},
  feat={},
  desc={``처남우덕[담양/나주/목포]'' 등장 가능성},
  qstn={},
  advq={}
}

\newpage
\Entry{
  word={남편},
  pred={남펜, 넴펜},
  feat={},
  desc={},
  qstn={},
  advq={}
}

\Entry{
  word={남편의 모친},
  pred={시어메, 씨어무니},
  feat={},
  desc={},
  qstn={},
  advq={}
}

\Entry{
  word={남편의 부친},
  pred={시아부이, 씨아부니},
  feat={},
  desc={},
  qstn={},
  advq={}
}

\Entry{
  word={남편의 손위 여성형제},
  pred={씨누},
  feat={},
  desc={},
  qstn={},
  advq={}
}

\newpage
\Entry{
  word={남편의 손위 여성형제의 남편},
  pred={-},
  feat={},
  desc={},
  qstn={},
  advq={}
}

\Entry{
  word={남편의 손위 남성형제},
  pred={씨숙, 큰씨아제},
  feat={},
  desc={``큰서방님'' 등장 가능성},
  qstn={},
  advq={}
}

\Entry{
  word={남편의 손위 남성형제의 아내},
  pred={-},
  feat={},
  desc={},
  qstn={},
  advq={}
}

\Entry{
  word={남편의 손아래 여성형제},
  pred={씨누},
  feat={},
  desc={``동숭에기, 에기씨'' 등장 가능성},
  qstn={},
  advq={}
}

\newpage
\Entry{
  word={남편의 손아래 여성형제의 남편},
  pred={-},
  feat={},
  desc={},
  qstn={},
  advq={}
}

\Entry{
  word={남편의 손아래 남성형제},
  pred={시동생},
  feat={},
  desc={},
  qstn={},
  advq={}
}

\Entry{
  word={남편의 손아래 남성형제의 아내},
  pred={-},
  feat={},
  desc={},
  qstn={},
  advq={}
}

\Entry{
  word={남편의 남성형제(시아주버니)},
  pred={씨아제},
  feat={},
  desc={},
  qstn={},
  advq={}
}

\newpage
\Entry{
  word={딸의 남편},
  pred={사우},
  feat={},
  desc={},
  qstn={},
  advq={}
}

\Entry{
  word={아들의 아내},
  pred={메느리},
  feat={},
  desc={},
  qstn={},
  advq={}
}

\Entry{
  word={아재},
  pred={아제},
  feat={ms},
  desc={아버지뻘의 친척. 어형의 존재 여부 및 \\ 정확한 의미 범위 확인},
  qstn={},
  advq={}
}

\Entry{
  word={자식의 자식},
  pred={손자, 손지},
  feat={},
  desc={},
  qstn={},
  advq={}
}

% II. 문법편
\chapter{문법편}

\subsection{일러두기}
\begin{itemize}[noitemsep]
  \item 이미 관찰된 문법 요소는 아래 체크리스트를 확인한 뒤 넘어가십시오.
  \item 예문과 조금 다르더라도 특정 문법 요소가 드러난다면 넘어가도 좋습니다.
  \item 표준형 문법 요소를 억지로 이끌어낼 필요는 없습니다. 다만 표준형 대신 쓰인 문법 요소를 잘 기록하고 넘어가도록 합니다. 이때 방언의 음운적 특징을 잘 반영하여(예: 뭐라카다, 해뿌리고) 그대로 기록합니다.
  \item 특정 문법 요소가 드러나지 않는다면 역질문을 통해 사용 여부를 확인합니다. 만약 사용한다는 답변을 들은 경우, 해당 문법 요소는 어떤 상황에서 어떤 의미로 사용하는지 함께 확인합니다.
\end{itemize}

\subsection{작성 요령}
\begin{itemize}[noitemsep]
  \item 문법 조사표 담당자는 자료제공인 조사와 어휘 조사 과정에서 발화되는 조사에 표시합니다.
  \item 반드시 점화효과를 고려하여 체크합니다. 자료제공인의 답변 직전의 조사자의 질문에 문법 표지가 먼저 등장한 경우, 자료제공인이 해당 문법 표지를 사용하더라도 체크하지 않습니다. <어휘편>에서 조사자의 자유발화 질문에 포함된 문법 표지에 의한 점화 효과는 고려하지 않되, 자유발화 질문이 점화 효과를 일으킨 것이 분명하다고 판단되면, 체크하지 않습니다. 
  \item 문법 표지가 자유발화에서 확인되면 `방언형' 난에 어느 조사 파트에서 어떤 형식으로 확인됐는지 적습니다.
    \begin{itemize}[noitemsep]
      \item 예 1: `벼'를 유도하던 중 `벼를'을 검출한 경우 → `벼' / 벼를
      \item 예 2: 자료제공인 조사 중 `해가지구'를 검출한 경우 → 자료제공인 조사 / 해가지구
    \end{itemize}
  \item 조사 현장에서 최대한 문법 표지를 확인한 뒤, 하루의 조사를 마무리할 때 녹음을 다시 들으며 놓친 문법 표지를 다시 확인합니다.
  \item 조사 과정에서 확인되지 않은 문법 사항은 아래 질문을 참고하여 보충합니다.
\end{itemize}

%!TEX root = main.tex

\section{조사}
※ 질문에서 조사 노출을 최소화하기 바랍니다. \\
※ 조사 과정에서 자연스럽게 출현한 형태만을 기록하고, 별도의 질문을 시행하지 않습니다.

\subsection{격조사}
\Gram{
  word={(격조사) -이/가},
  desc={선행 명사의 모음이 상승하는지도 확인(예: /속+이/ → [쇠기])}
}

\Gram{
  word={(격조사) -을/를}
}

\Gram{
  word={(격조사) -에게/게}
}

\Gram{
  word={(격조사) -에}
}

\Gram{
  word={(격조사) -에서}
}

\Gram{
  word={(격조사) -(으)로}
}

\Gram{
  word={(격조사) -와/과}
}

\Gram{
  word={(격조사) -보다}
}

\Gram{
  word={(격조사) -처럼}
}

\Gram{
  word={(격조사) -만큼}
}

\Gram{
  word={(격조사) -더러, -보고}
}

\Gram{
  word={(격조사) -(이)랑}
}

\Gram{
  word={(격조사) -커녕}
}

\Gram{
  word={(격조사) -아/야}
}

\newpage
\subsection{보조사}
\Gram{ word={(보조사) -은/는} }
\Gram{ word={(보조사) -만} }
\Gram{ word={(보조사) -도} }
\Gram{ word={(보조사) -마다} }
\Gram{ word={(보조사) -부터} }
\Gram{ word={(보조사) -까지} }
\Gram{ word={(보조사) -조차} }
\Gram{ word={(보조사) -(이)야} }
\Gram{ word={(보조사) -(이)라도} }
\Gram{ word={(보조사) -밖에} }
\Gram{ word={(보조사) -가지고} }

\subsection{문장 뒤 조사}
\Gram{ word={(간접인용) -고} }
\Gram{ word={(높임) -요} }


\section{종결어미}
※ 표준어와 다른 높임 체계(허씨요체, 허소체)와 표준어에 없는 어미(`-게라/라우', `-게', `-이-' 등)의 출현이 예상됩니다. \\
※ 표준어적 방책으로 [+격식성] 자질을 드러낸다는 연구가 있으니, 질문에 참고하기 바랍니다. \\
※ 친족어 조사와 연계하여 질문할 수 있습니다.

\newpage
\subsection{어린 아이에게}
\Gram{
  word={(해라, 의문) -니},
  qstn={어린 아이랑 외출을 하기로 했는데 밖에 날씨가 어떤지 아이에게 묻고 싶을 때 뭐라고 말씀하세요?}
}

\Gram{
  word={(해라, 청유) -자},
  qstn={아이가 밖을 보더니 비가 와서 나가기 싫다고 합니다. 그래도 오늘 산책은 해야 하는데 아이에게 같이 나갈 것을 권유할 때 뭐라고 말씀하세요?}
}

\Gram{
  word={(해라, 명령) -아라/어라},
  qstn={아이가 알겠다고 말합니다. 이제 나갈 준비를 해야 하는데, 아이에게 우비를 입고 장화를 신을 것을 말할 때 어떻게 말씀하세요?}
}

\Gram{
  word={(해라, 서술1) -ㄴ다},
  qstn={장모가 함께 식사를 하는 사위에게 오늘 술을 마실 건지 물어볼 때 뭐라고 말하나요?}
}

\subsection{장모가 사위에게}
\Gram{
  word={(하게, 의문) -나},
  qstn={장모가 함께 식사를 하는 사위에게 오늘 술을 마실 건지 물어볼 때 뭐라고 말하나요?}
}

\Gram{
  word={(하게, 청유) -세},
  qstn={사위가 술을 받아들었습니다. 술을 마시고 밥도 먹을 것을 말할 때 어떻게 말씀하시나요?}
}

\Gram{
  word={(하게, 명령) -게},
  qstn={사위가 술을 받아들었습니다. 술을 마시고 밥도 먹을 것을 말할 때 어떻게 말씀하시나요?}
}

\Gram{
  word={(하게, 서술1) -네},
  qstn={오랜만에 사위와 함께 식사하니 기분이 좋은 것을 말할 때 어떻게 말씀하시나요?}
}

\newpage
\subsection{사위가 장인·장모에게}
\Gram{
  word={(하십시오, 의문) -습니까, -나요},
  qstn={사위가 장모님 칠순여행을 어디로 가고 싶으신지 여쭤볼 때는 뭐라고 말하나요?}
}

\Gram{
  word={(하십시오, 청유) -십니다, -세요},
  qstn={장모님께서 여행 같은 건 안 가도 된다고 말씀하십니다. 그래도 칠순이시니 제주도에 함께 가자고 권유할 때는 뭐라고 말하나요?}
}

\Gram{
  word={(하십시오, 명령) -(으)십시오, -세요},
  qstn={장모님께서 고민해보겠다며 자리를 뜨려고 하십니다. 다시 앉으실 것을 어떻게 말하나요?}
}

\Gram{
  word={(하십시오, 서술1) -습니다, -네요},
  qstn={사위는 장모님의 기분을 전환하려 날씨 이야기를 합니다. 오늘 날씨가 좋다는 말을 어떻게 하는 게 좋을까요?}
}

\subsection{손아래 동서가 손위 동서에게}
\Gram{
  word={(하오, 의문) -오},
  qstn={손아래 동서가 손위 동서에게 어머님 칠순여행에 함께 갈 것인지 물을 때 어떻게 말하나요?}
}

\Gram{
  word={(하오, 청유) -(으)오},
  qstn={손위 동서는 아직 고민 중이라고 합니다. 함께 갈 것을 권유할 때는 어떻게 말하나요?}
}

\Gram{
  word={(하오, 명령) -(으)오},
  qstn={손위 동서가 고민해보겠다며 자리를 뜨려고 합니다. 다시 앉을 것을 어떻게 말하나요?}
}

\Gram{
  word={(하오, 서술1) -오},
  qstn={손아래 동서는 손위 동서의 기분을 전환하려고 날씨 이야기를 꺼냅니다. 오늘 날씨가 좋은 것을 어떻게 말하는 게 좋을까요?}
}

\newpage
\subsection{할머니가 또래 할머니에게}
\Gram{
  word={(하셔, 의문) -ㄴ가},
  qstn={할머니가 또래 할머니에게 요즘 잘 지내는지 물어볼 때 뭐라고 말하나요?}
}

\Gram{
  word={(하셔, 청유) -세},
  qstn={상대 할머니께서 요즘 자식 문제로 좀 힘들다고 말씀하셨습니다. 그럼 함께 차라도 한 잔 할 것을 권유할 때 뭐라고 말하나요?}
}

\Gram{
  word={(하셔, 명령) -셔},
  qstn={두 분이 함께 찻집에 갔습니다. 상대 할머니에게 자리에 앉을 것을 어떻게 말하나요?}
}

\Gram{
  word={(하셔, 서술1) -네},
  qstn={주문한 차가 나왔습니다. 상대 할머니에게 차가 따뜻한 것을 말할 때 어떻게 말하나요?}
}

\subsection{평서형 종결어미 (2)}
\Gram{
  word={(해라, 서술2) -(이)다},
  qstn={(어린 아이에게) 내일이 장날이라고 알려줄 때 뭐라고 말하나요?}
}

\Gram{
  word={(하게, 서술2) -(이)네, -일세},
  qstn={(장모가 사위에게) 내일이 장날이라고 알려줄 때 뭐라고 말하나요?}
}

\Gram{
  word={(하십시오, 서술2) -입니다, -이지요},
  qstn={(사위가 장인·장모에게) 내일이 장날이라고 알려줄 때 뭐라고 말하나요?}
}

\Gram{
  word={(하오, 서술2) -이오},
  qstn={(손아래 동서가 손위 동서에게) 내일이 장날이라고 알려줄 때 뭐라고 말하나요?}
}

\Gram{
  word={(하셔, 서술2) -이오},
  qstn={(할머니가 또래 할머니에게) 내일이 장날이라고 알려줄 때 뭐라고 말하나요?}
}

\newpage
\subsection{의문형 종결어미 (2)}
\Gram{
  word={(해라, 의문2) -(이)니},
  qstn={(어린 아이에게) 내일이 장날이냐고 물어볼 때 뭐라고 말하나요?}
}

\Gram{
  word={(하게, 의문2) -인가},
  qstn={(장모가 사위에게) 내일이 장날이냐고 물어볼 때 뭐라고 말하나요?}
}

\Gram{
  word={(하십시오, 의문2) -입니까},
  qstn={(사위가 장인·장모에게) 내일이 장날이냐고 물어볼 때 뭐라고 말하나요?}
}

\Gram{
  word={(하오, 의문2) -이오},
  qstn={(손아래 동서가 손위 동서에게) 내일이 장날이냐고 물어볼 때 뭐라고 말하나요?}
}

\Gram{
  word={(하셔, 의문2) -인가},
  qstn={(할머니가 또래 할머니에게) 내일이 장날이냐고 물어볼 때 뭐라고 말하나요?}
}

\subsection{의문형 종결어미 (3)}
\Gram{
  word={(해라, 의문3) -아},
  qstn={(어린 아이에게) 배가 고픈지 물어볼 때 뭐라고 말하나요?}
}

\Gram{
  word={(하게, 의문3) -가},
  qstn={(장모가 사위에게) 배가 고픈지 물어볼 때 뭐라고 말하나요?}
}

\Gram{
  word={(하십시오, 의문3) -십니까},
  qstn={(사위가 장인·장모에게) 배가 고픈지 물어볼 때 뭐라고 말하나요?}
}

\Gram{
  word={(하오, 의문3) -오},
  qstn={(손아래 동서가 손위 동서에게) 배가 고픈지 물어볼 때 뭐라고 말하나요?}
}

\Gram{
  word={(하셔, 의문3) -오},
  qstn={(할머니가 또래 할머니에게) 배가 고픈지 물어볼 때 뭐라고 말하나요?}
}

\newpage
\subsection{의문형 종결어미 (4)}
\Gram{
  word={(해라, 의문4) -지},
  qstn={(어린 아이에게) 밥을 먹을 것을 재차 확인하며 묻고 싶을 때 뭐라고 말하나요?}
}

\Gram{
  word={(하게, 의문4) -가},
  qstn={(장모가 사위에게) 밥을 먹을 것을 재차 확인하며 묻고 싶을 때 뭐라고 말하나요?}
}

\Gram{
  word={(하십시오, 의문4) -겠습니까},
  qstn={(사위가 장인·장모에게) 밥을 먹을 것을 재차 확인하며 묻고 싶을 때 뭐라고 말하나요?}
}

\Gram{
  word={(하오, 의문4) -오},
  qstn={(손아래 동서가 손위 동서에게) 밥을 먹을 것을 재차 확인하며 묻고 싶을 때 뭐라고 말하나요?}
}

\Gram{
  word={(하셔, 의문4) -오},
  qstn={(할머니가 또래 할머니에게) 밥을 먹을 것을 재차 확인하며 묻고 싶을 때 뭐라고 말하나요?}
}


\section{연결어미}
※ 조사 과정에서 자연스럽게 출현한 형태만을 기록하고, 별도의 질문을 시행하지 않습니다. \\

\Gram{ word={(연결어미) -고, -고서} }
\Gram{ word={(연결어미) -(으)면서} }
\Gram{ word={(연결어미) -아/어, -아서/어서} }
\Gram{ word={(연결어미) -(으)니, -(으)니까} }
\Gram{ word={(연결어미) -관데} }
\Gram{ word={(연결어미) -다가} }
\Gram{ word={(연결어미) -거든} }
\Gram{ word={(연결어미) -거든} }
\Gram{ word={(연결어미) -더라도} }
\Gram{ word={(연결어미) -(으)려고} }
\Gram{ word={(연결어미) -도록} }
\Gram{ word={(연결어미) -듯이} }
\Gram{ word={(연결어미) -지} }


\newpage
\section{시제}
\Gram{
  word={(시제) -는/ㄴ-},
  qstn={집 안에 있다가 창문을 열었는데 비가 오고 바람이 많이 불어요. 그때 밖의 날씨를 집 안에 있는 사람에게 말하는 것처럼 설명해보시겠어요? → `비가 온다', `바람이 분다' 등}
}

\Gram{
  word={(시제) -고 있-},
  qstn={%
  새들이 하늘에서 뭐를 하나요? → `날고 있다'}
}

\Gram{
  word={(시제) -았/었-},
  qstn={인사치레로 식사했는지를 어떻게 물어보나요? → `먹었니' 등}
}

\Gram{
  word={(시제) -았었/었었-},
  qstn={다른 지역으로 여행을 간 적이 있으세요? → `갔었다'}
}

\Gram{
  word={(시제) -더-},
  qstn={키가 큰 손자를 보고 와서, 손자가 키가 많이 컸다는 얘기를 다른 할머니한테 전할 거예요. 뭐라고 말하시겠어요? → `컸더라'}
}


\section{관형형}
※ 조사 과정에서 자연스럽게 출현한 형태만을 기록하고, 별도의 질문을 시행하지 않습니다. \\

\Gram{ word={(동사 관형형) 입는} }
\Gram{ word={(동사 관형형) 입었던} }
\Gram{ word={(동사 관형형) 입은} }
\Gram{ word={(동사 관형형) 입을} }
\Gram{ word={(형용사 관형형) 큰} }
\Gram{ word={(형용사 관형형) 크던} }
\Gram{ word={(형용사 관형형) 컸던} }


\newpage
\section{부정}
\Gram{
  word={안 먹었어, 먹지 않았어 (1)},
  qstn={%
    이웃이 밥 먹었느냐고 물었을 때 그렇지 않다고 대답하려면 어떻게 말하나요? \\
    (아니, 아직 안 먹었어. / 아니, 아직 먹지 않았어.)
  }
}

\Gram{
  word={안 먹었어, 먹지 않았어 (2)},
  qstn={%
    밥 먹을 시간이 지났는데도 아직 밥을 안 먹은 이유를 이웃이 나에게 물어볼 때는 어떻게 말하나요? \\
    (왜 아직도 밥을 안 먹었어? / 왜 아직도 안 밥 먹었어? / 왜 아직도 밥을 먹지 않았어?)
  }
}

\Gram{
  word={안 좋아, 좋지 않아},
  qstn={%
    이웃이 오늘 날씨가 좋으냐고 물었을 때 그렇지 않다고 대답하려면 어떻게 말하나요? \\
    (날씨가 안 좋다. / 날씨가 좋지 않다.)
  }
}

\Gram{
  word={안 깨끗해, 깨끗하지 않아},
  qstn={%
    이웃이 밭일을 하다 말고 왔는지 옷에 흙먼지가 묻어있네요. 이때 이웃이 자신의 옷이 깨끗하냐고 묻습니다. 어떻게 답하나요? \\
    (아니, 안 깨끗해. / 아니, 깨끗하지 않아. / 아니, 깨끗 안 해.)
  }
}

\Gram{
  word={안 갔어, 가지 않았어},
  qstn={%
    이웃이 아들이 장가 갔는지를 물었을 때 그렇지 않다고 대답하려면 어떻게 말하나요? \\
    (아직 장가 안 갔다. / 아직 안 장가갔다. / 아직 장가가지 않았다.)
  }
}

\Gram{
  word={안 만나 보았다, 만나 보지 않았다},
  qstn={%
    이웃이 안타까워하며 전에 한번 만나 보라고 한 사람을 만나 봤느냐고 물었을 때, 그렇지 않다고 대답하려면 어떻게 말하나요? \\
    (아직 안 만나 보았다. / 아직 만나 보지 않았다. / 아직 만나 안 보았다.)
  }
}

\Gram{
  word={먹도 않고},
  qstn={%
    이웃이 집에 있는 아기가 밥을 먹었냐고 물었을 때, 그렇지 않고 종일 잠만 자는 상황이라고 대답하려면 어떻게 말하나요? \\
    (먹지도 않고 잠만 잔다.)
  }
}

\Gram{
  word={못 마신다, 마시지 못한다},
  qstn={%
  술을 마실 줄 아느냐고 물었을 때 그렇지 않다고 대답하려면 어떻게 말하나요? \\
  (술을 못 마신다. / 술을 마시지 못한다.)
  }
}


\newpage
\section{사동표현과 피동표현}
※ 관련 질문은 <어휘·음운편>의 심화질문란을 참조하십시오. \\

\subsection{사동표현}
\Gram{ word={(사동표현) 살리다} }
\Gram{ word={(사동표현) 늘리다} }
\Gram{ word={(사동표현) 말리다[乾]} }
\Gram{ word={(사동표현) 말리다[挽]} }
\Gram{ word={(사동표현) 얼리다} }
\Gram{ word={(사동표현) 녹이다} }
\Gram{ word={(사동표현) 신기다} }
\Gram{ word={(사동표현) 입히다} }
\Gram{ word={(사동표현) 앉히다} }

\Gram{
  word={(사동표현) 죽이다},
  desc={V1 /ㅜ/ → /ㅟ/}
}

\subsection{피동표현}
\Gram{ word={(피동표현) 잡히다} }
\Gram{ word={(피동표현) 깎이다} }
\Gram{ word={(피동표현) 끼이다} }
\Gram{ word={(피동표현) 떼이다} }
\Gram{ word={(피동표현) 끊기다} }
\Gram{ word={(피동표현) 채이다} }
\Gram{ word={(피동표현) 실리다} }
\Gram{ word={(피동표현) 바뀌다} }

\Gram{
  word={(피동표현) 업히다},
  desc={V1 /ㅓ/ → /ㅔ/}
}


\newpage
\section{보조용언}
※ 조사 과정에서 자연스럽게 출현한 형태만을 기록하고, 별도의 질문을 시행하지 않습니다. \\

\Gram{ word={(보조용언) 싶다} }
\Gram{ word={(보조용언) 보다} }
\Gram{ word={(보조용언) 버리다} }
\Gram{ word={(보조용언) 대다} }
\Gram{ word={(보조용언) -나/는가보다} }

\vspace*{14cm}


\chapter{어휘 의미지도}
\section{농사와 식물}
\SemMap{img/maps/농사와 식물}

\section{가정과 생활}
\SemMap{img/maps/가정과 생활}

\section{자연과 일상}
\SemMap{img/maps/자연과 일상}

\section{바다}
\section{사회적 지식}
\SemMap{img/maps/바다-사회적 지식}

\section{신체}
\section{기타 동사}
\SemMap{img/maps/신체-기타 동사}

\section{친족어}
\SemMap{img/maps/친족어1}
\SemMap{img/maps/친족어2}

\chapter{사진 자료}
\ImagRefs


\backmatter
\makebackcover

\end{document}