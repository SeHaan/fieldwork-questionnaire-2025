\documentclass{snu-fl-questionnaire}

% Hangul font setup
\setmainhangulfont{KoPubWorldBatang_Pro}[
  Scale = MatchUppercase,
  UprightFont={* Light},
  BoldFont={* Bold},
  AutoFakeSlant = 0.15
]
\setsanshangulfont{KoPubWorldDotum_Pro}[
  Scale = MatchUppercase,
  BoldFont={* Bold},
]

% If there are problems in Hangul Font setup code,
% run the following line to update your cache:
%   $ fc-cache -fv
%
% Or, you can load fonts by their filenames, e.g.,
%
%\setmainhangulfont{KoPubWorldBatang_Pro}[
%  Scale = MatchUppercase,
%  UprightFont = {KoPubWorld Batang_Pro Light.otf}, % Exact filename
%  BoldFont = {KoPubWorld Batang_Pro Bold.otf},     % Exact filename
%  AutoFakeSlant = 0.15
%]
%\setsanshangulfont{KoPubWorldDotum_Pro}[
%  Scale = MatchUppercase,
%  UprightFont = {KoPubWorld Dotum_Pro Medium.otf}, % Exact filename (or Light)
%  BoldFont = {KoPubWorld Dotum_Pro Bold.otf}       % Exact filename
%]

% Metadata
\title{2025 한국 언어조사 질문지}
\author{서울대학교 인문대학 언어학과}
\date{2025년 10월 31일 \textasciitilde{} 11월 1일}
\printdate{2025년 10월 31일}
\issuedate{2025년 10월 31일}
\tel{(02) 880-6163, 6164}


\begin{document}

\frontmatter
\maketitle
\tableofcontents

\chapter{2025학년도 언어조사 개요}
2025학년도 언어조사의 일정, 장소, 교통편 등을 적습니다.

\chapter{2025학년도 언어조사 일정}
2025학년도 언어조사의 구체적인 일정을 적습니다.

\chapter{자료제공인 조사표 (1)}
\Consultant

\chapter{자료제공인 조사표 (2)}
\Consultant

\chapter{자료제공인 조사표 (3)}
\Consultant

\chapter{자료제공인 조사표 (4)}
\Consultant

\chapter{일러두기}
일러둘 내용을 적습니다.


\mainmatter
\chapter{어휘·음운편}
\subsection{일러두기}
일러둘 내용을 적습니다.

\section{농사와 식물}

\subsection{자유발화 질문}
농사를 지으시나요? / 어떤 농사를 하시나요? / 어떤 작물을 길러보셨나요? / (작물 이름)을 키우려면 어떤 과정이 필요한가요? / 좋아하시는 채소가 있으세요? / 어떤 과일을 자주 드시나요?

\subsection{목표 어휘}

\Entry{
  word = {벼},
  pred = {베},
  feat = {ms},
  desc = {%
    벼이삭, 벼의 열매, 식물 벼의 명칭, \\
    단어형 차이 (나락 등의 어휘 확인)
  },
  qstn = {%
    (정보제공자 본인 or 동네분들)은 어떤 농사를 지으시나요? \\
    논에서는 주로 무엇을 키우시나요? (익을수록 고개를 숙이는 작물)
  },
  advq = {%
    벼를 키울 때 주의해야 할 것은 무엇이 있을까요? \\
    벼가 다 자라면 어떻게 수확하고 처리하세요?
  }
}

\Entry{
  word={쌀},
  pred={쌀},
  feat={ms,sp,dp},
  desc={},
  qstn={벼(A01)의 껍질을 벗긴 것을 무엇이라고 하시나요? (밥을 해 먹을 때는 그 낟알)},
  advq={쌀을 어떻게 보관하나요?}
}

\newpage
\Entry{
  word={볍씨},
  pred={씬나락/종잣벼},
  feat={ms,sp,dp},
  desc={단어형 차이(씬나락/종잣벼)},
  qstn={벼(A01)의 씨를 무엇이라고 하나요?},
  advq={볍씨라는 말은 안 쓰나요?}
}

\Entry{
  word={못자리},
  pred={모짜리},
  feat={ms,sp,dp},
  desc={},
  qstn={모내기하기 전에 볍씨(A03)를 어디에 심나요?},
  advq={못자리와 논은 보통 근처에 있나요? 못자리에서 어떻게 논으로 모를 옮겨 심나요?}
}

\Entry{
  word={보리},
  pred={보리},
  feat={ms,sp,dp},
  desc={},
  qstn={벼를 수확하고 난 뒤에 겨울에 길러서 봄여름에 수확하는 곡식이 무엇인가요? (옛날에 쌀이 다 떨어지면 대신 먹는 곡식)},
  advq={보릿고개라는 말이 정확히 무슨 뜻인지 설명해주실 수 있나요?}
}

\Entry{
  word={깜부기},
  pred={까막보리/깜뎅이},
  feat={ms,sp,dp},
  desc={},
  qstn={보리(A05) 이삭에 병이 들어 까맣게 되는 것을 무엇이라고 하시나요?},
  advq={깜부기병 말고 곡식에 드는 병이 드나요?}
}

\newpage
\Entry{
  word={조},
  pred={-},
  feat={ms,sp,dp},
  desc={},
  qstn={(A08 수수, A09 기장과 함께) 잡곡밥에는 무엇을 넣나요? 벼와 보리 말고 농사 짓는 곡식에는 무엇이 있나요? (조: 노랗고 모래알만한 알맹이)},
  advq={조를 좁쌀이라고도 부르나요?}
}

\Entry{
  word={수수},
  pred={스슥/쑤수},
  feat={ms,sp,dp},
  desc={기타이형태 슈슈},
  qstn={줄기가 얼룩덜룩하게 붉으며, 알갱이는 불그스름하고 동글동글한데 경단도 만들어 먹는 곡식이 무엇인가요?},
  advq={수수와 기장을 잡곡밥에 넣기도 하나요? 보름날 먹는 오곡밥에는 무엇을 넣나요?}
}

\Entry{
  word={기장},
  pred={-},
  feat={ms,sp,dp},
  desc={},
  qstn={조처럼 노란데 조보다는 약간 굵고 알갱이가 더 큰 곡식이 무엇인가요?},
  advq={수수와 기장을 잡곡밥에 넣기도 하나요? 보름날 먹는 오곡밥에는 무엇을 넣나요?}
}

\newpage
\Entry{
  word={옥수수},
  pred={옥쑤수/옥쑤꽹이},
  feat={ms,sp,dp},
  desc={기타이형태 옥슈슈, 다른단어 강냉이, 열매와 열매 자루의 의미 차이가 있는지},
  qstn={알갱이 크기는 콩알만 한데 통째로 쪄 먹는 곡식을 무엇이라고 하시나요? (찌면 색깔이 노랗고 사료로 쓰기도 함.)},
  advq={서산에서 옥수수나 나나요? 아니면 다른 지역에서 옥수수가 들어오나요?}
}

\Entry{
  word={새끼(줄)},
  pred={새내끼/샌내끼},
  feat={ms,sp,dp},
  desc={164.새끼와의 성조/음장 차이 \\ 1음절 /ㅐ-ㅔ/대립},
  qstn={짚을 꼬아서 만든 끈을 무엇이라고 하시나요?},
  advq={밧줄과 새끼는 같은 뜻인가요?}
}

\Entry{
  word={새참},
  pred={-},
  feat={ms,sp,dp},
  desc={다른 단어 /곁두리/},
  qstn={그렇게 김도 매시고 타작도 하시다 보면 시장하시잖아요. 아침이랑 점심 사이에 시장하실 때 드시는 것이 무엇인가요?},
  advq={새참으로는 주로 어떤 것들을 드시나요?}
}

\newpage
\Entry{
  word={절구},
  pred={절구},
  feat={ms,sp,dp},
  desc={첫 자음 /ㅈ/},
  qstn={곡식을 넣고 찧을 때에는 어디에다 찧으시나요? (떡을 치는 기구)},
  advq={절구는 주로 돌로 만드나요? 절구를 찧을 때 쓰는 막대기를 뭐라고 하나요? (절구공이, 절구땡이로 나타날 수 있음)}
}

\Entry{
  word={겨},
  pred={저},
  feat={ms,sp,dp},
  desc={ㄱ 구개음화 1음절 겨},
  qstn={절구(A13)에 넣고 곡식을 찧고 나면 남는 껍질을 무엇이라고 하시나요?},
  advq={겨와 쭉정이의 차이점이 무엇인가요?}
}

\Entry{
  word={키},
  pred={치},
  feat={ms,sp,dp},
  desc={},
  qstn={곡식에서 쭉정이나 티끌을 골라낼 때 무엇으로 고르시나요? (오줌싸개가 쓰고 다니기도 함.)},
  advq={키질은 어떻게 하나요? 키로 쭉정이를 털어낼 때 키를 어떻게 잡아야 하나요?}
}

\Entry{
  word={김매기},
  pred={짐매다},
  feat={ms,sp,dp},
  desc={C1 + V1 /기/ \\ V1 다음 반치음이 탈락하였음},
  qstn={논밭에 잡초가 많이 나면 어떻게 하시나요? 논에서 잡초를 없애는 것을 무엇이라고 하나요?},
  advq={논에서 하는 것과 밭에서 하는 것 모두 김매기라고 부르나요? 차이가 있나요?}
}

\newpage
\Entry{
  word={호미},
  pred={호미},
  feat={ms,sp,dp},
  desc={기타이형태 호믜},
  qstn={밭에서 김매기(A16)를 하실 때는 무엇을 가지고 하시나요? (감자나 고구마 등을 캘 때 사용하는 도구)},
  advq={호미가 농사에서 이것저것 쓰임새가 많다고 알고 있는데, 호미로 할 수 있는 작업에는 무엇이 있나요?}
}

\Entry{
  word={써레},
  pred={쓰레},
  feat={ms,sp,dp},
  desc={써흐레>써레},
  qstn={갈아 놓은 논의 바닥을 고를 때 어떤 농기구를 쓰시나요?},
  advq={(A21 도리깨와 함께 진행)}
}

\Entry{
  word={쇠스랑},
  pred={소스랑/소시랑},
  feat={ms,sp,dp},
  desc={'쇼시랑, 소시랑(이) 등},
  qstn={땅을 파헤쳐 고르거나 두엄을 칠 때 사용하는 갈퀴 모양의 농기구를 무엇이라고 하시나요?},
  advq={(A21 도리깨와 함께 진행)}
}

\Entry{
  word={가래(농기구)},
  pred={-},
  feat={ms,sp,dp},
  desc={2음절 V2 /ㅐ-ㅔ/ 대립},
  qstn={논바닥에서 흙을 퍼서 던질 때 어떤 농기구를 쓰시나요? 한 사람이 자루를 잡고 두 사람이 줄을 잡아당겨 흙을 퍼서 던지죠.},
  advq={혹시 서산에서는 삽을 가래라고 부르기도 하나요? 아니면 가래를 삽이라고 부르기도 하나요? (A21 도리깨와 함께 진행)}
}

\newpage
\Entry{
  word={도리깨},
  pred={도리끼/도리채/도리캐/도링캐},
  feat={ms,sp,dp},
  desc={},
  qstn={긴 막대기에 싸리나 대나무를 달아 곡식을 두드려 타작하는 농기구를 무엇이라고 하시나요?},
  advq={써레(A18)/쇠스랑(A19)/가래(A20)/도리깨를 요즘 농사지을 때에도 사용하시나요? 옛날 농사에만 사용했나요?}
}

\Entry{
  word={멍석},
  pred={멩석},
  feat={ms,sp,dp},
  desc={둥근 것과 네모난 것의 명칭 차이},
  qstn={곡식을 타작하거나 널어 말릴 때 어디에 깔아 놓으시나요?},
  advq={요즘에도 멍석을 깔고 무언가를 말리기도 하나요? 요즘에는 멍석에다 무엇을 말리나요?}
}

\Entry{
  word={깨},
  pred={꽤},
  feat={ms,sp,dp},
  desc={V1 /ㅐ-ㅔ/ 대립},
  qstn={참기름이나 들기름은 무엇을 짜서 만드시나요?},
  advq={참깨와 들깨는 어떻게 구분하나요?}
}

\Entry{
  word={나물},
  pred={너물},
  feat={ms,sp,dp},
  desc={},
  qstn={산이나 들에 나는 풀 중에서 캐다가 무쳐 먹는 것을 무엇이라고 하시나요? (대보름에 오곡밥과 함께 먹는 것)},
  advq={명절에 나물반찬 많이 하시나요? 어떤 나물들을 하시나요? (B25 부침개나 A38 배, A39 밤과 관련해 명절 음식으로 대화 이어나갈 수 있음)}
}

\newpage
\Entry{
  word={냉이},
  pred={나신갱이/나승갱이/나시/나싱개/나싱갱이},
  feat={ms,sp,dp},
  desc={나ᅀᅵ, 낭히... > 냉이},
  qstn={봄에 뿌리째로 캐서 된장국을 해 먹는 것은 무엇이라고 하시나요? (들이나 밭에서 캐 먹음.)},
  advq={냉이랑 비슷하게 생겼는데 냉이가 아닌 식물들에는 무엇이 있나요? 냉이랑 구분할 수 있는 방법이 있나요?}
}

\Entry{
  word={달래},
  pred={달리},
  feat={ms,sp,dp},
  desc={},
  qstn={봄에 들에서 나는 뿌리는 조그맣고 동글동글하면서 줄기는 실처럼 가늘고 긴 것은 무엇인가요? (파와 같은 냄새가 나고 매운 맛이 있는 나물)},
  advq={(‘양파’라고 잘못 얘기할 경우 ‘양파보다 훨씬 작고 줄기도 실처럼 가늘다’ 하고 이야기한다.)}
}

\Entry{
  word={콩나물},
  pred={콩너물},
  feat={ms,sp,dp},
  desc={},
  qstn={콩에 물을 주어 길러 먹는 것은 무엇이라고 하시나요?},
  advq={콩나물로는 무슨 반찬을 해 드시나요?}
}

\Entry{
  word={상추},
  pred={부루},
  feat={ms,sp,dp},
  desc={},
  qstn={쌈을 싸 드실 때 어디에 많이 싸서 드시나요?},
  advq={서산에서는 상추를 어느 계절에 기르나요?}
}

\newpage
\Entry{
  word={오이},
  pred={우이},
  feat={ms,sp,dp},
  desc={},
  qstn={덩굴에서 나는데 녹색으로 길쭉하게 생기고 겉이 우툴두툴한 것은 무엇인가요? (시원한 맛이 있고 물이 많음)},
  advq={오이로 무슨 반찬을 해 드시나요?}
}

\Entry{
  word={부추},
  pred={졸},
  feat={ms,sp,dp},
  desc={다른 단어 정구지, 솔, 소풀, 세우리 등},
  qstn={파처럼 생겼는데 좀더 가늘고, 겉절이나 무침으로도 먹는 것을 무엇이라고 하시나요?},
  advq={서산에서는 부추를 주로 어떤 음식으로 해 먹나요? 무슨 음식에 넣어서 먹나요?}
}

\Entry{
  word={진달래},
  pred={진달레/진달래},
  feat={ms,sp,dp},
  desc={1. 참꽃 의 의미 2. 나무와 꽃의 명칭이 다른지 3. 단어 첫 자음 /ㅈ/},
  qstn={봄에 산에서 가장 먼저 피는 꽃은 무엇인가요? (색깔이 벌겋고 화전을 해 먹거나 술을 담가 먹기도 함.)},
  advq={(A32 철쭉과 함께 진행)}
}

\Entry{
  word={철쭉},
  pred={철쭉},
  feat={ms,sp,dp},
  desc={1. 개꽃의 의미 2. 철쭉이 꽃을 의미하는지},
  qstn={진달래(29번 방언형)랑 비슷하게 생겼는데 좀 더 늦게 피고 먹지 못하는 것은 무엇이라고 하시나요?},
  advq={진달래와 철쭉의 차이점은 무엇인가요? 둘을 어떻게 구분할 수 있나요?}
}

\newpage
\Entry{
  word={삘기},
  pred={삐비},
  feat={ms,sp,dp},
  desc={고어 삐유기},
  qstn={들에 나는 풀 종류이고, 실뭉치 같은 속줄기를 뽑아서 먹는 풀을 무엇이라고 하시나요? (다 자라면 ‘띠’라고 함.)},
  advq={삘기가 무슨 맛인지 아시나요? (먹어보신 적이 있으시다면) 설명해주실 수 있나요? (B33-B36 맛 형용사들 참고)}
}

\Entry{
  word={머루},
  pred={머루},
  feat={ms,sp,dp},
  desc={고어 멀위, 머뤼},
  qstn={산에서 나고 포도랑 비슷하게 생겼는데 좀 작고 담금주도 담가 먹는 열매는 무엇인가요?},
  advq={머루와 포도의 차이점은 무엇인가요? 구별하는 법이 있나요?}
}

\Entry{
  word={다래},
  pred={다래},
  feat={ms,sp,dp},
  desc={},
  qstn={산에 많고 크기는 대추만 한데 초록색이고 가을에 누런 풀색으로 익으면 따먹는 열매는 무엇인가요?},
  advq={머루(A34)와 다래는 산에서 난다고 들었는데, 이걸 키우거나 농사짓는 분은 없나요?}
}

\newpage
\Entry{
  word={고욤},
  pred={고염},
  feat={ms,sp,dp},
  desc={돌감과의 구별에 유의},
  qstn={빛깔은 감이랑 비슷한데 더 작고 처음에는 떫지만 익으면 달아지는 열매는 무엇인가요?},
  advq={고욤은 떫다고 들었는데 고욤을 먹는 방법이 따로 있나요? \\ 돌감과 고욤은 다른 열매인가요?}
}

\Entry{
  word={복숭아},
  pred={복쑹아},
  feat={ms,sp,dp},
  desc={복셩화, 복숑화... > 복숭},
  qstn={크기는 사과만 한데 색깔은 하얗거나 분홍색이고 물이 많은 과일은 무엇인가요?},
  advq={(A38 배와 함께 진행) 혹시 복숭아를 제사상에 올리기도 하나요?}
}

\Entry{
  word={배(과일)},
  pred={베},
  feat={ms,sp,dp},
  desc={배(신체), 배(계산)와의 성조음장 차이},
  qstn={사과보다 좀 크고 껍질이 노란데, 맛이 달며 살이 연하고 물이 아주 많이 나는 과일은 무엇인가요? (제사상에 올라가는 과일: 사과, 배, 감)},
  advq={제사상에 올릴 때 과일들 순서는 어떻게 되고 얼마나 올리나요? (A39 밤과 함께 진행할 수 있음)}
}

\newpage
\Entry{
  word={밤(열매)},
  pred={밤:},
  feat={ms,sp,dp},
  desc={170. 밤[夜] 과의 성조음장 차이},
  qstn={가시가 난 송이에 싸여 있고 얇고 맛이 떫은 속껍질이 있으며 날것으로 먹거나 굽거나 삶아서 먹을 수 있는 열매를 무엇이라 하시나요?},
  advq={서산에서는 밤이 많이 나나요?}
}

\Entry{
  word={뿌리},
  pred={뿌리},
  feat={ms,sp,dp},
  desc={불휘, 불희... > 뿌리},
  qstn={식물의 아래 부분으로 보통 땅속에 묻혀 있는 부분을 무엇이라고 하시나요? (수분과 양분을 빨아들이는 부분)},
  advq={달래(A26)는 뿌리 부분까지도 먹나요?}
}

\Entry{
  word={무},
  pred={무수},
  feat={ms,sp,dp},
  desc={장모음 여부, 무ᅀᅮ, 무우 > 무},
  qstn={색깔이 하얗고 굵은데, 뿌리(A40)를 먹으려고 키우는 농작물은 무엇인가요? (김장해 먹기도 함.)},
  advq={무의 줄기도 먹나요? 열무가 혹시 무의 줄기인가요?}
}

\newpage
\section{절 예시 (2)}
\subsection{일러두기}
일러둘 내용을 적습니다.

\subsection{자유발화 질문}
자유발화 질문을 적습니다.

\subsection{목표 어휘}
목표 어휘와 예상 형태, 질문을 적습니다. 다음과 같이 적습니다. \\

\Entry{
  word = {부엌},
  pred = {부엌},
  feat = {ms,sp},
  desc = {%
    아궁이를 의미하는지, 다른단어 정지 등, \\
    기저 종성 /ㅋ/
  },
  qstn = {%
    식사를 준비하거나 설거지를 할 때 어디로 가시죠? \\
    집안에서 요리기구들이 주로 어디에 있죠?
  },
  advq = {%
    정지라는 말을 들어보셨나요/쓰시나요? \\
    들어보셨다면 부엌과 정지 사이에 의미 차이가 있나요?
  }
}

\chapter{문법편}
\subsection{일러두기}
일러둘 내용을 적습니다.

\section{절 예시}
\subsection{일러두기}
일러둘 내용을 적습니다.

\subsection{소절 예시}
목표 문법 표지와 질문을 적습니다.


\chapter{어휘 의미지도}
\section{절 예시}
주제별로 의미지도를 그립니다.


\chapter{사진 자료}
참고할 사진을 넣습니다.


\chapter*{부록}
\begin{appendices}

\section{어휘 체크리스트}
어휘 체크리스트를 적습니다.


\section{문법 체크리스트}
문법 체크리스트를 적습니다.

\end{appendices}


\backmatter
\makebackcover

\end{document}