%!TEX root = main.tex
\section{사회적 지식}
\subsection{일러두기}
일러둘 내용을 적습니다.

\subsection{자유발화 질문}
자유발화 질문을 적습니다.

\subsection{목표 어휘}
\Entry{
  word={마을},
  pred={동네/마실},
  feat={ms,sp,dp},
  desc={/ᄆᆞᅀᆞᇕ.../ > /마을/},
  qstn={지금 사시는 곳 or 예전에 살아오셨던 곳을 소개해주세요. \\ 집집들이 모여서 서로 돕고 사는 곳을 뭐라고 부르나요?},
  advq={옆 동네/옆 마을과는 어떤 관계인가요?}
}

\Entry{
  word={외국},
  pred={},
  feat={ms,sp,dp},
  desc={V1 /ㅚ/},
  qstn={우리나라 말고 다른 곳으로 여행 가보신 적 있으신가요? / 어디를 가고 싶으신가요? / 우리나라가 아닌 다른 나라를 무엇이라고 부르나요?},
  advq={다른 가족분들이 외국에 대해 해주신 이야기가 있을까요?}
}

\Entry{
  word={어디},
  pred={위디, 워디, 어디서:, 어서, 어이서, 워이서, 워디서},
  feat={ms,sp,dp},
  desc={},
  qstn={친구분이 급하게 막 가고 있을 때, 어떤 곳을 향해 가는지 어떻게 물어보시나요? / 어떤 곳에서 왔는지는 어떻게 물어보나요?},
  advq={어린 아이라면 어떻게 물어보시겠어요? \\ 어머니 혹은 아버지라면 어떻게 물어보시겠어요?}
}

\Entry{
  word={모시다/모셔라},
  pred={메시다. 모시고, 모셔야},
  feat={ms,sp,dp},
  desc={},
  qstn={자녀분들이 성인이 되고 나서, 혹은 성인이 되고 나신 후 부모님과 함께 여행하신 적이 있으신가요? / 그분들과 어떻게 여행을 다니셨나요?},
  advq={지금 자녀분들과 함께, 혹은 과거로 돌아가 부모님과 함께 가고 싶은 여행지가 있으신가요?}
}

\Entry{
  word={하나, 둘, ... 열},
  pred={하나, 둘:, 싯/스이/셋:, 닛/느이/넷:, 다섯, 여섯, 일곱, 여덜, 아홉, 열:},
  feat={ms,sp,dp},
  desc={},
  qstn={물건을 세듯이 하나부터 열까지만 천천히 숫자를 세주시겠어요?},
  advq={너희 중 하나가 밥을 먹느냐는 질문에, /하나/의 자리에 다른 숫자를 넣어서 천천히 한 번 더 말씀해 주시겠어요?}
}

\Entry{
  word={스물, 서른, ... 아흔},
  pred={스물, 서른, 마은/마흔, 쉰/시운, 예순, 이른, 여든, 아은/아흔},
  feat={ms,sp,dp},
  desc={/마ᅀᅳᆫ/ 관련 이형태 \\ /스믈ㅎ.../ > /스물//셜흔.../ > /서른/ \\ /려쉰.../ > /예순//닐흔.../ > /일흔/},
  qstn={이번에는 스물, 서른 하는 식으로 백까지만 천천히 숫자를 세주시겠어요?},
  advq={내 나이가 열이다라는 문장에, /열/의 자리에 스물, 서른 등의 숫자를 넣어서 천천히 한 번 더 말씀해주시겠어요?}
}

\Entry{
  word={에누리},
  pred={에느리},
  feat={ms,sp,dp},
  desc={},
  qstn={시장에서 물건을 사실 때, 무조건 써진 가격으로만 사야 하나요? \\ 조금 싸게 사려고 가격을 깎는 것을 무엇이라고 하나요?},
  advq={에누리를 잘하려면 어떻게 해야 하나요?}
}

\Entry{
  word={되},
  pred={데/되(단모음)},
  feat={ms,sp,dp},
  desc={V1 /ㅚ/},
  qstn={시장에서 콩이나 팥 같은 것을 어떤 단위로 파나요? \\ 열 홉이 모이면 무엇이 되나요?},
  advq={한 되씩 혹은 두세 되씩 파는 물건에는 또 무엇이 있나요?}
}

\Entry{
  word={말(단위)},
  pred={말},
  feat={ms,sp,dp},
  desc={/말/(동물), /말/(언어)와의 성조 음장 차이},
  qstn={시장에서 기름을 짜기 위한 들깨 같은 것을 어떤 단위로 파나요? \\ 열 되가 모이면 무엇이 되나요?},
  advq={한 말씩 혹은 두세 말씩 파는 물건에는 또 무엇이 있나요?}
}

\Entry{
  word={배(연산)},
  pred={},
  feat={ms,sp,dp},
  desc={/배/(과일), /배/(신체)와의 성조 음장 차이},
  qstn={한 말은 한 되의 열 무엇이라고 하시나요? \\ 곱절을 다른 말로 무엇이라고 하시나요?},
  advq={n 배가 관찰된 경우) 혹시 n을 빼고 발음하면 어떻게 되나요?}
}

\Entry{
  word={금(금속)},
  pred={},
  feat={ms,sp,dp},
  desc={/금긋다/의 /금/과의 성조 음장 차이},
  qstn={시장에서는 보석도 파나요? / 비싼 보석이나 반지, 목걸이는 어디서 파나요? / 결혼하실 때 그곳에서 반지(가락지)를 맞추셨나요?},
  advq={금덩어리를 정해진 양으로 네모낳게 만든 것을 무엇이라고 하나요?}
}