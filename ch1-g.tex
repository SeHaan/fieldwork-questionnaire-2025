%!TEX root = main.tex
\section{기타 동사}
\subsection{일러두기}
일러둘 내용을 적습니다.

\subsection{자유발화 질문}
자유발화 질문을 적습니다.

\subsection{목표 어휘}
% \Entry 부분을 수정하시면 됩니다.
% word: 표준어형
% pred: 예상형태
% feat: 유의점
%       feat에 들어가는 인자는 쉼표(,)로 구분되며,
%       ms, sp, dp 세 종류가 있습니다.
%       feat에 들어간 인자가 볼드체로 표시됩니다.
% desc: 설명
% qstn: 유도질문
% advq: 심화질문
%
% 그리고 줄바꿈이 필요할 땐 엔터가 아니라 '\\'를 사용해주세요(중요!!).

\Entry{
  word={매맞다/매맞아라},
  pred={},
  feat={ms,sp,dp},
  imag={img/refs/매맞다-매맞아라},
  desc={V1 /ㅐ/ 장단음 \\ V1 /ㅐ-ㅔ/ 대립},
  qstn={어린 시절 부모님은 엄한 분이셨나요? / 어른이 혼을 내면서 회초리 같은 것을 떄리고 있으면, 아이가 무엇을 맞고 있다고 하나요?},
  advq={매를 맞은 경우) 그때는 친구분들도 대부분 그랬나요? \\ 매를 맞지 않은 경우) 혹시 부모님이 그러시지 않은 이유를 아시나요?}
}

\Entry{
  word={잃다/잃어버리다},
  pred={잃다(잃어뻐리다). 일치도, 이른, 이러따, 이러버려따},
  feat={ms,sp,dp},
  desc={잊어버리다와의 어휘 병합, 의미 차이},
  qstn={소중하게 가지고 계시던 게 없어졌던 적이 있으신가요? / 가지고 있던 물건이 갑자기 어디론가 없어지면 그것을 어떻게 했다고 하시나요?},
  advq={잃어버리셔서 많이 슬프셨겠어요. 도둑맞거나 한 것은 아니지요?}
}

\Entry{
  word={잊다/잊어버리다},
  pred={잊어뻐리다. 이저따},
  feat={ms,sp,dp},
  desc={잃어버리다와의 어휘 병합, 의미 차이},
  qstn={생각하고 있던 것이 머릿속에서 생각나지 않을 때는 어떻게 했다고 하시나요?},
  advq={무언가 잊어버리셔서 곤란하신 적이 있었나요?}
}

\Entry{
  word={가르치다},
  pred={가리치다. 가르키는, 가르처},
  feat={ms,sp,dp},
  imag={img/refs/가르치다},
  desc={가리키다와의 어휘 병합, 의미 차이},
  qstn={최근에 새로운 사실을 배운 적 또는 알려주신 적이 있으신가요? 선생님이 학생에게 이런저런 것을 알려주는 것을 무엇을 한다고 하나요?},
  advq={정말 많은 도움이 되었겠어요! 그걸 다른 분에게도 말씀해주신 적이 있을까요?}
}

\Entry{
  word={가리키다},
  pred={가리치다. 가리켜, 일러줘야, 가리키면, 가르처},
  feat={ms,sp,dp},
  imag={img/refs/가리키다},
  desc={가르치다와의 어휘 병합, 의미 차이},
  qstn={윗분을 손가락으로 이렇게(허공을 가리키며) 하면 버릇없다고 하잖아요, 어떻게 하면 버릇이 없다고 하는 건가요?},
  advq={그럼 윗분을 예의 바르게 콕 집으려면 어떻게 해야 되나요?}
}