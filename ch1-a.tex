%!TEX root = main.tex
\section{농사와 식물}

\subsection{자유발화 질문}
\begin{itemize}[noitemsep]
  \item 농사를 지으시나요?
  \item 어떤 농사를 하시나요?
  \item 논/밭에서는 주로 무엇을 기르시나요?
  \item 어떤 작물을 길러보셨나요?
  \item (작물 이름)을 키우려면 어떤 과정이 필요한가요?
  \item 좋아하시는 채소가 있으세요?
  \item 어떤 과일을 자주 드시나요?
\end{itemize}

\subsection{목표 어휘}
% \Entry 부분을 수정하시면 됩니다.
% word: 표준어형
% pred: 예상형태
% feat: 유의점
%       feat에 들어가는 인자는 쉼표(,)로 구분되며,
%       ms, sp, dp 세 종류가 있습니다.
%       feat에 들어간 인자가 볼드체로 표시됩니다.
% desc: 설명
% qstn: 유도질문
% advq: 심화질문
%
% 그리고 줄바꿈이 필요할 땐 엔터가 아니라 '\\'를 사용해주세요(중요!!).

\Entry{
  word = {벼},
  pred = {나락},
  feat = {ms},
  imag = {img/refs/벼},
  desc = {%
    벼이삭(나락모가지), 벼의 열매(쌀), \\
    쭉정이, 식물 벼의 명칭 등 의미 차이 확인
  },
  qstn = {논에서 자라는, 익을수록 고개를 숙이는 작물을 뭐라고 하시나요?},
  advq = {%
    벼가 다 자라면 어떻게 수확하고 처리하세요? \\
    다 자란 벼를 베고 남는 밑부분을 뭐라고 하시나요? (예상형태: 나락뜰컹/나락끌텅)
  }
}

\Entry{
  word={쌀},
  pred={쌀},
  feat={},
  imag={img/refs/쌀},
  desc={},
  qstn={벼(A01)의 껍질을 벗긴 것을 뭐라고 하시나요?},
  advq={%
    쌀을 어떻게 보관하나요? (예상형태: 쌀똑/차똑) \\
    덜 익은 벼이삭의 껍질을 벗긴 쌀을 뭐라고 하시나요? (예상형태: 묵지쌀/북떼기쌀)
  }
}

\newpage
\Entry{
  word={볍씨},
  pred={씬나락},
  feat={},
  imag={img/refs/볍씨},
  desc={},
  qstn={벼(A01)의 씨를 무엇이라고 하시나요?},
  advq={볍씨/종잣벼라는 말도 쓰시나요?}
}

\Entry{
  word={못자리},
  pred={모짜리},
  feat={},
  imag={img/refs/못자리},
  desc={},
  qstn={모내기하기 전에 볍씨(A03)를 어디에 심나요?},
  advq={%
    못자리와 논은 보통 근처에 있나요? \\
    못자리에서 어떻게 논으로 모를 옮겨 심나요?
  }
}

\Entry{
  word={보리},
  pred={보리},
  feat={},
  imag={img/refs/보리},
  desc={},
  qstn={%
    벼를 수확하고 난 뒤에 겨울에 길러서 봄여름에 수확하는 곡식이 무엇인가요? \\
    옛날에 쌀이 다 떨어지면 대신 먹는 곡식이 무엇이었나요?
  },
  advq={%
    보릿고개라는 말이 정확히 무슨 뜻인지 설명해주실 수 있나요? \\
    보리쌀이라는 말도 쓰시나요?
  }
}

\newpage
\Entry{
  word={깜부기},
  pred={깜ː북},
  feat={},
  imag={img/refs/깜부기},
  desc={},
  qstn={보리(A05) 이삭(모가지)에 병이 들어 까맣게 되는 것을 뭐라고 하시나요?},
  advq={깜부기병 말고 곡식에 드는 병이 있나요?}
}

\Entry{
  word={조},
  pred={서ː숙, 서ː숙쌀},
  feat={},
  imag={img/refs/조},
  desc={},
  qstn={%
    (A08 수수, A09 기장과 함께) 잡곡밥에는 무엇을 넣나요? \\
    벼/보리 말고 농사 짓는 곡식에는 무엇이 있나요? \\
    (조) 노랗고 모래알처럼 알갱이가 조그마한 곡식을 무엇이라고 하나요?
  },
  advq={좁쌀이라는 말도 쓰시나요?}
}

\Entry{
  word={수수},
  pred={쑤시, 쒸시},
  feat={},
  imag={img/refs/수수},
  desc={},
  qstn={알갱이가 붉은색을 띠고 동글동글한데 경단도 만들어 먹는 곡식을 뭐라고 하시나요?},
  advq={%
    (A09 기장과 함께) 수수와 기장을 잡곡밥에 넣기도 하나요? \\
    보름날 먹는 오곡밥에는 무엇을 넣나요?
  }
}

\newpage
\Entry{
  word={기장},
  pred={-},
  feat={},
  imag={img/refs/기장},
  desc={},
  qstn={조처럼 노란데 조보다는 약간 굵고 알갱이가 더 큰 곡식을 뭐라고 하시나요?},
  advq={}
}

\Entry{
  word={옥수수},
  pred={옥소시, 강넹이},
  feat={ms},
  imag={img/refs/옥수수},
  desc={옥수수/강냉이의 의미 차이, \\ 열매와 열매 자루의 의미 차이 확인},
  qstn={통째로 쪄서 먹는데, 찌면 알갱이가 노랗고, 사료로도 쓰는 곡식을 뭐라고 하시나요?},
  advq={%
    옥소시/강넹이라는 말도 쓰나요? 뭐가 다른가요? \\
    벌교에서 옥수수가 나나요? 아니면 다른 지역에서 옥수수가 들어오나요?
  }
}

\Entry{
  word={새끼(줄)},
  pred={사네끼, 사쳉이},
  feat={sp},
  imag={img/refs/새끼(줄)},
  desc={2음절 /ㅐ-ㅔ/대립},
  qstn={지푸라기를 비벼 만든 끈을 뭐라고 하시나요?},
  advq={지푸라기로 새끼를 만드는 것을 뭐라고 하시나요? (새끼를 까다)}
}

\Entry{
  word={새참},
  pred={셋ː꺼리, 셋ː빱, 쎄클},
  feat={},
  imag={img/refs/새참},
  desc={},
  qstn={김도 매고 타작도 하고 일하다가, 시장할 때 드시는 것을 뭐라고 하시나요?},
  advq={새참으로는 주로 어떤 것들을 드시나요?}
}

\newpage
\Entry{
  word={절구},
  pred={도ː구통},
  feat={},
  imag={img/refs/절구},
  desc={},
  qstn={곡식이나 떡을 찧을 때에는 어디에다 찧으시나요?},
  advq={%
    절구는 주로 돌로 만드나요? \\
    절구를 찧을 때 쓰는 막대기를 뭐라고 하시나요? (도굿대 등)
  }
}

\Entry{
  word={겨},
  pred={제},
  feat={},
  imag={img/refs/겨},
  desc={},
  qstn={절구(A13)에 곡식을 찧고 나면 남는 껍질을 뭐라고 하시나요?},
  advq={%
    겨와 쭉정이의 차이점이 무엇인가요? \\
    겉껍질(`왕겨' 껏불제 등)과 속껍질(`등겨' 누무께, 등게 등)을 각각 뭐라고 하시나요?
  }
}

\Entry{
  word={키},
  pred={쳉이},
  feat={},
  imag={img/refs/키},
  desc={},
  qstn={%
    곡식에서 쭉정이나 티끌을 털어서 골라낼 때 쓰는 것을 뭐라고 하시나요? \\
    예전에 오줌싸개가 쓰고 다니던 것이 무엇인가요?
  },
  advq={}
}

\newpage
\Entry{
  word={김매기},
  pred={지심메다},
  feat={sp,dp},
  imag={img/refs/김매기},
  desc={\jamoword{gi/zim/m@i/da}>김매다},
  qstn={논밭에 잡초가 많이 나면 어떻게 하시나요? 논에서 잡초를 없애는 것을 무엇이라고 하나요?},
  advq={%
    논에서 하는 것과 밭에서 하는 것 모두 김매기라고 부르나요? \\
    마지막 김매기를 부르는 말이 있나요? (맘ː물, 만드리) \\
    마지막 김매기를 끝내고 하루를 즐기는 풍습이 있나요? 뭐라고 부르나요? (두레지심, 모심다레기, 술멕이, 시름짱, 풍년다레기, 써레시침)
  }
}

\Entry{
  word={호미},
  pred={호무, 호멩이},
  feat={},
  imag={img/refs/호미},
  desc={},
  qstn={%
    밭에서 김매기(A16)를 하실 때는 무엇을 가지고 하시나요? \\
    감자나 고구마 등을 캘 때 사용할 때 어떤 도구를 쓰시나요?
  },
  advq={%
    가장 첫 호미질을 부르는 말이 있나요? (호무꾸리) \\
    호미가 농사에서 이것저것 쓰임새가 많다는데, 호미로 할 수 있는 작업에는 무엇이 있나요?
  }
}

\Entry{
  word={써레},
  pred={써ː레},
  feat={dp},
  imag={img/refs/써레},
  desc={써흐레>써레},
  qstn={긴 나무에 발을 박아 놓고 끌면서, 갈아 놓은 논의 바닥을 고르는 농기구를 뭐라고 하시나요? },
  advq={(A18 {\textasciitilde} A21을 함께 진행)}
}

\newpage
\Entry{
  word={쇠스랑},
  pred={소시랑},
  feat={},
  imag={img/refs/쇠스랑},
  desc={},
  qstn={땅을 파헤쳐 고르거나 두엄을 칠 때 사용하는 갈퀴 모양의 농기구를 뭐라고 하시나요?},
  advq={(A18 {\textasciitilde} A21을 함께 진행)} % TODO: '두엄'도 확인할까?
}

\Entry{
  word={가래(농기구)},
  pred={가레},
  feat={sp},
  imag={img/refs/가래(농기구)},
  desc={2음절 /ㅐ-ㅔ/ 대립},
  qstn={%
    한 사람이 자루를 잡고 두 사람이 줄을 잡아당겨 흙을 퍼서 던지는 농기구를 뭐라고 하시나요?
  },
  advq={%
    가래는 삽과는 다른 것인가요? 어떻게 다른가요? \\
    (A18 {\textasciitilde} A21을 함께 진행)
  }
}

\Entry{
  word={도리깨},
  pred={돌께, 도리께},
  feat={},
  imag={img/refs/도리깨},
  desc={},
  qstn={긴 막대기에 싸리나 대나무를 달아 곡식을 두드려 타작하는 농기구를 뭐라고 하시나요?},
  advq={써레/쇠스랑/가래/도리깨(A18 {\textasciitilde} A21)를 요즘 농사지을 때에도 사용하시나요? 옛날 농사에만 사용했나요?}
}

\newpage
\Entry{
  word={멍석},
  pred={덕썩},
  feat={},
  imag={img/refs/멍석},
  desc={},
  qstn={곡식을 타작하거나 널어 말릴(7.3) 때 어디에 깔아 놓으시나요?},
  advq={%
    요즘에도 멍석을 깔고 무언가를 말리기도 하나요? 무엇을 말리나요? \\
    곡식을 말리는 멍석을 부르는 말이 따로 있나요? (예상형태: 우게덕썩[화순])
  }
}

\Entry{
  word={깨},
  pred={꾀},
  feat={},
  imag={img/refs/깨},
  desc={},
  qstn={참기름이나 들기름은 무엇을 짜서 만드시나요?},
  advq={참깨와 들깨는 어떻게 구분하나요?}
}

\newpage
% new lexicon
\Entry{
  word={나무},
  pred={나무, 낭구},
  feat={ms,dp},
  imag={img/refs/나무},
  desc={다양한 조사 결합형을 조사할 수 있도록 할 것; /나모{\textasciitilde}남ㄱ/ (< /*\jamoword{na/m@k/}? *\jamoword{nam/k@/}?)},
  qstn={%
    단단한 줄기와 가지와 뿌리와 잎을 가진 식물을 뭐라고 부르지요?
  },
  advq={%
    나무는 어디에서 많이 자라나요? \\
    나무의 종류로는 어떤 것들이 있나요? \\
    나무의 줄기, 가지, 잎, 뿌리 등은 어디에 쓸 수 있나요? \\
    나무를 올라 보신 적이 있으신가요?
  }
}

\newpage
\Entry{
  word={나물},
  pred={노무세, 노물},
  feat={},
  imag={img/refs/나물},
  desc={},
  qstn={%
    산이나 들에 나는 풀 중에서 캐다가 무쳐 먹는 것을 뭐라고 하시나요?
    대보름에 오곡밥과 함께 먹는 것을 뭐라고 하시나요?
  },
  advq={%
    나물 종류로는 어떤 것이 있나요? (A26 {\textasciitilde} A28) \\
    명절에 나물반찬 많이 하시나요? 어떤 나물들을 하시나요? (B26 부침개나 A39 배, A40 밤 등 명절 음식 관련)
  }
}

\Entry{
  word={냉이},
  pred={나상구, 나상게, 나셍기, 나싱게, 나셍이},
  feat={dp},
  imag={img/refs/냉이},
  desc={\jamoword{na/zi/}, 낭히>냉이},
  qstn={봄에 들이나 밭에서 뿌리(A41)째로 캐서 된장국을 해 먹는 것을 뭐라고 하시나요?},
  advq={냉이랑 비슷하게 생긴 식물은 무엇이 있나요? 냉이랑 구분할 수 있는 방법이 있나요?}
}

\newpage
\Entry{
  word={달래},
  pred={달룽게},
  feat={},
  imag={img/refs/달래},
  desc={},
  qstn={%
    파와 같은 냄새가 나고 매운 맛이 나는 나물인데, 양파보다는 훨씬 작은 것은 무엇인가요? \\
    봄에 들에서 나는, 뿌리(A41)는 조그맣고 동글동글하면서 줄기는 실처럼 가늘고 긴 것은 무엇인가요?
  },
  advq={}
}

\Entry{
  word={콩나물},
  pred={콩노믈},
  feat={ms},
  imag={img/refs/콩나물},
  desc={무친 콩나물/무치지 않은 콩나물의 명칭 차이},
  qstn={콩에 물을 주어 길러 먹는 것은 뭐라고 하시나요?},
  advq={콩나물로는 무슨 반찬을 해 드시나요? 콩지름/콩질금이라는 말도 쓰시나요?}
}

\Entry{
  word={상추},
  pred={상추},
  feat={},
  imag={img/refs/상추},
  desc={},
  qstn={쌈을 싸 드실 때 어디에 많이 싸서 드시나요?},
  advq={상추는 어느 계절에 기르나요? (C06 겨울 및 관련 어휘)}
}

\Entry{
  word={오이},
  pred={믈외, 외ː},
  feat={},
  imag={img/refs/오이},
  desc={},
  qstn={%
    덩굴에서 나는데 녹색으로 길쭉하게 생기고 겉이 우툴두툴하니, \\
    시원한 맛이 있고 물이 많은 것은 무엇인가요?
  },
  advq={오이로 무슨 반찬을 해 드시나요?}
}

\newpage
\Entry{
  word={부추},
  pred={솔ː},
  feat={},
  imag={img/refs/부추},
  desc={},
  qstn={파처럼 생겼는데 좀 더 가늘고, 겉절이나 무침으로도 먹는 것을 뭐라고 하시나요?},
  advq={%
    소풀/소불이라는 말은 안 쓰시나요? \\
    부추를 주로 어떤 음식으로 해 먹나요? 무슨 음식에 넣어서 먹나요?
  }
}

\Entry{
  word={진달래},
  pred={창꼿},
  feat={ms,sp},
  imag={img/refs/진달래},
  desc={참꽃/진달래의 의미 차이, 나무와 꽃의 명칭 차이},
  qstn={%
    봄에 산에서 가장 먼저 피는 꽃은 무엇인가요? \\
    색깔이 벌겋고 화전을 해 먹거나 술을 담가 먹기도 하는 꽃은 무엇인가요?
  },
  advq={진달레라는 말도 쓰시나요? 창꼿(참꽃)은 무슨 뜻인가요?}
}

\Entry{
  word={철쭉},
  pred={게ː꼿},
  feat={ms},
  imag={img/refs/철쭉},
  desc={개꽃의 의미, 나무와 꽃의 명칭 차이},
  qstn={진달래(A30)와 비슷하게 생겼는데 좀 더 늦게 피고 먹지 못하는 것은 뭐라고 하시나요?},
  advq={%
    철쭉이라는 말도 쓰시나요? 게ː꼿(개꽃)은 무슨 뜻인가요? \\
    진달래와 철쭉의 차이점은 무엇인가요? 둘을 어떻게 구분할 수 있나요?
  }
}

\newpage
\Entry{
  word={삘기},
  pred={삐ː비, 뛰},
  feat={ms,dp},
  imag={img/refs/삘기},
  desc={삐유기>삘기 /k>p/, 삘기가 다 자라면 띠},
  qstn={들에 나는 풀 종류이고, 실뭉치 같은 속줄기를 뽑아서 먹는 풀을 무엇이라고 하시나요?},
  advq={%
    삐ː비/뛰라는 말도 쓰시나요? 뭐가 다른가요? \\
    삘기를 드셔 보셨나요? 무슨 맛인지 설명해주실 수 있나요? (B33 맛 형용사)
  }
}

\Entry{
  word={머루},
  pred={멀구},
  feat={ms,dp},
  imag={img/refs/머루},
  desc={머뤼>머루},
  qstn={산에서 나는, 포도랑 비슷한데 좀 작고 담금주도 담가 먹는 열매는 무엇인가요?},
  advq={머루와 포도의 차이점은 무엇인가요? 구별하는 법이 있나요?}
}

\Entry{
  word={다래},
  pred={다ː레},
  feat={},
  imag={img/refs/다래},
  desc={},
  qstn={산에 많고 크기는 대추만 한데 초록색이고 가을에 누런 풀색으로 익으면 따먹는 열매는 무엇인가요?},
  advq={머루(A35)와 다래는 산에서 난다고 들었는데, 이걸 키우거나 농사짓는 분은 없나요?}
}

\newpage
\Entry{
  word={고욤},
  pred={괴양감},
  feat={ms},
  imag={img/refs/고욤},
  desc={돌감과의 의미 차이},
  qstn={빛깔은 감이랑 비슷한데 훨씬 작고, 처음에는 떫지만 익으면 달아지는 열매는 무엇인가요?},
  advq={고욤은 돌감(산감)과는 다른 열매인가요? 고욤은 떫다고 들었는데 먹는 방법이 따로 있나요?}
}

\newpage
\Entry{
  word={복숭아},
  pred={복성},
  feat={},
  imag={img/refs/복숭아},
  desc={},
  qstn={크기는 사과만 한데 색깔은 하얗거나 분홍색이고 물이 많은 과일은 무엇인가요?},
  advq={(A38 {\textasciitilde} A40을 함께 진행) 복숭아를 제사상에 올리기도 하나요?}
}

\Entry{
  word={배(과일)},
  pred={베},
  feat={sp},
  imag={img/refs/배(과일)},
  desc={배(신체), 배(연산)와의 성조, 장단 차이},
  qstn={사과보다 좀 크고 껍질이 노란데, 살이 연하고 달며 물이 많은 과일은 무엇인가요?},
  advq={(A38 {\textasciitilde} A40과 함께) 제사상에 어떤 과일을 올리시나요? 올리는 순서는 어떻게 되나요?}
}

\Entry{
  word={밤(열매)},
  pred={밤:},
  feat={sp},
  imag={img/refs/밤(열매)},
  desc={밤(夜)과의 성조, 장단 차이},
  qstn={가시가 난 송이에 싸여 있고 얇고 맛이 떫은 속껍질이 있으며 날것으로 먹거나 굽거나 삶아서 먹을 수 있는 열매를 뭐라고 하시나요?},
  advq={(A38 {\textasciitilde} A40을 함께 진행) 밤을 제사상에 올리기도 하나요?}
}

\Entry{
  word={뿌리},
  pred={꿀텅, 뿌리, 뿌렝이},
  feat={ms,dp},
  imag={img/refs/뿌리},
  desc={꿀텅/뿌리 의미 차이, 불휘/불희>뿌리},
  qstn={보통 땅속에 묻혀서 물과 양분을 빨아들이는 식물의 아래 부분을 무엇이라고 하시나요?},
  advq={꿀텅/뿌리라는 말도 쓰시나요? 어떤 차이가 있나요? 달래(A27)는 뿌리까지도 먹나요?}
}

\newpage
\Entry{
  word={무},
  pred={무시, 무수},
  feat={sp,dp},
  imag={img/refs/무},
  desc={장모음 여부, \jamoword{mu/zu}/무우>무},
  qstn={하얗고 굵은데, 뿌리(A41)를 먹으려고 키우고, 김장해 먹기도 하는 농작물은 무엇인가요?},
  advq={%
    열무라는 말도 쓰나요? 열무가 혹시 무의 줄기인가요? 무의 줄기도 먹나요? \\
    무를 김장하면 뭐라고 부르나요? (석박지, 똑딱지, 쪼각지 등)
  }
}