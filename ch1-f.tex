%!TEX root = main.tex
\section{신체}
\subsection{일러두기}
일러둘 내용을 적습니다.

\subsection{자유발화 질문}
\begin{itemize}[noitemsep]
  \item 요즘 날이 추운데, 어디 불편하시거나 편찮으신 데는 없으세요?
  \item 감기에 걸리면 여기저기가 많이 아프잖아요, 어디가 아프다고 주로 얘기하세요?
\end{itemize}

\subsection{목표 어휘}
% \Entry 부분을 수정하시면 됩니다.
% word: 표준어형
% pred: 예상형태
% feat: 유의점
%       feat에 들어가는 인자는 쉼표(,)로 구분되며,
%       ms, sp, dp 세 종류가 있습니다.
%       feat에 들어간 인자가 볼드체로 표시됩니다.
% desc: 설명
% qstn: 유도질문
% advq: 심화질문
%
% 그리고 줄바꿈이 필요할 땐 엔터가 아니라 '\\'를 사용해주세요(중요!!).

\Entry{
  word={위(방향)},
  pred={욱},
  feat={sp,dp},
  imag={img/refs/위(방향)},
  desc={V1 /ㅟ/ \\ /웋/ > /위/},
  qstn={%
    요통: 그러면 (손으로 가리키며) 여기 엉치뼈 쪽이 아프신 거에요? \\
    두통: 옆에서 누르는 듯이 아프세요? 아니면…. (손으로 머리를 누르며)
  },
  advq={많이 힘드시겠어요…. 그 근처에 더 불편한 곳은 없으세요?}
}

\Entry{
  word={아래},
  pred={아레},
  feat={sp},
  imag={img/refs/아래},
  desc={V2 /ㅐ-ㅔ/ 대립},
  qstn={%
    요통: 그러면 (손으로 가리키며) 여기 날개뼈 근처가 아프신 거에요? \\
    무릎 통증: 여기 무릎 위에 톡 튀어나온 부분이 아프세요? 아니면….
  },
  advq={많이 힘드시겠어요…. 그 근처에 더 불편한 곳은 없으세요?}
}

\Entry{
  word={머리카락},
  pred={멀크락},
  feat={},
  imag={img/refs/머리카락},
  desc={},
  qstn={(지시) 이것은 무엇이라고 하시나요?},
  advq={요즘 사람들 보면, 색깔도 바꾸고 파마도 하고 하잖아요, 어떻게 보이세요? 어떤 게 제일 예쁘다고 생각하세요?}
}

\Entry{
  word={비듬},
  pred={지게미},
  feat={},
  imag={img/refs/비듬},
  desc={},
  qstn={날이 추워지거나 머리를 잘 안 감으면, 머리에서 하얀 게 떨어지잖아요, 그걸 뭐라고 부르세요?},
  advq={그런 (비듬이 많은/비듬이 막 떨어지는) 사람을 보신 적이 있으세요? 그러면 혹시 조금 더럽다고 느껴질까요?}
}

\Entry{
  word={눈(신체)},
  pred={눈},
  feat={sp},
  imag={img/refs/눈(신체)},
  desc={/눈/(날씨)과의 성조·음장 차이},
  qstn={%
    안경이나 돋보기 같은 것을 사용하세요? 항상 필요하신 편이세요? 언제부터 필요하다고 생각하셨어요? \\
    (지시) 이것은 무엇이라고 하시나요?
  },
  advq={젊으셨을 적에는 어땠어요? (군필 남성의 경우 사격 이야기를 곁들여도 좋을 듯)}
}

\Entry{
  word={귀},
  pred={귀},
  feat={sp},
  imag={img/refs/귀},
  desc={V1 /ㅟ/},
  qstn={%
    (지시) 이것은 무엇이라고 하시나요? \\
    소리를 잘 듣는 사람에게 무엇이 밝다고 이야기하시나요?
  },
  advq={무언가를 잘 들으려고 집중할 때, (귀에 손을 모아 가져다 대고 허리를 살짝 숙이며) 이렇게 하는 것을 어떻게 한다고 이야기하시나요?}
}

\Entry{
  word={말(언어)},
  pred={말:},
  feat={sp},
  desc={/말/(동물), /말/(단위)과의 성조·음장 차이},
  qstn={사람들이 입으로 소리를 내어서 다른 사람에게 뭐라고 하는 것을 무엇이라고 하시나요? (또는 말과 관련된 여러 속담 사용)},
  advq={}
}

\Entry{
  word={혀},
  pred={세},
  feat={ms,sp},
  imag={img/refs/혀},
  desc={/혓바닥/과의 의미 차이 \\ C1 + V1 /혀/},
  qstn={(지시) 이것은 무엇이라고 하시나요?},
  advq={혀랑 관련된 재밌는 표현들 없을까요? 흔히 말을 잘하는 사람한테 혀를 잘 놀린다고 말하기도 하잖아요.}
}

\Entry{
  word={딸꾹질},
  pred={포깍질, 포꼭질},
  feat={},
  imag={img/refs/딸꾹질},
  desc={},
  qstn={밥을 갑자기 빨리 먹거나 목이 마르면 목에서 뭔가 꺽꺽하고 올라오는데 그것을 무엇이라고 하시나요?},
  advq={그걸 멈추려면 어떻게 하는 게 가장 효과가 좋으셨나요?}
}

\Entry{
  word={언청이},
  pred={얼쳉이},
  feat={ms},
  desc={/결슌/ 등 계통이 다른 어휘},
  qstn={태어날 때부터 윗입술이 찢어진 사람을 무엇이라고 하시나요? (비칭임에 유의)},
  advq={그런 사람을 보신 적이 있으신가요?}
}

\Entry{
  word={하품},
  pred={하얌},
  feat={},
  imag={img/refs/하품},
  desc={},
  qstn={잠이 올 때 입이 저절로 벌어지면서 숨을 크게 들이쉬는 것을 무엇 한다고 하시나요?},
  advq={이것이 혹시 버릇없는 행동일 수 있을까요? 그렇다면 하품이 나올 때 지켜야 할 예의가 있을까요? (손으로 입을 가린다거나 하는)}
}

\Entry{
  word={가래},
  pred={가레},
  feat={},
  desc={},
  qstn={%
    감기 걸리면 무엇 때문에 가장 힘드세요? \\
    감기에 걸리면 목에 잔뜩 끼어서 기침이 나오게 하는 것을 무엇이라고 부르시나요?
  },
  advq={저는 아무리 기침을 해도 잘 안 뱉어지더라구요. 혹시 잘 뱉는 법을 아시나요?}
}

\Entry{
  word={목},
  pred={목},
  feat={ms},
  imag={img/refs/목},
  desc={/모가지/와의 의미 차이},
  qstn={%
    감기 걸리면 어디가 제일 아프세요? / 혹시 지금 자세는 편안하세요? \\
    지금까지 말씀 많이 하셨는데, 어디 불편하신 데는 없으세요?
  },
  advq={동물을 잡을 때도 여기를 먼저 치나요?}
}

\Entry{
  word={뺨},
  pred={빰:},
  feat={dp,ms},
  imag={img/refs/뺨},
  desc={/뺨따귀/ 등이 비칭인지 \\ /볼/과의 차이},
  qstn={%
    서로 싸울 때, 화가 나면 어디를 치기도 하지요? \\
    흥부가 놀부 부인한테 주걱으로 어디를 맞았다고 하나요?
  },
  advq={친구끼리 장난으로 여기를 때리는 것도 문제가 될 수 있을까요?}
}

\Entry{
  word={주름살},
  pred={주룸쌀},
  feat={},
  imag={img/refs/주름살},
  desc={},
  qstn={%
    나이가 드시고 난 뒤에 가장 크게 달라진 게 무엇이라고 생각하시나요? \\
    나이가 드신 분의 피부가 자글자글해지는 것을 무엇이라고 부르시나요?
  },
  advq={부모님의 것을 볼 때/자신의 것을 볼 때/자녀의 것을 볼 때 어떤 기분이 드셨나요?}
}

\Entry{
  word={배(신체)},
  pred={베},
  feat={sp},
  imag={img/refs/배(신체)},
  desc={/ㅔ/와 /ㅐ/의 구분(전남 일부 지역에는 남아 있다고 보고됨); /배/(과일), /배/(계산)와의 성조·음장 차이},
  qstn={소화가 잘 안되시지는 않으세요? 어떤 음식이 잘 안 받는 것 같다거나 하시는 건 없으세요? 그 음식을 드시면 어떠세요?},
  advq={윗배가 아픈 것과 아랫배가 아픈 게 어떻게 다른가요?}
}

\Entry{
  word={무릎},
  pred={물팍},
  feat={sp},
  imag={img/refs/무릎},
  desc={기저 종성 /ㅍ/},
  qstn={오래 걸으시면 어디가 주로 아프세요? \\ 비가 올 때 주로 어디가 쑤시는 편이세요?},
  advq={많이 불편하시겠어요…. 조금이라도 덜 아프려면 어떻게 해야 되나요?}
}

\Entry{
  word={뼈},
  pred={뻬다구},
  feat={},
  imag={img/refs/뼈},
  desc={},
  qstn={(지시) 살 속에 들어 있는 딱딱한 것이 무엇인가요?},
  advq={닭이나 생선 말고, 다른 동물 뼈를 보신 적 있으세요? \\ 동물 뼈도 사람 뼈랑 비슷하게 생겼나요?}
}

\Entry{
  word={고름(신체)},
  pred={고름},
  feat={sp},
  desc={V2 /ㅡ/ \\ /고름/(의복)과의 비교},
  qstn={상처가 바로 낫지 않고 곪으면 노랗게 무엇이 나오나요?},
  advq={
    고름이 생기면 짜는 게 좋을까요, 아니면 그대로 두는 게 좋을까요? 혹시 이유도 아시나요? \\
    (B41 구멍 관련) 종기 같은 게 곪아 터지면 그 자리가 뚫려 있잖아요. 그곳을 부르는 이름이 있나요?
  }
}

% new lexicon
\Entry{
  word={부풀다},
  pred={부크다},
  feat={dp},
  desc={C2 /ㅍ/, V2 /ㅜ/},
  qstn={%
    기름진 음식 먹으면 소화가 잘 안 되고, 배에 가스가 차고 그렇잖아요. 이럴 때 배가 어떻게 된다고 하시나요? \\
    사람이 임신하면 배가 이렇게 차오르잖아요. 그걸 배가 어떻게 된다고 하시나요? \\
    풍선에 바람을 넣으면 풍선이 어떻게 된다고 하시나요?
  },
  advq={}
}

\Entry{
  word={엄살},
  pred={엄살},
  feat={},
  desc={},
  qstn={별로 아프지도 않은데 거짓말로 아프다고 막 소리치는 것을 걸 무엇 피운다, 혹은 무엇 부린다고 하나요?},
  advq={누가 아프다고 말할 때 엄살인지 아닌지 알 수 있는 방법이 있을까요?}
}

\Entry{
  word={귀신},
  pred={}, % TODO: 예상형태 기재
  feat={sp},
  imag={img/refs/귀신},
  desc={V1 /ㅟ/},
  qstn={사람이 죽으면 이것이 된다고도 하고, 한을 많이 품고 있어서 산 사람을 괴롭힌다고 하는데, 무엇을 표현하는 말인가요?},
  advq={실제로 귀신을 보신 적이 있으신가요? \\ 혹시 어떤 종류의 귀신들이 있는지 아시나요?}
}

\Entry{
  word={방귀},
  pred={방:구},
  feat={ms},
  imag={img/refs/방귀},
  desc={/똥:/도 쓰이는지 확인할 것},
  qstn={배가 아프면 냄새나는 가스가 나오기도 하는데, 이걸 뭐라고 하나요? \\ 고구마 많이 먹으면 어떤 냄새가 지독해지나요?},
  advq={(F23 구린내와 함께 진행)}
}

\Entry{
  word={구린내},
  pred={},
  feat={ms},
  desc={/고린내/와의 의미 차이},
  qstn={그러면 방귀를 뀌면 어떤 냄새가 난다고 하나요?},
  advq={(/고린내/도 검출된 경우) 두 냄새는 어떤 차이가 있나요?}
}

\Entry{
  word={고린내},
  pred={꼬랑네},
  feat={ms},
  desc={/구린내/와의 의미 차이},
  qstn={안 씻은 발이나 발가락 사이에서는 어떤 냄새가 난다고 하나요?},
  advq={(/구린내/도 검출된 경우) 두 냄새는 어떤 차이가 있나요?}
}

\Entry{
  word={씻다/씻어라},
  pred={씿다, 씨끄다, 싯다}
  feat={dp},
  imag={img/refs/씻다-씻어라},
  desc={/싯다/, /슺다/... > /씻다/; 어원적으로 사동 표지가 개재되었을 가능성도 있음},
  qstn={기분 나쁜 냄새가 나거나 꼬질꼬질해 보이는 사람한테는 뭐라고 명령하시겠어요?},
  advq={목욕이든 샤워든 했는지 물어볼때는 어떻게 물어보세요? \\ 이제 목욕한다고 가족들한테 얘기하실 때 어떻게 말씀하세요?}
}

\Entry{
  word={굵다},
  pred={굵:다, 꿁다, 퉁겁다(퉁거와)},
  feat={ms},
  imag={img/refs/굵다},
  desc={/두껍다/와의 의미 차이},
  qstn={엄지손가락은 새끼손가락보다 어떻다고 하죠?},
  advq={사전이나 전화번호부는 공책보다 굵은 건가요, 두꺼운 건가요? \\ 콩은 쌀보다 알이 굵은가요, 두꺼운가요?}
}

\Entry{
  word={가늘다},
  pred={가늘다, 가늘허다},
  feat={ms},
  imag={img/refs/가늘다},
  desc={/얇다/와의 의미 차이},
  qstn={새끼손가락은 엄지손가락보다 어떻다고 하죠?},
  advq={공책은 사전이나 전화번호부보다 가는 건가요, 얇은 건가요? \\ 보슬비는 억세게 내리는 장맛비보다 빗방울이 가는가요, 얇은가요?}
}