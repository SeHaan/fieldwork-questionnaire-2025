%!TEX root = main.tex
\section{가정과 생활}
\subsection{일러두기}
일러둘 내용을 적습니다.

\subsection{자유발화 질문}
자유발화 질문을 적습니다.

\subsection{목표 어휘}
% \Entry 부분을 수정하시면 됩니다.
% word: 표준어형
% pred: 예상형태
% feat: 유의점
%       feat에 들어가는 인자는 쉼표(,)로 구분되며,
%       ms, sp, dp 세 종류가 있습니다.
%       feat에 들어간 인자가 볼드체로 표시됩니다.
% desc: 설명
% qstn: 유도질문
% advq: 심화질문
%
% 그리고 줄바꿈이 필요할 땐 엔터가 아니라 '\\'를 사용해주세요(중요!!).

\Entry{
  word={부엌},
  pred={부엌},
  feat={ms,sp,dp},
  imag={img/refs/부엌},
  desc={아궁이를 의미하는지, 다른단어 정지 등, 기저 종성 /ㅋ/},
  qstn={식사를 준비하거나 설거지를 할 때 어디로 가시죠? \\ 집안에서 요리기구들이 주로 어디에 있죠?},
  advq={정지라는 말을 들어보셨나요/쓰시나요? 들어보셨다면 부엌과 정지 사이에 의미 차이가 있나요?}
}

\Entry{
  word={솥},
  pred={솟},
  feat={ms,sp,dp},
  imag={img/refs/솥},
  desc={기저 종성 /ㅌ/},
  qstn={주로 쇠붙이로 만들어진, 밥을 짓거나 국을 끓이는 그릇을 뭐라고 부르시나요?},
  advq={요즘에도 가마솥을 쓰나요? 전기밥솥으로 밥을 지을 때와 가마솥으로 밥을 지을 때는 무슨 차이가 있나요?}
}

\Entry{
  word={아궁이},
  pred={아궁이},
  feat={ms,sp,dp},
  imag={img/refs/아궁이},
  desc={부엌과의 의미 범위, 다른 단어 부엌 등, 아궁ㄱ, 아궁지... > 아궁},
  qstn={부엌(B01)에서 솥(B02)을 안치는 곳을 무엇이라고 하시나요? \\ 온돌방을 데울 때 여기다가 불을 때기도 하는데, 어디다 때지요?},
  advq={}
}

\Entry{
  word={부지깽이},
  pred={부지깽이/부지땡이},
  feat={ms,sp,dp},
  imag={img/refs/부지깽이},
  desc={기타 이형태 부지대, 부짓대},
  qstn={아궁이(B03)에 불을 땔 때 쓰는 막대기를 무엇이라고 하시나요?},
  advq={(B05 부삽, B06 화로와 함께 진행)}
}

\Entry{
  word={부삽},
  pred={부삽},
  feat={ms,sp,dp},
  imag={img/refs/부삽},
  desc={},
  qstn={불덩어리를 담아 옮길 때에 쓰는 조그마한 삽을 무엇이라고 하시나요? 난로에 숯을 퍼 담을 때 쓰는 삽을 무엇이라고 부르시나요?},
  advq={부지깽이(B04)와 부삽은 아궁이에 불을 지필 때 사용할 수 있나요? 부삽을 가지고 어디로 불을 옮기나요?}
}

\Entry{
  word={화로},
  pred={화루},
  feat={ms,sp,dp},
  imag={img/refs/화로},
  desc={},
  qstn={주로 불씨를 보존하거나 난방을 위해 쓰는, 숯불을 담아 놓는 그릇을 무엇이라고 하시나요?},
  advq={겨울에 방 안에서 화로를 쓰나요? 화로를 쓸 때 부지깽이(B04)나 부삽(B05)도 사용하나요?}
}

\Entry{
  word={베개},
  pred={버개},
  feat={ms,sp,dp},
  imag={img/refs/베개},
  desc={V1, V2 /ㅐ-ㅔ/ 대립},
  qstn={주무실 때 머리에 무엇을 받치고 주무시나요? (‘베다’라는 동사를 쓸 경우 여기에 이끌려 정확한 방언형이 나오지 않을 수 있음.)},
  advq={어릴 적에는 어떤 베개를 쓰셨나요? 요즘 나오는 베개들과 다른 점이 있나요?}
}

\Entry{
  word={의자},
  pred={이자},
  feat={ms,sp,dp},
  imag={img/refs/의자},
  desc={V1 /ㅢ/},
  qstn={높은 식탁 같은 데에 앉으실 때는 방바닥에 바로 앉지 않고 어디에 앉으시나요?},
  advq={옛날엔 의자보다 방바닥에 주로 앉으셨나요? 집에 의자가 있었나요?}
}

\Entry{
  word={마루},
  pred={말래},
  feat={ms,sp,dp},
  imag={img/refs/마루},
  desc={},
  qstn={집채 안에 널빤지나 나무로 넓게 깔아 놓고 쉬는 곳을 무엇이라고 하시나요?},
  advq={한옥에서 살아보신 경험이 있으신가요? \\ 한옥에는 어떤 마루가 있나요? (대청마루, 툇마루 등)}
}

\Entry{
  word={벽},
  pred={벽},
  feat={ms,sp,dp},
  imag={img/refs/벽},
  desc={벽과 바람벽을 구분하는지},
  qstn={방과 방 사이를 구분하기 위해 저렇게(벽을 가리킴) 천장에서 바닥까지 나누어 놓은 것을 무엇이라고 하시나요?},
  advq={바람벽이라는 말을 쓰시나요? 바람벽과 벽의 뜻에 차이가 있나요?}
}

\Entry{
  word={이엉},
  pred={이엉},
  feat={ms,sp,dp},
  imag={img/refs/이엉},
  desc={},
  qstn={초가지붕을 덮기 위해 짚으로 엮은 것은 무엇이라고 하시나요?},
  advq={초가집의 지붕을 이엉이라고 부르나요? 아니면 지붕의 재료를 이엉이라고 부르나요? 짚으로 된 지붕이 썩으면 이엉을 어떻게 고치나요?}
}

\Entry{
  word={주춧돌},
  pred={주춧똘},
  feat={ms,sp,dp},
  imag={img/refs/주춧돌},
  desc={C1 /ㅈ/},
  qstn={기둥 밑에서 기둥을 받치는 돌은 무엇이라고 하시나요?},
  advq={주춧돌은 네모나게 생겼었나요? 다듬지 않은 돌을 주춧돌로 쓰기도 하나요?}
}

\Entry{
  word={서까래},
  pred={서까래},
  feat={ms,sp,dp},
  imag={img/refs/서까래},
  desc={혓가래... > 서까래},
  qstn={한옥 지붕을 옆으로 비스듬하게 받치고 있는 긴 통나무를 무엇이라고 하시나요?},
  advq={대들보와 서까래의 차이점이 무엇인가요?}
}

\Entry{
  word={변소},
  pred={벤소, 뒷깐},
  feat={ms,sp,dp},
  imag={img/refs/변소},
  desc={다른 단어 뒷간},
  qstn={대변이나 소변은 어디에서 보시나요?},
  advq={(‘화장실’ 등으로 대답하실 경우, ‘주로 시골에서 볼 수 있고 집 밖에 위치한 옛날식 화장실을 무엇이라고 하시나요?’라고 질문한다.)}
}

\Entry{
  word={우물},
  pred={우물},
  feat={ms,sp,dp},
  imag={img/refs/우물},
  desc={바가지 우물, 두레박 우물, 펌프 우물의 명칭 및 의미 차이},
  qstn={땅을 파서 물을 긷는 곳을 무엇이라고 하시나요?},
  advq={바가지로 물을 뜨는 우물과 두레박으로 뜨는 우물을 다르게 부르나요? 펌프로 물을 퍼올리는 우물은 어떤가요?}
}

\Entry{
  word={도랑},
  pred={또랑, 똘캉/똘강},
  feat={ms,sp,dp},
  imag={img/refs/도랑},
  desc={다른 단어 걸, 거렁 등, 돌항... > 도랑},
  qstn={땅을 좁고 깊게 파서 물이 흐르도록 만든 것을 무엇이라고 하시나요?},
  advq={혹시 사람이 판 게 아니라 원래 자연적으로 흐르던 개울도 도랑이라고 부르나요?}
}

\Entry{
  word={민물},
  pred={민물},
  feat={ms,sp,dp},
  desc={},
  qstn={강이나 호수 따위와 같이 짠맛이 나지 않는 물을 무엇이라고 하나요?},
  advq={서산에서는 민물에서 나는 물고기도 먹나요?}
}

\Entry{
  word={광주리},
  pred={광우리},
  feat={ms,sp,dp},
  imag={img/refs/광주리},
  desc={},
  qstn={과일과 곡식 등을 담을 수 있는 버들 따위를 엮어 만든 그릇은 무엇인가요?},
  advq={한 광주리, 두 광주리... 하는 식으로 광주리를 물건을 세는 단위로 쓰시기도 하나요?}
}

\Entry{
  word={궤짝},
  pred={궤(짝)},
  feat={ms,sp,dp},
  imag={img/refs/궤짝},
  desc={V1의 장모음 여부, V1 /ㅞ/},
  qstn={쌀이나 돈과 같은 물건을 넣도록 네모나게 나무로 짠 가구를 무엇이라고 하시나요?},
  advq={(‘궤’라고 대답한 경우) 궤짝이라는 단어는 안 쓰시나요?}
}

\Entry{
  word={맷돌},
  pred={매},
  feat={ms,sp,dp},
  imag={img/refs/맷돌},
  desc={},
  qstn={콩이나 팥 같은 것을 갈 때 어디에 가시나요? (믹서기 말고 옛날에 쓰던 것. 돌로 만든 것)},
  advq={맷돌을 사람 혼자서 사용할 수 있었나요?}
}

\Entry{
  word={체/어레미/도두미},
  pred={-},
  feat={ms,sp,dp},
  imag={img/refs/체-어레미-도두미},
  desc={이형태(어레미에 대해) 얼멍이/얼멩이},
  qstn={가루를 곱게 칠 때 어디에 치시나요?},
  advq={구멍이 굵은 것이랑 가는 것을 다른 이름으로 부르기도 하시나요?}
}

\Entry{
  word={뚝배기},
  pred={뚝배기},
  feat={ms,sp,dp},
  imag={img/refs/뚝배기},
  desc={},
  qstn={그릇 중에서 찌개를 끓이실 때는 무엇을 주로 쓰시나요?},
  advq={국물 요리 중에 뚝배기는 어떤 요리에 쓰시고, 냄비는 어떤 요리에 쓰시나요?}
}

\Entry{
  word={깍두기},
  pred={깍두기},
  feat={ms,sp,dp},
  imag={img/refs/깍두기},
  desc={},
  qstn={무(A41)를 네모나게 썰어 담근 김치를 무엇이라고 하시나요?},
  advq={총각김치와 깍두기의 차이점은 무엇인가요?}
}

\Entry{
  word={두부},
  pred={두부},
  feat={ms,sp,dp},
  imag={img/refs/두부},
  desc={},
  qstn={콩을 갈아서 만든 것으로 하얗고 네모나게 만들어서 된장찌개에 넣어 드시는 것은 무엇인가요?},
  advq={(제보자가 잘못 듣고 ‘메주’라 할 경우 ‘간수를 넣어 만드는 것’이라고 설명한다.)}
}

\Entry{
  word={부침개},
  pred={-},
  feat={ms,sp,dp},
  imag={img/refs/부침개},
  desc={다른 단어 전, 찌짐, 빈대떡, 저냐, 누름적 등},
  qstn={밀가루나 메밀가루를 풀어서 프라이팬에 넓적하게 부쳐 먹는 것을 무엇이라고 하시나요?},
  advq={명절에 만드는 부침개에는 무엇이 있나요? (B37 만들다/만들어라를 유도할 수 있다면 유념해두기) 명절 외에도 부침개를 자주 드시나요?}
}

\Entry{
  word={주걱},
  pred={주걱},
  feat={ms,sp,dp},
  imag={img/refs/주걱},
  desc={C1 /ㅈ/},
  qstn={솥(B02)에서 밥을 풀 때 무엇으로 푸시나요?},
  advq={밥을 풀 때 말고 주걱을 쓰는 경우가 있나요?}
}

\Entry{
  word={누룽지},
  pred={누릉갱이},
  feat={ms,sp,dp},
  imag={img/refs/누룽지},
  desc={의미범위: 솥에서 긁어낸 누룽지를 말린 것의 명칭},
  qstn={솥(B02)에서 밥을 다 펐는데도 솥에 눌어붙은 것은 무엇인가요?},
  advq={(B28 숭늉과 함께 진행)}
}

\Entry{
  word={숭늉},
  pred={숭님},
  feat={ms,sp,dp},
  imag={img/refs/숭늉},
  desc={기타 이형태 /슉랭/},
  qstn={누룽지(B27)에 물을 넣고 끓인 것은 무엇인가요?},
  advq={그냥 밥으로도 숭늉을 끓일 수 있나요? 누룽지로 끓인 것만을 숭늉이라고 하나요?}
}

\Entry{
  word={수제비},
  pred={수제비},
  feat={ms,sp,dp},
  imag={img/refs/수제비},
  desc={찹쌀로 만든 것과 밀가루로 만든 것의 구별 유무},
  qstn={밀가루 반죽을 넓적하게 뜯어서 넣어 만든 국을 무엇이라고 하시나요?},
  advq={서산에서는 수제비를 맑은 국물에 먹나요 빨간 국물에 먹나요? 아니면 둘 다 먹나요?}
}

\Entry{
  word={튀밥},
  pred={광밥},
  feat={ms,sp,dp},
  imag={img/refs/튀밥},
  desc={재료가 옥수수, 보리쌀, 콩일 경우 명칭에 차이가 있는지, 다른 단어 포데기, 뻥튀기, 광밥, 강냉이 등},
  qstn={옥수수(A10)를 튀겨서 만든 간식거리를 무엇이라고 하시나요? 그러면 쌀(A02)을 튀겨 만든 간식거리는 무엇이라고 하시나요? (반드시 두 개를 모두 질문하도록 한다. 두 개를 구분하는 방언이 많다. 기본적으로 튀밥은 튀긴 쌀과 튀긴 옥수수 둘 다 포괄하는 단어다.)},
  advq={(안 나온 어형에 대해 서산에서 쓰는 말인지 물어보기)}
}

\Entry{
  word={식혜/감주},
  pred={감주},
  feat={ms,sp,dp},
  imag={img/refs/식혜-감주},
  desc={찹쌀(식혜)과 멥쌀(감주)의 차이 \\ 다른 단어 청감, 단밥, 단술 등},
  qstn={밥에다가 엿기름을 우려낸 물을 부어 삭히면 되나요? (시원하게 만들어 밥알을 띄워 먹기도 함.)},
  advq={식혜/감주 중 한 어휘가 검출되면 다른 쪽 어휘를 역질문하도록 한다. \\ ex. 식혜 검출 → 감주는 무엇인가요?}
}

\Entry{
  word={왜간장},
  pred={왜간장, 지랑},
  feat={ms,sp,dp},
  desc={V1 /ㅙ/, 간장을 ‘지랑’이라 하기도 함에 주의},
  qstn={일본식으로 만든 간장을 뭐라고 하시나요? (집에서 만드는 재래식 장 다음에 들어온 장. 일본간장/양조간장)},
  advq={조선간장이랑 왜간장이랑 차이가 무엇인가요? 쓰임새가 어떻게 다르나요?}
}

\Entry{
  word={짜다, 쓰다,맵다, 떫다},
  pred={짜다(짜다고, 짜서) / -매웁따, 매운, 매워 / 뜰:븐, 뜰:버},
  feat={ms,sp,dp},
  desc={},
  qstn={소금/약/고추/덜 익은 감은 맛이 어떻다고 하죠?},
  advq={오미자에서 다섯 가지 맛이 나서 오미자라고 하잖아요, 이 다섯 가지 맛이 무엇인지 혹시 아시나요?}
}

\Entry{
  word={만들다/만들어라},
  pred={맨글다, 맨들다. 맨든다, 맨드러라},
  feat={ms,sp,dp},
  desc={},
  qstn={집은 짓는다고 하지만 음식은   (        )다고 하지요.},
  advq={(이전 음식 어휘들에서 포착)}
}

\Entry{
  word={고쟁이},
  pred={고쟁이},
  feat={ms,sp,dp},
  imag={img/refs/고쟁이},
  desc={},
  qstn={한복을 입을 때 여자들이 안에 입는 속옷인데, 속곳 위에 입는 것을 무엇이라고 하시나요?},
  advq={고쟁이는 겨울에도 입나요? 여름에만 입나요?}
}

\Entry{
  word={모시},
  pred={-},
  feat={ms,sp,dp},
  imag={img/refs/모시},
  desc={다른 단어 저, 저마, 저포, 장단음, C2 /ㅅ/},
  qstn={삼베하고 비슷한데 좀 더 촘촘하고, 여름에 옷을 해 입으면 시원한 것은 무엇인가요?},
  advq={‘모시떡’ 할 때 모시가 이 모시와 같나요?}
}

\Entry{
  word={골무},
  pred={골무},
  feat={ms,sp,dp},
  imag={img/refs/골무},
  desc={},
  qstn={바느질할 때 손가락에 끼는 것을 무엇이라고 하시나요?},
  advq={골무는 어느 손가락에 끼나요?}
}

\Entry{
  word={가위},
  pred={가새},
  feat={ms,sp,dp},
  imag={img/refs/가위},
  desc={V2 /ㅟ/, ᄀᆞᅀᅢ > 가},
  qstn={종이나 옷감은 무엇으로 자르시나요? \\ 국수나 냉면 면을 자를 때 무엇을 쓰시나요?},
  advq={옛날 가위도 요즘 가위와 비슷하게 생겼나요?}
}

\Entry{
  word={고름(의복)},
  pred={고름},
  feat={ms,sp,dp},
  imag={img/refs/고름(의복)},
  desc={},
  qstn={한복을 입을 때 옷깃을 여미는 끈을 무엇이라고 하시나요?},
  advq={고름은 어떻게 묶나요?}
}