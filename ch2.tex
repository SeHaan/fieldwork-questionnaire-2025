%!TEX root = main.tex
\chapter{문법편}

\subsection{일러두기}
\begin{itemize}[noitemsep]
  \item 이미 관찰된 문법 요소는 아래 체크리스트를 확인한 뒤 넘어가십시오.
  \item 예문과 조금 다르더라도 특정 문법 요소가 드러난다면 넘어가도 좋습니다.
  \item 표준형 문법 요소를 억지로 이끌어낼 필요는 없습니다. 다만 표준형 대신 쓰인 문법 요소를 잘 기록하고 넘어가도록 합니다. 이때 방언의 음운적 특징을 잘 반영하여(예: 뭐라카다, 해뿌리고) 그대로 기록합니다.
  \item 특정 문법 요소가 드러나지 않는다면 역질문을 통해 사용 여부를 확인합니다. 만약 사용한다는 답변을 들은 경우, 해당 문법 요소는 어떤 상황에서 어떤 의미로 사용하는지 함께 확인합니다.
\end{itemize}

\subsection{작성 요령}
\begin{itemize}[noitemsep]
  \item 문법 조사표 담당자는 자료제공인 조사와 어휘 조사 과정에서 발화되는 조사에 표시합니다.
  \item 반드시 점화효과를 고려하여 체크합니다. 자료제공인의 답변 직전의 조사자의 질문에 문법 표지가 먼저 등장한 경우, 자료제공인이 해당 문법 표지를 사용하더라도 체크하지 않습니다. <어휘편>에서 조사자의 자유발화 질문에 포함된 문법 표지에 의한 점화 효과는 고려하지 않되, 자유발화 질문이 점화 효과를 일으킨 것이 분명하다고 판단되면, 체크하지 않습니다. 
  \item 문법 표지가 자유발화에서 확인되면 `방언형' 난에 어느 조사 파트에서 어떤 형식으로 확인됐는지 적습니다.
    \begin{itemize}[noitemsep]
      \item 예 1: `벼'를 유도하던 중 `벼를'을 검출한 경우 → `벼' / 벼를
      \item 예 2: 자료제공인 조사 중 `해가지구'를 검출한 경우 → 자료제공인 조사 / 해가지구
    \end{itemize}
  \item 조사 현장에서 최대한 문법 표지를 확인한 뒤, 하루의 조사를 마무리할 때 녹음을 다시 들으며 놓친 문법 표지를 다시 확인합니다.
  \item 조사 과정에서 확인되지 않은 문법 사항은 아래 질문을 참고하여 보충합니다.
\end{itemize}

\newpage
\section{조사}
※ 질문에서 조사 노출을 최소화하기 바랍니다. \\
※ 조사 과정에서 자연스럽게 출현한 형태만을 기록하고, 별도의 질문을 시행하지 않습니다.

\subsection{격조사}
\Gram{
  word={(격조사) -이/가},
  desc={선행 명사의 모음이 상승하는지도 확인(예: /속+이/ → [쇠기])}
}

\Gram{
  word={(격조사) -을/를}
}

\Gram{
  word={(격조사) -에게/게}
}

\Gram{
  word={(격조사) -에}
}

\Gram{
  word={(격조사) -에서}
}

\Gram{
  word={(격조사) -(으)로}
}

\Gram{
  word={(격조사) -와/과}
}

\Gram{
  word={(격조사) -보다}
}

\Gram{
  word={(격조사) -처럼}
}

\Gram{
  word={(격조사) -만큼}
}

\Gram{
  word={(격조사) -더러, -보고}
}

\Gram{
  word={(격조사) -(이)랑}
}

\Gram{
  word={(격조사) -커녕}
}

\Gram{
  word={(격조사) -아/야}
}

\newpage
\subsection{보조사}
\Gram{ word={(보조사) -은/는} }
\Gram{ word={(보조사) -만} }
\Gram{ word={(보조사) -도} }
\Gram{ word={(보조사) -마다} }
\Gram{ word={(보조사) -부터} }
\Gram{ word={(보조사) -까지} }
\Gram{ word={(보조사) -조차} }
\Gram{ word={(보조사) -(이)야} }
\Gram{ word={(보조사) -(이)라도} }
\Gram{ word={(보조사) -밖에} }
\Gram{ word={(보조사) -가지고} }

\subsection{문장 뒤 조사}
\Gram{ word={(간접인용) -고} }
\Gram{ word={(높임) -요} }


\section{종결어미}
※ 표준어와 다른 높임 체계(허씨요체, 허소체)와 표준어에 없는 어미(`-게라/라우', `-게', `-이-' 등)의 출현이 예상됩니다. \\
※ 표준어적 방책으로 [+격식성] 자질을 드러낸다는 연구가 있으니, 질문에 참고하기 바랍니다. \\
※ 친족어 조사와 연계하여 질문할 수 있습니다.

\newpage
\subsection{어린 아이에게}
\Gram{
  word={(해라, 의문) -니},
  qstn={어린 아이랑 외출을 하기로 했는데 밖에 날씨가 어떤지 아이에게 묻고 싶을 때 뭐라고 말씀하세요?}
}

\Gram{
  word={(해라, 청유) -자},
  qstn={아이가 밖을 보더니 비가 와서 나가기 싫다고 합니다. 그래도 오늘 산책은 해야 하는데 아이에게 같이 나갈 것을 권유할 때 뭐라고 말씀하세요?}
}

\Gram{
  word={(해라, 명령) -아라/어라},
  qstn={아이가 알겠다고 말합니다. 이제 나갈 준비를 해야 하는데, 아이에게 우비를 입고 장화를 신을 것을 말할 때 어떻게 말씀하세요?}
}

\Gram{
  word={(해라, 서술1) -ㄴ다},
  qstn={장모가 함께 식사를 하는 사위에게 오늘 술을 마실 건지 물어볼 때 뭐라고 말하나요?}
}

\subsection{장모가 사위에게}
\Gram{
  word={(하게, 의문) -나},
  qstn={장모가 함께 식사를 하는 사위에게 오늘 술을 마실 건지 물어볼 때 뭐라고 말하나요?}
}

\Gram{
  word={(하게, 청유) -세},
  qstn={사위가 술을 받아들었습니다. 술을 마시고 밥도 먹을 것을 말할 때 어떻게 말씀하시나요?}
}

\Gram{
  word={(하게, 명령) -게},
  qstn={사위가 술을 받아들었습니다. 술을 마시고 밥도 먹을 것을 말할 때 어떻게 말씀하시나요?}
}

\Gram{
  word={(하게, 서술1) -네},
  qstn={오랜만에 사위와 함께 식사하니 기분이 좋은 것을 말할 때 어떻게 말씀하시나요?}
}

\newpage
\subsection{사위가 장인·장모에게}
\Gram{
  word={(하십시오, 의문) -습니까, -나요},
  qstn={사위가 장모님 칠순여행을 어디로 가고 싶으신지 여쭤볼 때는 뭐라고 말하나요?}
}

\Gram{
  word={(하십시오, 청유) -십니다, -세요},
  qstn={장모님께서 여행 같은 건 안 가도 된다고 말씀하십니다. 그래도 칠순이시니 제주도에 함께 가자고 권유할 때는 뭐라고 말하나요?}
}

\Gram{
  word={(하십시오, 명령) -(으)십시오, -세요},
  qstn={장모님께서 고민해보겠다며 자리를 뜨려고 하십니다. 다시 앉으실 것을 어떻게 말하나요?}
}

\Gram{
  word={(하십시오, 서술1) -습니다, -네요},
  qstn={사위는 장모님의 기분을 전환하려 날씨 이야기를 합니다. 오늘 날씨가 좋다는 말을 어떻게 하는 게 좋을까요?}
}

\subsection{손아래 동서가 손위 동서에게}
\Gram{
  word={(하오, 의문) -오},
  qstn={손아래 동서가 손위 동서에게 어머님 칠순여행에 함께 갈 것인지 물을 때 어떻게 말하나요?}
}

\Gram{
  word={(하오, 청유) -(으)오},
  qstn={손위 동서는 아직 고민 중이라고 합니다. 함께 갈 것을 권유할 때는 어떻게 말하나요?}
}

\Gram{
  word={(하오, 명령) -(으)오},
  qstn={손위 동서가 고민해보겠다며 자리를 뜨려고 합니다. 다시 앉을 것을 어떻게 말하나요?}
}

\Gram{
  word={(하오, 서술1) -오},
  qstn={손아래 동서는 손위 동서의 기분을 전환하려고 날씨 이야기를 꺼냅니다. 오늘 날씨가 좋은 것을 어떻게 말하는 게 좋을까요?}
}

\newpage
\subsection{할머니가 또래 할머니에게}
\Gram{
  word={(하셔, 의문) -ㄴ가},
  qstn={할머니가 또래 할머니에게 요즘 잘 지내는지 물어볼 때 뭐라고 말하나요?}
}

\Gram{
  word={(하셔, 청유) -세},
  qstn={상대 할머니께서 요즘 자식 문제로 좀 힘들다고 말씀하셨습니다. 그럼 함께 차라도 한 잔 할 것을 권유할 때 뭐라고 말하나요?}
}

\Gram{
  word={(하셔, 명령) -셔},
  qstn={두 분이 함께 찻집에 갔습니다. 상대 할머니에게 자리에 앉을 것을 어떻게 말하나요?}
}

\Gram{
  word={(하셔, 서술1) -네},
  qstn={주문한 차가 나왔습니다. 상대 할머니에게 차가 따뜻한 것을 말할 때 어떻게 말하나요?}
}

\subsection{평서형 종결어미 (2)}
\Gram{
  word={(해라, 서술2) -(이)다},
  qstn={(어린 아이에게) 내일이 장날이라고 알려줄 때 뭐라고 말하나요?}
}

\Gram{
  word={(하게, 서술2) -(이)네, -일세},
  qstn={(장모가 사위에게) 내일이 장날이라고 알려줄 때 뭐라고 말하나요?}
}

\Gram{
  word={(하십시오, 서술2) -입니다, -이지요},
  qstn={(사위가 장인·장모에게) 내일이 장날이라고 알려줄 때 뭐라고 말하나요?}
}

\Gram{
  word={(하오, 서술2) -이오},
  qstn={(손아래 동서가 손위 동서에게) 내일이 장날이라고 알려줄 때 뭐라고 말하나요?}
}

\Gram{
  word={(하셔, 서술2) -이오},
  qstn={(할머니가 또래 할머니에게) 내일이 장날이라고 알려줄 때 뭐라고 말하나요?}
}

\newpage
\subsection{의문형 종결어미 (2)}
\Gram{
  word={(해라, 의문2) -(이)니},
  qstn={(어린 아이에게) 내일이 장날이냐고 물어볼 때 뭐라고 말하나요?}
}

\Gram{
  word={(하게, 의문2) -인가},
  qstn={(장모가 사위에게) 내일이 장날이냐고 물어볼 때 뭐라고 말하나요?}
}

\Gram{
  word={(하십시오, 의문2) -입니까},
  qstn={(사위가 장인·장모에게) 내일이 장날이냐고 물어볼 때 뭐라고 말하나요?}
}

\Gram{
  word={(하오, 의문2) -이오},
  qstn={(손아래 동서가 손위 동서에게) 내일이 장날이냐고 물어볼 때 뭐라고 말하나요?}
}

\Gram{
  word={(하셔, 의문2) -인가},
  qstn={(할머니가 또래 할머니에게) 내일이 장날이냐고 물어볼 때 뭐라고 말하나요?}
}

\subsection{의문형 종결어미 (3)}
\Gram{
  word={(해라, 의문3) -아},
  qstn={(어린 아이에게) 배가 고픈지 물어볼 때 뭐라고 말하나요?}
}

\Gram{
  word={(하게, 의문3) -가},
  qstn={(장모가 사위에게) 배가 고픈지 물어볼 때 뭐라고 말하나요?}
}

\Gram{
  word={(하십시오, 의문3) -십니까},
  qstn={(사위가 장인·장모에게) 배가 고픈지 물어볼 때 뭐라고 말하나요?}
}

\Gram{
  word={(하오, 의문3) -오},
  qstn={(손아래 동서가 손위 동서에게) 배가 고픈지 물어볼 때 뭐라고 말하나요?}
}

\Gram{
  word={(하셔, 의문3) -오},
  qstn={(할머니가 또래 할머니에게) 배가 고픈지 물어볼 때 뭐라고 말하나요?}
}

\newpage
\subsection{의문형 종결어미 (4)}
\Gram{
  word={(해라, 의문4) -지},
  qstn={(어린 아이에게) 밥을 먹을 것을 재차 확인하며 묻고 싶을 때 뭐라고 말하나요?}
}

\Gram{
  word={(하게, 의문4) -가},
  qstn={(장모가 사위에게) 밥을 먹을 것을 재차 확인하며 묻고 싶을 때 뭐라고 말하나요?}
}

\Gram{
  word={(하십시오, 의문4) -겠습니까},
  qstn={(사위가 장인·장모에게) 밥을 먹을 것을 재차 확인하며 묻고 싶을 때 뭐라고 말하나요?}
}

\Gram{
  word={(하오, 의문4) -오},
  qstn={(손아래 동서가 손위 동서에게) 밥을 먹을 것을 재차 확인하며 묻고 싶을 때 뭐라고 말하나요?}
}

\Gram{
  word={(하셔, 의문4) -오},
  qstn={(할머니가 또래 할머니에게) 밥을 먹을 것을 재차 확인하며 묻고 싶을 때 뭐라고 말하나요?}
}


\section{연결어미}
※ 조사 과정에서 자연스럽게 출현한 형태만을 기록하고, 별도의 질문을 시행하지 않습니다. \\

\Gram{ word={(연결어미) -고, -고서} }
\Gram{ word={(연결어미) -(으)면서} }
\Gram{ word={(연결어미) -아/어, -아서/어서} }
\Gram{ word={(연결어미) -(으)니, -(으)니까} }
\Gram{ word={(연결어미) -관데} }
\Gram{ word={(연결어미) -다가} }
\Gram{ word={(연결어미) -거든} }
\Gram{ word={(연결어미) -거든} }
\Gram{ word={(연결어미) -더라도} }
\Gram{ word={(연결어미) -(으)려고} }
\Gram{ word={(연결어미) -도록} }
\Gram{ word={(연결어미) -듯이} }
\Gram{ word={(연결어미) -지} }


\newpage
\section{시제}
\Gram{
  word={(시제) -는/ㄴ-},
  qstn={집 안에 있다가 창문을 열었는데 비가 오고 바람이 많이 불어요. 그때 밖의 날씨를 집 안에 있는 사람에게 말하는 것처럼 설명해보시겠어요? → `비가 온다', `바람이 분다' 등}
}

\Gram{
  word={(시제) -고 있-},
  qstn={%
  새들이 하늘에서 뭐를 하나요? → `날고 있다'}
}

\Gram{
  word={(시제) -았/었-},
  qstn={인사치레로 식사했는지를 어떻게 물어보나요? → `먹었니' 등}
}

\Gram{
  word={(시제) -았었/었었-},
  qstn={다른 지역으로 여행을 간 적이 있으세요? → `갔었다'}
}

\Gram{
  word={(시제) -더-},
  qstn={키가 큰 손자를 보고 와서, 손자가 키가 많이 컸다는 얘기를 다른 할머니한테 전할 거예요. 뭐라고 말하시겠어요? → `컸더라'}
}


\section{관형형}
※ 조사 과정에서 자연스럽게 출현한 형태만을 기록하고, 별도의 질문을 시행하지 않습니다. \\

\Gram{ word={(동사 관형형) 입는} }
\Gram{ word={(동사 관형형) 입었던} }
\Gram{ word={(동사 관형형) 입은} }
\Gram{ word={(동사 관형형) 입을} }
\Gram{ word={(형용사 관형형) 큰} }
\Gram{ word={(형용사 관형형) 크던} }
\Gram{ word={(형용사 관형형) 컸던} }


\newpage
\section{부정}
\Gram{
  word={안 먹었어, 먹지 않았어 (1)},
  qstn={%
    이웃이 밥 먹었느냐고 물었을 때 그렇지 않다고 대답하려면 어떻게 말하나요? \\
    (아니, 아직 안 먹었어. / 아니, 아직 먹지 않았어.)
  }
}

\Gram{
  word={안 먹었어, 먹지 않았어 (2)},
  qstn={%
    밥 먹을 시간이 지났는데도 아직 밥을 안 먹은 이유를 이웃이 나에게 물어볼 때는 어떻게 말하나요? \\
    (왜 아직도 밥을 안 먹었어? / 왜 아직도 안 밥 먹었어? / 왜 아직도 밥을 먹지 않았어?)
  }
}

\Gram{
  word={안 좋아, 좋지 않아},
  qstn={%
    이웃이 오늘 날씨가 좋으냐고 물었을 때 그렇지 않다고 대답하려면 어떻게 말하나요? \\
    (날씨가 안 좋다. / 날씨가 좋지 않다.)
  }
}

\Gram{
  word={안 깨끗해, 깨끗하지 않아},
  qstn={%
    이웃이 밭일을 하다 말고 왔는지 옷에 흙먼지가 묻어있네요. 이때 이웃이 자신의 옷이 깨끗하냐고 묻습니다. 어떻게 답하나요? \\
    (아니, 안 깨끗해. / 아니, 깨끗하지 않아. / 아니, 깨끗 안 해.)
  }
}

\Gram{
  word={안 갔어, 가지 않았어},
  qstn={%
    이웃이 아들이 장가 갔는지를 물었을 때 그렇지 않다고 대답하려면 어떻게 말하나요? \\
    (아직 장가 안 갔다. / 아직 안 장가갔다. / 아직 장가가지 않았다.)
  }
}

\Gram{
  word={안 만나 보았다, 만나 보지 않았다},
  qstn={%
    이웃이 안타까워하며 전에 한번 만나 보라고 한 사람을 만나 봤느냐고 물었을 때, 그렇지 않다고 대답하려면 어떻게 말하나요? \\
    (아직 안 만나 보았다. / 아직 만나 보지 않았다. / 아직 만나 안 보았다.)
  }
}

\Gram{
  word={먹도 않고},
  qstn={%
    이웃이 집에 있는 아기가 밥을 먹었냐고 물었을 때, 그렇지 않고 종일 잠만 자는 상황이라고 대답하려면 어떻게 말하나요? \\
    (먹지도 않고 잠만 잔다.)
  }
}

\Gram{
  word={못 마신다, 마시지 못한다},
  qstn={%
  술을 마실 줄 아느냐고 물었을 때 그렇지 않다고 대답하려면 어떻게 말하나요? \\
  (술을 못 마신다. / 술을 마시지 못한다.)
  }
}


\newpage
\section{사동표현과 피동표현}
※ 관련 질문은 <어휘·음운편>의 심화질문란을 참조하십시오. \\

\subsection{사동표현}
\Gram{ word={(사동표현) 살리다} }
\Gram{ word={(사동표현) 늘리다} }
\Gram{ word={(사동표현) 말리다[乾]} }
\Gram{ word={(사동표현) 말리다[挽]} }
\Gram{ word={(사동표현) 얼리다} }
\Gram{ word={(사동표현) 녹이다} }
\Gram{ word={(사동표현) 신기다} }
\Gram{ word={(사동표현) 입히다} }
\Gram{ word={(사동표현) 앉히다} }

\Gram{
  word={(사동표현) 죽이다},
  desc={V1 /ㅜ/ → /ㅟ/}
}

\subsection{피동표현}
\Gram{ word={(피동표현) 잡히다} }
\Gram{ word={(피동표현) 깎이다} }
\Gram{ word={(피동표현) 끼이다} }
\Gram{ word={(피동표현) 떼이다} }
\Gram{ word={(피동표현) 끊기다} }
\Gram{ word={(피동표현) 채이다} }
\Gram{ word={(피동표현) 실리다} }
\Gram{ word={(피동표현) 바뀌다} }

\Gram{
  word={(피동표현) 업히다},
  desc={V1 /ㅓ/ → /ㅔ/}
}


\newpage
\section{보조용언}
※ 조사 과정에서 자연스럽게 출현한 형태만을 기록하고, 별도의 질문을 시행하지 않습니다. \\

\Gram{ word={(보조용언) 싶다} }
\Gram{ word={(보조용언) 보다} }
\Gram{ word={(보조용언) 버리다} }
\Gram{ word={(보조용언) 대다} }
\Gram{ word={(보조용언) -나/는가보다} }

\vspace*{14cm}