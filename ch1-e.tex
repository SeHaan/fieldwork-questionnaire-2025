%!TEX root = main.tex
\section{사회적 지식}
\subsection{일러두기}
일러둘 내용을 적습니다.

\subsection{자유발화 질문}
\begin{itemize}[noitemsep]
  \item 벌교 밖으로 여행 가보신 경험 있으세요? (아래 문법 유도질문과 연계)
  \item 자주 가시는 시장이 있나요? 그 시장에서는 어떤 걸 파나요? 
  \item 벌교에서만 파는 특별한 물건들이 있나요?
\end{itemize}

\subsection{문법 유도질문}
※ 이전 발화에서 포착되지 않은 조사를 유도하기 위해 노력합니다.
또한 7{\textasciitilde}8번 문장에서는 [싶다], [보다] 등의 보조용언을 포착하는 것이 목표입니다.

\begin{enumerate}[noitemsep]
  \item 전에 여행 가보신 곳 있으세요?
  \item 어디 가셨나요?
  \item 어떤 일로 가셨나요?
  \item 누구 데리고 가셨나요?
  \item 누가 먼저 여행을 제안하셨나요?
  \item 여행 계획은 어떻게 세우셨나요?
  \item 언젠가 여행 가길 원하는 곳 있으세요?
  \item 그곳에서 하길 원하는 것 있으세요?
\end{enumerate}

\subsection{목표 어휘}
% \Entry 부분을 수정하시면 됩니다.
% word: 표준어형
% pred: 예상형태
% feat: 유의점
%       feat에 들어가는 인자는 쉼표(,)로 구분되며,
%       ms, sp, dp 세 종류가 있습니다.
%       feat에 들어간 인자가 볼드체로 표시됩니다.
% desc: 설명
% qstn: 유도질문
% advq: 심화질문
%
% 그리고 줄바꿈이 필요할 땐 엔터가 아니라 '\\'를 사용해주세요(중요!!).

\Entry{
  word={마을},
  pred={동네/마실},
  feat={ms,sp,dp},
  imag={img/refs/마을},
  desc={/\jamoword{m@/z@rbq/}/ > /마을/},
  qstn={지금 사시는 곳 or 예전에 살아오셨던 곳을 소개해주세요. \\ 집집들이 모여서 서로 돕고 사는 곳을 뭐라고 부르나요?},
  advq={옆 동네/옆 마을과는 어떤 관계인가요?}
}

\Entry{
  word={외국},
  pred={},
  feat={ms,sp,dp},
  imag={img/refs/외국},
  desc={V1 /ㅚ/},
  qstn={우리나라 말고 다른 곳으로 여행 가보신 적 있으신가요? / 어디를 가고 싶으신가요? / 우리나라가 아닌 다른 나라를 무엇이라고 부르나요?},
  advq={다른 가족분들이 외국에 대해 해주신 이야기가 있을까요?}
}

\Entry{
  word={어디},
  pred={위디, 워디, 어디서:, 어서, 어이서, 워이서, 워디서},
  feat={ms,sp,dp},
  desc={},
  qstn={친구분이 급하게 막 가고 있을 때, 어떤 곳을 향해 가는지 어떻게 물어보시나요? / 어떤 곳에서 왔는지는 어떻게 물어보나요?},
  advq={어린 아이라면 어떻게 물어보시겠어요? \\ 어머니 혹은 아버지라면 어떻게 물어보시겠어요?}
}

\Entry{
  word={모시다/모셔라},
  pred={메시다. 모시고, 모셔야},
  feat={ms,sp,dp},
  imag={img/refs/모시다-모셔라},
  desc={},
  qstn={자녀분들이 성인이 되고 나서, 혹은 성인이 되고 나신 후 부모님과 함께 여행하신 적이 있으신가요? / 그분들과 어떻게 여행을 다니셨나요?},
  advq={지금 자녀분들과 함께, 혹은 과거로 돌아가 부모님과 함께 가고 싶은 여행지가 있으신가요?}
}

\Entry{
  word={하나, 둘, ... 열},
  pred={하나, 둘:, 싯/스이/셋:, 닛/느이/넷:, 다섯, 여섯, 일곱, 여덜, 아홉, 열:},
  feat={ms,sp,dp},
  desc={},
  qstn={물건을 세듯이 하나부터 열까지만 천천히 숫자를 세주시겠어요?},
  advq={너희 중 하나가 밥을 먹느냐는 질문에, /하나/의 자리에 다른 숫자를 넣어서 천천히 한 번 더 말씀해 주시겠어요?}
}

\Entry{
  word={스물, 서른, ... 아흔},
  pred={스물, 서른, 마은/마흔, 쉰/시운, 예순, 이른, 여든, 아은/아흔},
  feat={ms,sp,dp},
  desc={/마\jamoword{zvn/}/ 관련 이형태 \\ /스믈ㅎ.../ > /스물//셜흔.../ > /서른/ \\ /려쉰.../ > /예순//닐흔.../ > /일흔/},
  qstn={이번에는 스물, 서른 하는 식으로 백까지만 천천히 숫자를 세주시겠어요?},
  advq={내 나이가 열이다라는 문장에, /열/의 자리에 스물, 서른 등의 숫자를 넣어서 천천히 한 번 더 말씀해주시겠어요?}
}

\Entry{
  word={에누리},
  pred={에느리},
  feat={ms,sp,dp},
  desc={},
  qstn={시장에서 물건을 사실 때, 무조건 써진 가격으로만 사야 하나요? \\ 조금 싸게 사려고 가격을 깎는 것을 무엇이라고 하나요?},
  advq={에누리를 잘하려면 어떻게 해야 하나요?}
}

\Entry{
  word={되},
  pred={데/되(단모음)},
  feat={ms,sp,dp},
  imag={img/refs/되},
  desc={V1 /ㅚ/},
  qstn={시장에서 콩이나 팥 같은 것을 어떤 단위로 파나요? \\ 열 홉이 모이면 무엇이 되나요?},
  advq={한 되씩 혹은 두세 되씩 파는 물건에는 또 무엇이 있나요?}
}

\Entry{
  word={말(단위)},
  pred={말},
  feat={ms,sp,dp},
  imag={img/refs/말(단위)},
  desc={/말/(동물), /말/(언어)와의 성조 음장 차이},
  qstn={시장에서 기름을 짜기 위한 들깨 같은 것을 어떤 단위로 파나요? \\ 열 되가 모이면 무엇이 되나요?},
  advq={한 말씩 혹은 두세 말씩 파는 물건에는 또 무엇이 있나요?}
}

\Entry{
  word={배(연산)},
  pred={},
  feat={ms,sp,dp},
  desc={/배/(과일), /배/(신체)와의 성조 음장 차이},
  qstn={한 말은 한 되의 열 무엇이라고 하시나요? \\ 곱절을 다른 말로 무엇이라고 하시나요?},
  advq={n 배가 관찰된 경우) 혹시 n을 빼고 발음하면 어떻게 되나요?}
}

\Entry{
  word={금(금속)},
  pred={},
  feat={ms,sp,dp},
  imag={img/refs/금(금속)},
  desc={/금긋다/의 /금/과의 성조 음장 차이},
  qstn={시장에서는 보석도 파나요? / 비싼 보석이나 반지, 목걸이는 어디서 파나요? / 결혼하실 때 그곳에서 반지(가락지)를 맞추셨나요?},
  advq={금덩어리를 정해진 양으로 네모낳게 만든 것을 무엇이라고 하나요?}
}