%!TEX root = main.tex
\section{바다}
\subsection{일러두기}
일러둘 내용을 적습니다.

\subsection{자유발화 질문}
자유발화 질문을 적습니다.

\subsection{목표 어휘}
% \Entry 부분을 수정하시면 됩니다.
% word: 표준어형
% pred: 예상형태
% feat: 유의점
%       feat에 들어가는 인자는 쉼표(,)로 구분되며,
%       ms, sp, dp 세 종류가 있습니다.
%       feat에 들어간 인자가 볼드체로 표시됩니다.
% desc: 설명
% qstn: 유도질문
% advq: 심화질문
%
% 그리고 줄바꿈이 필요할 땐 엔터가 아니라 '\\'를 사용해주세요(중요!!).

\Entry{
  word={어부},
  pred={},
  feat={ms,sp,dp},
  imag={img/refs/어부},
  desc={},
  qstn={생선이나 그런 해산물을 잡아서 파시는 분들을 뭐라고 부르나요?},
  advq={이런 분들이 모여서 사는 곳이 있나요? 어떤 곳인가요?}
}

\Entry{
  word={그물},
  pred={},
  feat={ms,sp,dp},
  imag={img/refs/그물},
  desc={},
  qstn={물고기를 잡는 방법에는 어떤 것이 있을까요? \\ 요즘 전어가 철인데, 전어를 어떻게 잡는지 아시나요?},
  advq={그물은 배 위에서 던지는 방법만 쓰나요? 아니면 물고기나 다른 해산물들을 잡을 수 있는 여러 방법이 있나요?}
}

\Entry{
  word={아가미},
  pred={아가미},
  feat={ms,sp,dp},
  imag={img/refs/아가미},
  desc={},
  qstn={생선 손질해 보신 적 있으신가요? 어떻게 손질하나요? \\ 물고기가 물속에서 숨을 쉴 수 있게 해주는 부위를 무엇이라고 하나요?},
  advq={게, 새우가 이미 관찰되었을 시) 물고기 말고 게나 새우에도 아가미나 비슷한 역할을 하는 부위가 있나요? 그 부분을 무엇이라고 부르나요?}
}

\Entry{
  word={지느러미},
  pred={지느레미},
  feat={ms,sp,dp},
  imag={img/refs/지느러미},
  desc={},
  qstn={생선 손질해 보신 적 있으신가요? 어떻게 손질하나요? 물고기의 꼬리에서 파닥거려서 헤엄칠 수 있게 하는 부위를 무엇이라고 하나요?},
  advq={꼬리 말고 등이나 옆면에 달린 비슷한 부위도 지느러미라고 하나요?}
}

\Entry{
  word={바닷물},
  pred={},
  feat={ms,sp,dp},
  imag={img/refs/바닷물},
  desc={},
  qstn={바다는 개울물과 다르게 짠데, 그 짠 물을 무엇이라고 부르나요?},
  advq={짠 물은 모두 그렇게 부르나요? 아니면 바다에 있는 짠물만 그렇게 부르나요?}
}

\Entry{
  word={파랗다},
  pred={},
  feat={ms,sp,dp},
  imag={img/refs/파랗다},
  desc={},
  qstn={날씨가 좋은 날, 바다나 하늘의 색깔을 어떻다고 말씀하시나요?},
  advq={바다의 색깔을 표현하는 다른 말도 있을까요?}
}

\Entry{
  word={밀물},
  pred={},
  feat={ms,sp,dp},
  imag={img/refs/밀물},
  desc={},
  qstn={서산 바다는 물이 들어왔다가, 나갔다가 하잖아요. 이렇게 들어오는 물이나 물이 들어오는 그때를 무엇이라고 부르시나요?},
  advq={날에 따라서 물이 들어오는 정도가 다른가요? 많이 들어오는 날은 무엇이라고 부르나요?}
}

\Entry{
  word={썰물},
  pred={},
  feat={ms,sp,dp},
  imag={img/refs/썰물},
  desc={/혈물/ > /썰물/},
  qstn={서산 바다는 물이 들어왔다가, 나갔다가 하잖아요. 이렇게 나가는 물이나 물이 빠지는 그때를 무엇이라고 부르시나요?},
  advq={날에 따라서 물이 빠지는 정도가 다른가요? 많이 빠지는 날은 무엇이라고 부르나요?}
}

\Entry{
  word={갯벌},
  pred={},
  feat={ms,sp,dp},
  imag={img/refs/갯벌},
  desc={개펄과의 의미 차이 여부},
  qstn={조개나 소라를 잡아보신 적이 있으신가요? 어떻게 잡으셨나요? \\ 바다에 물이 싹 빠지고 드러난 질퍽질퍽한 땅을 무엇이라고 부르나요?},
  advq={갯벌에서는 어떤 해산물을 잡을 수 있나요? \\ 갯벌 일을 전문적으로 하시는 분들을 부르는 이름이 있나요?}
}

\Entry{
  word={곶},
  pred={},
  feat={ms,sp,dp},
  imag={img/refs/곶},
  desc={기저 종성 /ㅈ/},
  qstn={황금산과 몽돌해변은 어느 동네에 있나요? (독곶, 독곶리) \\ 독곶리처럼, 바다 쪽으로, 뾰족하게 뻗은 육지를 무엇이라고 하나요?},
  advq={독곶리를 부르는 다른 이름은 없나요? \\ 옛날에는 그 지역을 뭐라고 불렀나요?}
}

\Entry{
  word={만},
  pred={},
  feat={ms,sp,dp},
  imag={img/refs/만},
  desc={},
  qstn={서산 북쪽 바다(가로림만)와 남쪽 바다(천수만)를 무엇으로 부르나요? \\ 바다 쪽에서 육지 속으로 파고들어 와 있는 곳을 무엇이라고 하나요?},
  advq={가로림만이나 천수만을 부르는 다른 이름은 없나요? \\ 옛날에는 그 바다를 뭐라고 불렀나요?}
}

\Entry{
  word={통발},
  pred={},
  feat={ms,sp,dp},
  imag={img/refs/통발},
  desc={},
  qstn={물고기나 다른 해산물을 잡을 때, 어떤 방법들이 있나요? \\ 10월이 꽃게 철이었는데, 꽃게잡이는 어떻게 하는지 아시나요?},
  advq={통발에는 주로 무엇이 잡히나요? \\ 통발 말고 물고기를 가두어서 잡는 방법은 또 없나요?}
}

\Entry{
  word={새우},
  pred={물쌔우, 새우, 대화, 새우, 새오},
  feat={ms,sp,dp},
  imag={img/refs/새우},
  desc={민물 or 바다, 크기에 따른 어휘 차이 \\ /사ᄫᅵ/ > ???},
  qstn={얼마 전에 철이었는데, 가재랑 닮았고 등이 굽어있는 해산물을 뭐라고 부르나요? 크기에 따라서 먹는 방법이 조금씩 다릅니다.},
  advq={큰 것과 작은 것을 부르는 이름이 다른가요? 새우는 큰 것만 가리키나요 작은 것만 가리키나요?}
}

\Entry{
  word={게},
  pred={그이},
  feat={ms,sp,dp},
  imag={img/refs/게},
  desc={V1 /ㅐ-ㅔ/ 대립},
  qstn={서산의 유명한 해산물에는 뭐가 있나요? \\ 집게발을 가졌고, 옆으로 걸어다니는 해산물을 무엇이라고 부르나요?},
  advq={게를 사용한 맛있는 요리를 소개해 주실 수 있나요? \\ 게는 어떻게 손질하나요?}
}

\Entry{
  word={바위},
  pred={},
  feat={ms,sp,dp},
  imag={img/refs/바위},
  desc={V2 /ㅟ/ \\ 바회, 바히... > 바위},
  qstn={꽃게 말고, 갯벌에서 보이는 작은 게들은 보통 어디에 사나요? / 고둥이나 따개비는 어디에 붙어서 사나요? / 큰 돌을 뭐라고 부르나요?},
  advq={바다에 잠겨있어서 보일락말락 하는 커다란 돌도 바위라고 부르나요?}
}